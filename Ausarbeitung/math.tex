%!TEX root = main.tex

\mathtoolsset{showonlyrefs=true}

%%% Zahlenmengen
% Abkürzungen für die oft benötigten Zahlenmengen, z.B. \IZ für das ganze
% Zahlen Z etc.
\newcommand{\bbR}{\ensuremath{\mathbb{R}}}
\newcommand{\bbN}{\ensuremath{\mathbb{N}}}
\newcommand{\bbP}{\ensuremath{\mathbb{P}}}

% griechisches Alphabet
\renewcommand{\epsilon}{\varepsilon}
\renewcommand{\theta}{\vartheta}

%%% Operatoren und Funktionen
\DeclarePairedDelimiter{\abs}{\lvert}{\rvert}
\DeclarePairedDelimiter{\norm}{\lVert}{\rVert}

\DeclarePairedDelimiter{\ceil}{\lceil}{\rceil}
\DeclarePairedDelimiter{\floor}{\lfloor}{\rfloor}

\newcommand{\skprod}[2]{\left\langle#1,#2\right\rangle}
\newcommand{\fracpart}[2]{\frac{\partial#1}{\partial#2}}

\newcommand{\Transp}{^{\mathrm T}}
\newcommand{\Stern}{^{*}}
\newcommand{\Int}[1]{#1^\circ}
\newcommand{\Ext}[1]{\overline{#1}}

\DeclareMathOperator{\spn}{span}
\DeclareMathOperator{\ee}{e}
\DeclareMathOperator{\ii}{i}

\newcommand{\grad}{\nabla}
\newcommand{\hesse}{\nabla^2}

\newcommand{\fa}{\text{für alle}~}

% fetter Vektor
\renewcommand{\v}[1]{\mathbf{#1}}

% ntheorem

\theorempreskipamount 1.5em
\theoremnumbering{arabic}
% \theoremstyle{break}
\theorembodyfont{\itshape}
\theoremheaderfont{\normalfont\bfseries}
\theoremseparator{}
\renewtheorem{Satz}{Satz}
\renewtheorem{satz}[Satz]{Satz}
\renewtheorem{Theorem}[Satz]{Theorem}
\renewtheorem{theorem}[Satz]{Theorem}
\renewtheorem{Proposition}[Satz]{Proposition}
\renewtheorem{proposition}[Satz]{Proposition}
\renewtheorem{Lemma}[Satz]{Lemma}
\renewtheorem{lemma}[Satz]{Lemma}
\renewtheorem{Korollar}[Satz]{Korollar}
\renewtheorem{korollar}[Satz]{Korollar}

\theorembodyfont{\upshape}
\renewtheorem{Beispiel}[Satz]{Beispiel}
\renewtheorem{beispiel}[Satz]{Beispiel}
\renewtheorem{Bemerkung}[Satz]{Bemerkung}
\renewtheorem{bemerkung}[Satz]{Bemerkung}
\renewtheorem{Anmerkung}[Satz]{Anmerkung}
\renewtheorem{anmerkung}[Satz]{Anmerkung}
\renewtheorem{Definition}[Satz]{Definition}
\renewtheorem{definition}[Satz]{Definition}
\newtheorem{Algorithmus}[Satz]{Algorithmus}

% normales Latex-Theorem-Zeugs
% \newtheorem{satz}{Satz}
% \newthereom{lemma}[lemma]{Lemma}
% \newthereom{prop}[prop]{Proposition}
% \newthereom{koro}[koro]{Korollar}
