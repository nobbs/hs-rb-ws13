%!TEX root = main.tex

\section{Einleitung} % (fold)
\label{sec:einleitung}

Für die Reduzierte-Basis-Methode sind A-posteriori-Fehlerschätzer für das Ausgabefunktional $s(\mu)$ beziehungsweise für die Lösung $u(\mu)$, sowohl während der Offline- als auch während der Online-Phase, von großer Bedeutung.
Daraus ergeben sich einige Anforderungen an die Fehlerschätzer; diese sollten \emph{effizient}, \emph{zuverlässig} und \emph{präzise} sein.

Da der gegebene Parameterbereich $\mathcal D \subset \mathbb{R}^n$ im Allgemeinen keine diskrete Menge ist, können wir diesen mittels des vorgestellten Greedy-Algorithmus nicht vollständig ausschöpfen.
Stattdessen müssen wir uns mit diskreten Teilmengen $\Xi \subset \mathcal D$ zufrieden geben.
Berechnen wir nun für jedes $\mu \in \Xi$ die Finite-Elemente-Lösung um so den Fehler zu bestimmen, dann wird dies sehr aufwändig und impraktikabel.
Verwenden wir statt der exakten Fehlerbestimmung einen \emph{effizienten}, das heißt einfach und schnell berechenbaren, Fehlerschätzer, so können wir $\Xi$ deutlich umfangreicher wählen, und damit ein besseres Bild des Fehlerverhaltens über $\mathcal D$ erhalten.
Insbesondere sollte die Berechnung des Fehlerschätzers unabhängig von der Dimension $\mathcal N$ des zugrundeliegenden Finite-Elemente-Ansatzraumes $X$ sein.

Auch in der Online-Phase kommt der Fehlerschätzer zum Einsatz.
Die meisten $\mu \in \mathcal D$ werden in der Offline-Phase nicht abgedeckt; der Fehlerschätzer sollte also \emph{zuverlässig} sein, das heißt, auch für diese $\mu$ sinnvolle Fehlerschranken liefern.
Insbesondere darf der Fehlerschätzer den tatsächlichen Fehler nicht unterschätzen.
Gleichzeitig wollen wir aber auch, dass der Fehlerschätzer \emph{präzise}, also möglichst nahe am tatsächlichen Fehler, ist.

% section einleitung (end)
