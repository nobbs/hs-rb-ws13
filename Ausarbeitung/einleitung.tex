%!TEX root = main.tex

\section{Einleitung} % (fold)
\label{sec:einleitung}

Für die Reduzierte-Basis-Methode sind A-posteriori-Fehlerschätzer für das Ausgabefunktional $s$ beziehungsweise für die Lösung $u(\mu)$, sowohl während der Offline- als auch während der Online-Phase, von großer Bedeutung.
Daraus ergeben sich einige Anforderungen an die Fehlerschätzer; diese sollten effizient, zuverlässig und präzise sein.

Da der gegebene Parameterbereich $\mathcal D \subset \mathbb{R}^n$ im Allgemeinen keine diskrete Menge ist, können wir diesen mittels des vorgestellten Greedy-Algorithmus nicht vollständig ausschöpfen.
Stattdessen müssen wir uns mit diskreten Teilmengen $\Xi \subset \mathcal D$ zufrieden geben.
Die Verwendung eines effizienten, das heißt einfach und schnell berechenbaren, Fehlerschätzers, erlaubt es uns $\Xi$ umfangreicher zu wählen, und damit ein besseres Bild des Fehlerverhaltens über $\mathcal D$ zu erhalten.
Insbesondere sollte die Berechnung des Fehlerschätzers unabhängig von der Dimension $\mathcal N$ des zugrundeliegenden Galerkin-Ansatzraumes $X$ sein.

In der Online-Phase kann der Fehlerschätzer zur Verifikation der Lösung verwendet werden, wenn eine bestimmte Fehlertoleranz eingehalten werden soll.
Der Fehlerschätzer sollte deswegen zuverlässig sein, das heißt, er darf den tatsächlichen Fehler nicht unterschätzen.
Gleichzeitig wollen wir aber auch, dass der Fehlerschätzer präzise, also möglichst nahe am tatsächlichen Fehler, ist.

% section einleitung (end)
