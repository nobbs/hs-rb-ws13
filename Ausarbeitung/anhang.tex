%!TEX root = main.tex

\section{Benötigte Grundlagen} % (fold)
\label{sec:benoetigte_grundlagen}

\begin{Definition}[Hilbertraum]
    \label{def:hilbertraum}
    Wir nennen einen reellen Vektorraum $H$ mit einem Skalarprodukt $\skprod{\cdot}{\cdot}$, der bezüglich der durch das Skalarprodukt induzierten Norm $\norm{\cdot}_H = \sqrt{\skprod{\cdot}{\cdot}}$ vollständig ist, einen Hilbertraum.
\end{Definition}

% \begin{Lemma}
%     \label{lemma:endlich_dim_ur_von_hr_ist_hr}
%     Sei $(H, \skprod{\cdot}{\cdot})$ ein Hilbertraum und sei $W \subset H$ ein endlichdimensionaler Unterraum. Dann ist $(W, \skprod{\cdot}{\cdot})$ ebenfalls ein Hilbertraum.
% \end{Lemma}

\begin{Satz}[Rieszscher Darstellungssatz]
    \label{satz:rieszscher_darstellungssatz}
    Sei $(H, \skprod{\cdot}{\cdot})$ ein Hilbertraum. Dann existiert zu jedem stetigen linearen Funktional $f \colon H \to \bbR$ genau ein $y \in H$, sodass gilt
    \begin{equation}
        f(x) = \skprod{x}{y} \quad \fa x \in H,
    \end{equation}
    und
    \begin{equation}
         \norm{f}_{H'} := \sup_{x \in H} \frac{\abs{f(x)}}{\norm{x}_H} = \norm{y}_H.
    \end{equation}
\end{Satz}

% \begin{Definition}[Eigenschaften einer parametrischen Bilinearform]
%     \label{def:eigenschaften_einer_bilinearform}
%     Seien $X$ und $Y$ zwei Vektorräume, $\mathcal D \subset \bbR^n$ ein kompakter Parameterbereich und $a \colon X \times Y \times \mathcal D \to \bbR$ eine parametrische Bilinearform, also $a(\cdot, \cdot; \mu) \colon X \times Y \to \bbR$ eine Bilinearform für jedes $\mu \in \mathcal{D}$.

%     \begin{enumerate}[leftmargin=*]
%     \item\label{def:bilinearform:stetig}
%     Wir nennen $a$ \emph{stetig} über $X$ und $Y$, wenn
%     \begin{equation}
%         \gamma(\mu) = \sup_{w\in X} \sup_{v \in Y} \frac{a(w, v; \mu)}{\norm{w}_X \norm{v}_Y} < \infty
%     \end{equation}
%     für alle $\mu \in \mathcal{D}$ ist, und bezeichenen dann $\gamma_0 := \max_{\mu \in \mathcal{D}} \gamma(\mu) < \infty$ als \emph{Stetigkeitskonstante} von $a$.

%     \item\label{def:bilinearform:koerziv}
%     Gilt $X = Y$, dann nennen wir a \emph{koerziv} über $X$, falls
%     \begin{equation}
%         \alpha(\mu) = \inf_{w \in X} \frac{a(w, w; \mu)}{\norm{w}^2_X} > 0
%     \end{equation}
%     für alle $\mu \in \mathcal{D}$ gilt, und setzen $\alpha_0 := \min_{\mu \in \mathcal{D}} \alpha(\mu) > 0$, die sogenannte \emph{Koerzivitätskonstante} von $a$.

%     \item\label{def:bilinearform:inf_sup_stabil}
%     Wir definieren die sogenannte \emph{inf-sup-Konstante} $\beta(\mu)$ von $a$ als
%     \begin{equation}
%         \beta(\mu) = \inf_{w \in X} \sup_{v \in Y} \frac{a(w, v; \mu)}{\norm{w}_X \norm{v}_Y}.
%     \end{equation}
%     Es gilt stets $\beta(\mu) \geq 0$. Existiert zudem ein $\beta_0 > 0$, sodass $\beta(\mu) \geq \beta_0$, für alle $\mu \in \mathcal{D}$, gilt, dann nennen wir $a$ \emph{inf-sup-stabil}.

%     \item\label{def:bilinearform:parametrisch_affin}
%     Existieren ein $Q_f \in \bbN$, Abbildungen $\Theta_f^k \colon \mathcal{D} \to \bbR$, $1 \leq k \leq Q_f$, sowie Bilinearformen $a^k \colon X \times Y \to \bbR$, $1 \leq k \leq Q_f$, sodass
%     \begin{equation}
%         a(w, v; \mu) = \sum_{k = 1}^{Q_f} \Theta_f^k(\mu) a^k(w, v)
%     \end{equation}
%     für alle $w \in X$, $y \in Y$ und $\mu \in \mathcal{D}$ gilt, dann nennen wir $a$ \emph{parametrisch affin}.

%     \item\label{def:bilinearform:parametrisch_koerziv}
%     Gilt $X = Y$ und ist $a$ eine koerzive, parametrisch affine Bilinearform, für die $\Theta_f^k(\mu) > 0$, für alle $\mu \in \mathcal{D}$, $1 \leq k \leq Q_f$, sowie $c^k(w, w) \geq 0$, für alle $w \in X$, $1 \leq k \leq Q_f$, gilt, dann nennen wir $a$ \emph{parametrisch koerziv}.
%     \end{enumerate}
% \end{Definition}

% section ben_tigte_grundlagen (end)
