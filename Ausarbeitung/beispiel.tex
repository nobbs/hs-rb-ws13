%!TEX root = main.tex

\section{Numerische Beispiele} % (fold)
\label{sec:beispiel}

Der in diesem Abschnitt verwendete \texttt{MatLab}-Sourcecode kann unter \url{https://github.com/nobbs/hs-rb-ws13} eingesehen werden.

\subsection{Durchschnittliche und maximale Effektivität} % (fold)
\label{sub:durchschnittliche_und_maximale_effektivitaet}

Bevor wir uns den Beispielen widmen, wollen wir zunächst einige Maße einführen, mit denen man die Güte unserer Fehlerschätzer bewerten kann.
Dazu betrachten wir vor allem die durchschnittliche und die maximale Effektivität des Fehlerschätzers $\Delta^s_N$.

% Um Aussagen über die Güte der Fehlerschätzer zu treffen, ist es sinnvoll sich die durchschnittlichen beziehungsweise maximalen Effektivitäten anzusehen.
Sei dazu $\Xi_{\text{test}} \subset \mathcal D$ eine endliche Teilmenge des Parameterraumes mit $n_\text{test} \in \mathbb{N}$ Elementen.
Wir nennen, wobei $\bullet$ ein Platzhalter für \glqq{}$\text{en}$\grqq{}, \glqq{}$s$\grqq{}, \glqq{} \grqq{} sei,
\begin{equation}
    \eta^\bullet_{N,\max} := \max_{\mu \in \Xi_\text{test}} \eta^\bullet_N(\mu), \qquad
    \eta^\bullet_{N,\text{ave}} := \frac{1}{n_\text{test}} \sum_{\mu \in \Xi_\text{test}} \eta^\bullet_N(\mu)
\end{equation}
maximale respektive durchschnittliche Effektivität des Fehlerschätzers $\Delta^\bullet_N$ über $\Xi_\text{test}$.

Wir beschränken uns nun im Wesentlichen auf die Effektivität von $\Delta^s_N$. Korollar \ref{korollar:effektivitaeten} liefert uns
\begin{equation}
    \eta^s_{N,\max} \leq \max_{\mu \in \Xi_\text{test}} \frac{\gamma(\mu)}{\alpha_{\text{LB}}(\mu)} \leq \max_{\mu \in D} \frac{\gamma(\mu)}{\alpha_{\text{LB}}(\mu)} =: \eta^s_{\max,\text{UB}}.
\end{equation}
Diese obere Schranke ist sowohl unabhängig von der Dimension $N$ des Reduzierte-Basis-Ansatzraumes als auch von der Dimension $\mathcal N$ des Finite-Elemente-Ansatzraumes und bestärkt uns damit in der Wahl unserer Fehlerschätzer. Allerdings kann $\eta^s_{\max,\text{UB}}$ recht groß werden, da hierbei vom \emph{worst-case} ausgegangen wird.

Ein weiteres Maß für die Güte der Fehlerschätzer liefert der Quotient des maximalen geschätzten Fehlers und des maximalen echten Fehlers (statt wie oben das Maximum der Quotienten dieser Werte), gegeben durch
\begin{equation}
     \rho^s_{\text{err},N} := \frac{\max\limits_{\mu \in \Xi_\text{test}} \Delta_N^s(\mu)}{\max\limits_{\mu \in \Xi_\text{test}} (s(\mu) - s_N(\mu))}.
\end{equation}

% subsection durchschnittliche_und_maximale_effektivitaet   (end)

\subsection{Eindimensionaler Parameterraum} % (fold)
\label{sub:eindimensionaler_parameterraum}

% subsection eindimensionaler_parameterraum (end)

Wir betrachten nun das Thermal-Block-Beispiel aus [TODO:Quelle], allerdings nur für einen eindimensionalen Parameter.

Es sei $\Omega = [ 0, 1 ]^2 \subset \mathbb{R}^2$, für dieses Beispiel teilen wir das Gebiet in zwei Rechtecke $\Omega_1 = [0, \frac{1}{2}] \times [ 0, 1 ] $ und $\Omega_2 = [ \frac{1}{2}, 1 ] \times [ 0, 1 ]$ auf. Wir teilen den Rand von $\Omega$ auf in $\Gamma_1 = [0, 1] \times \{ 0 \}$, $\Gamma_2 = \{ 1 \} \times [0, 1]$, $\Gamma_1 = [0, 1] \times \{ 1 \}$ und $\Gamma_4 = \{ 0 \} \times [0, 1]$.

Sei der Parameterbereich gegeben durch $\mathcal D = [0.1, 10]$. Als Referenzparameter wählen wir $\bar \mu = 1$.
Das Variatonsproblem lautet nun:
\begin{addmargin}[2em]{2em}
Sei $\mu \in \mathcal D$. Gesucht ist ein $u(\mu) \in H^1(\Omega)$, sodass gilt
\begin{equation}
    \mu \int_{\Omega_1} \nabla u(\mu) \nabla v \, \mathrm d x + \int_{\Omega_2} \nabla u(\mu) \nabla v \, \mathrm d x = \int_{\Gamma_1} v \, \mathrm{d} x, \quad \fa v \in H^1(\Omega),
\end{equation}
mit den Neumann-Randbedingungen
\begin{equation}
    \frac{\partial u(\mu)}{\partial \nu} = 1
    \quad \text{auf}~ \Gamma_1,
    \qquad
    \frac{\partial u(\mu)}{\partial \nu} = 0
    \quad \text{auf} ~ \Gamma_2 ~\text{und}~ \Gamma_4,
\end{equation}
und der Dirichlet-Randbedingung
\begin{equation}
    u(\mu) = 0 \quad \text{auf} ~\Gamma_3.
\end{equation}
Wir interessieren uns nun für
\begin{equation}
    s(\mu) = \int_{\Gamma_1} u(\mu) \, \mathrm{d}x.
\end{equation}
\end{addmargin}

Wir schreiben dieses Variatonsproblem nun als
\begin{equation}
    a(w, v;\mu) = \Theta_a^1(\mu) a^1(w, v) + \Theta_a^2(\mu) a^2(w, v) = \Theta_f^1(\mu) f^1(v),
\end{equation}
wobei
\begin{align}
    a^1(w, v) &= \int_{\Omega_1} \nabla w \nabla v \, \mathrm{d}x,
    \quad
    &a^2(w, v) &= \int_{\Omega_2} \nabla w \nabla v \, \mathrm{d}x,
    \quad
    &f^1(v) &= \int_{\Gamma_1} v \, \mathrm{d}x,
    \\
    \Theta_a^1(\mu) &= \mu,
    &\Theta_a^2(\mu) &= 1,
    &\Theta_f^1(\mu) &= 1.
\end{align}

Damit erhalten wir für unsere untere Schranke der Koerzivitätskonstante
\begin{equation}
    \alpha_\text{LB}(\mu) = \min_{k = 1 \ldots 2} \frac{\Theta_a^k(\mu)}{\Theta_a^k(\bar \mu)} = \min \left(  \frac{\mu}{\bar \mu}, 1  \right)  = \min(\mu, 1).
\end{equation}

Analog zu $\alpha_\text{LB}$ aus Abschnitt \ref{sub:untere_schranke_f_r_die_koerzivit_tskonstante} lässt sich für die Stetigkeitskonstante eine obere Schranke bestimmen mittels
\begin{equation}
    \gamma(\mu) \leq \gamma_\text{UB}(\mu) := \Theta_a^{\max, \bar \mu}(\mu) = \max_{k = 1 \ldots 2} \frac{\Theta_a^k(\mu)}{\Theta_a^k(\bar \mu)} = \max(\mu, 1), \quad \fa \mu \in \mathcal D.
\end{equation}
Damit können wir nun $\eta^s_{\max,\text{UB}}$ nach oben abschätzen und erhalten für unsere Wahl von $\mathcal D$
\begin{equation}
    \eta^s_{\max,\text{UB}} = \max_{\mu \in D} \frac{\gamma(\mu)}{\alpha_{\text{LB}}(\mu)} \leq \max_{\mu \in D} \frac{\gamma_\text{UB}(\mu)}{\alpha_{\text{LB}}(\mu)} \leq 10.
\end{equation}

Wir verwenden $\Xi_\text{train} = \Xi_\text{test} \subset \mathcal D$ mit $10000$ Elementen, welche zufällig gleichverteilt aus $\mathcal D$ gewählt werden.
Als ersten Snapshot-Parameter wählen wir $\mu_1 = \bar \mu$, alle weiteren werden mittels Greedy-Verfahren bestimmt. Als Fehlertoleranz für $s(\mu) - s_N(\mu)$ wird $10^{-8}$ verwendet.

\begin{figure}[tb]
    \begin{center}
        \tiny
        \newlength\figureheight
        \newlength\figurewidth
        \setlength\figureheight{5cm}
        \setlength\figurewidth{0.4\textwidth}
        \begin{subfigure}[b]{0.45\textwidth}
            ~
            % % This file was created by matlab2tikz v0.4.6 running on MATLAB 8.1.
% Copyright (c) 2008--2014, Nico Schlömer <nico.schloemer@gmail.com>
% All rights reserved.
% Minimal pgfplots version: 1.3
%
% The latest updates can be retrieved from
%   http://www.mathworks.com/matlabcentral/fileexchange/22022-matlab2tikz
% where you can also make suggestions and rate matlab2tikz.
%
\begin{tikzpicture}

\begin{axis}[%
width=\figurewidth,
height=\figureheight,
scale only axis,
xmin=0,
xmax=10,
ymode=log,
ymin=1e-05,
ymax=10,
yminorticks=false
]
\addplot [color=red,solid,forget plot]
  table[row sep=crcr]{
0.100016	1.78553	\\
0.100732	1.76772	\\
0.101143	1.75763	\\
0.101868	1.74001	\\
0.102291	1.72985	\\
0.102765	1.71857	\\
0.103121	1.71017	\\
0.103483	1.70171	\\
0.103876	1.69256	\\
0.104476	1.67877	\\
0.104665	1.67446	\\
0.105362	1.65871	\\
0.105744	1.65016	\\
0.106062	1.64309	\\
0.106298	1.63788	\\
0.106648	1.63019	\\
0.10715	1.61927	\\
0.107607	1.60941	\\
0.108043	1.60009	\\
0.108317	1.59428	\\
0.108894	1.58213	\\
0.109224	1.57523	\\
0.109604	1.56736	\\
0.110172	1.5557	\\
0.110505	1.54893	\\
0.110927	1.5404	\\
0.111227	1.53439	\\
0.111839	1.5222	\\
0.112341	1.51232	\\
0.112781	1.50374	\\
0.113469	1.49046	\\
0.113904	1.48215	\\
0.114661	1.46786	\\
0.115014	1.46127	\\
0.115804	1.44665	\\
0.11616	1.44014	\\
0.11676	1.42926	\\
0.1172	1.42135	\\
0.117616	1.41394	\\
0.118117	1.40508	\\
0.118705	1.39479	\\
0.119312	1.38428	\\
0.119771	1.3764	\\
0.120376	1.36614	\\
0.121135	1.35339	\\
0.121529	1.34686	\\
0.122052	1.33823	\\
0.12263	1.3288	\\
0.123087	1.32142	\\
0.123626	1.31278	\\
0.124085	1.3055	\\
0.124676	1.29618	\\
0.125112	1.28938	\\
0.125592	1.28195	\\
0.126054	1.27486	\\
0.126291	1.27123	\\
0.126904	1.26194	\\
0.127492	1.25312	\\
0.127792	1.24866	\\
0.128534	1.23771	\\
0.129167	1.22848	\\
0.129774	1.21972	\\
0.130294	1.21229	\\
0.130818	1.20486	\\
0.131487	1.19547	\\
0.132056	1.18756	\\
0.132644	1.17947	\\
0.133001	1.1746	\\
0.133666	1.16559	\\
0.134361	1.15629	\\
0.13507	1.1469	\\
0.135588	1.1401	\\
0.135991	1.13486	\\
0.13657	1.12739	\\
0.13714	1.1201	\\
0.13769	1.11313	\\
0.138229	1.10635	\\
0.138825	1.09893	\\
0.139339	1.09257	\\
0.140146	1.08272	\\
0.140895	1.07367	\\
0.141373	1.06796	\\
0.141897	1.06174	\\
0.142535	1.05424	\\
0.143589	1.042	\\
0.144197	1.03504	\\
0.144743	1.02884	\\
0.145248	1.02315	\\
0.145703	1.01805	\\
0.146385	1.0105	\\
0.147035	1.00335	\\
0.147801	0.995037	\\
0.148257	0.990126	\\
0.149005	0.982151	\\
0.149751	0.974279	\\
0.150789	0.963468	\\
0.151369	0.957503	\\
0.15236	0.947425	\\
0.153206	0.938933	\\
0.153923	0.931819	\\
0.154616	0.925009	\\
0.155702	0.914478	\\
0.156276	0.908976	\\
0.157109	0.901075	\\
0.157913	0.893536	\\
0.158792	0.885393	\\
0.159332	0.880432	\\
0.160076	0.873675	\\
0.160613	0.868831	\\
0.161543	0.860541	\\
0.161952	0.856925	\\
0.162943	0.848249	\\
0.163795	0.840887	\\
0.164577	0.834211	\\
0.165563	0.825882	\\
0.166526	0.817859	\\
0.167275	0.811696	\\
0.167903	0.806574	\\
0.168356	0.8029	\\
0.169313	0.795223	\\
0.170372	0.786839	\\
0.171347	0.779227	\\
0.172157	0.772975	\\
0.172836	0.767789	\\
0.173425	0.763325	\\
0.173801	0.760496	\\
0.174706	0.753736	\\
0.175346	0.749005	\\
0.176423	0.741125	\\
0.177117	0.736112	\\
0.178062	0.729352	\\
0.178898	0.72344	\\
0.180102	0.715037	\\
0.180862	0.709797	\\
0.182027	0.701865	\\
0.184089	0.688107	\\
0.18487	0.682988	\\
0.185944	0.676026	\\
0.18661	0.671759	\\
0.187504	0.666083	\\
0.188082	0.66245	\\
0.188911	0.657279	\\
0.189884	0.651272	\\
0.190693	0.646338	\\
0.191648	0.640575	\\
0.19297	0.632701	\\
0.194005	0.626625	\\
0.195319	0.619018	\\
0.19641	0.612791	\\
0.197237	0.608125	\\
0.198188	0.602816	\\
0.198714	0.599907	\\
0.199929	0.593247	\\
0.200903	0.587981	\\
0.201795	0.583211	\\
0.202715	0.57834	\\
0.203536	0.574035	\\
0.204572	0.568667	\\
0.205426	0.564288	\\
0.207216	0.555246	\\
0.207983	0.551425	\\
0.208602	0.548366	\\
0.209736	0.542821	\\
0.210653	0.538388	\\
0.211348	0.535059	\\
0.212512	0.529543	\\
0.213482	0.524996	\\
0.214593	0.519852	\\
0.215334	0.516455	\\
0.216055	0.513179	\\
0.217124	0.508364	\\
0.21783	0.505218	\\
0.219062	0.499785	\\
0.219841	0.496385	\\
0.220808	0.492208	\\
0.222028	0.486997	\\
0.222801	0.483729	\\
0.223564	0.480532	\\
0.224473	0.476757	\\
0.225435	0.472804	\\
0.226929	0.466743	\\
0.228105	0.462039	\\
0.229383	0.456992	\\
0.230274	0.453513	\\
0.231502	0.448774	\\
0.23291	0.443413	\\
0.233592	0.440845	\\
0.234564	0.437215	\\
0.235642	0.433236	\\
0.237129	0.427814	\\
0.238486	0.422939	\\
0.239473	0.419437	\\
0.240467	0.415944	\\
0.241136	0.413614	\\
0.242537	0.408785	\\
0.244035	0.403696	\\
0.245076	0.400204	\\
0.246009	0.397107	\\
0.247	0.393847	\\
0.247885	0.390963	\\
0.249225	0.386647	\\
0.250428	0.382819	\\
0.251459	0.379575	\\
0.252329	0.376865	\\
0.253517	0.373198	\\
0.254847	0.369145	\\
0.255904	0.365962	\\
0.257098	0.362404	\\
0.258282	0.358917	\\
0.259441	0.355543	\\
0.260833	0.35154	\\
0.262131	0.347855	\\
0.262987	0.345451	\\
0.263852	0.34304	\\
0.264951	0.340005	\\
0.266916	0.334661	\\
0.268033	0.331666	\\
0.269649	0.327388	\\
0.271025	0.323797	\\
0.272281	0.320561	\\
0.274193	0.315706	\\
0.275028	0.313613	\\
0.276161	0.310797	\\
0.27739	0.307778	\\
0.278326	0.305501	\\
0.279306	0.303138	\\
0.280599	0.300052	\\
0.281796	0.297227	\\
0.282404	0.295807	\\
0.282916	0.294613	\\
0.284345	0.291317	\\
0.286539	0.286338	\\
0.288465	0.282046	\\
0.289577	0.279604	\\
0.290789	0.276969	\\
0.291891	0.274596	\\
0.292729	0.272809	\\
0.293549	0.271072	\\
0.294692	0.268673	\\
0.29527	0.267469	\\
0.296798	0.264316	\\
0.298342	0.261171	\\
0.299756	0.258329	\\
0.301579	0.254719	\\
0.302548	0.252822	\\
0.304394	0.249253	\\
0.306557	0.245144	\\
0.308453	0.241605	\\
0.310287	0.238236	\\
0.311845	0.235416	\\
0.313014	0.233325	\\
0.314818	0.230139	\\
0.316266	0.227616	\\
0.317626	0.225274	\\
0.319309	0.222415	\\
0.320928	0.219701	\\
0.322521	0.217068	\\
0.323438	0.215567	\\
0.324975	0.213079	\\
0.32696	0.209912	\\
0.328393	0.207659	\\
0.329228	0.206358	\\
0.330537	0.204336	\\
0.331616	0.202686	\\
0.332773	0.200934	\\
0.334735	0.198	\\
0.335838	0.196371	\\
0.336532	0.195353	\\
0.338708	0.192201	\\
0.339764	0.190692	\\
0.340906	0.189073	\\
0.342263	0.18717	\\
0.344496	0.184084	\\
0.346563	0.181275	\\
0.348333	0.178908	\\
0.349845	0.176913	\\
0.352231	0.173811	\\
0.353715	0.171912	\\
0.355412	0.169768	\\
0.35655	0.168346	\\
0.358077	0.166459	\\
0.360099	0.163995	\\
0.361818	0.161931	\\
0.364139	0.159187	\\
0.36546	0.157647	\\
0.367669	0.15511	\\
0.369975	0.152506	\\
0.371557	0.150746	\\
0.372694	0.149495	\\
0.373774	0.148317	\\
0.375262	0.14671	\\
0.377306	0.144531	\\
0.380459	0.141238	\\
0.382933	0.13871	\\
0.384678	0.136955	\\
0.385902	0.135738	\\
0.388727	0.132972	\\
0.391005	0.130784	\\
0.393414	0.128511	\\
0.395144	0.126904	\\
0.396611	0.125558	\\
0.398773	0.123602	\\
0.400663	0.121917	\\
0.403071	0.119804	\\
0.404874	0.118247	\\
0.406304	0.117027	\\
0.407502	0.116015	\\
0.408969	0.114787	\\
0.410986	0.113122	\\
0.413545	0.111044	\\
0.415279	0.109658	\\
0.418128	0.10742	\\
0.420064	0.105925	\\
0.421823	0.104584	\\
0.424593	0.102509	\\
0.426398	0.10118	\\
0.428145	0.0999086	\\
0.430117	0.0984941	\\
0.431719	0.0973592	\\
0.434004	0.0957634	\\
0.435639	0.0946373	\\
0.438002	0.0930339	\\
0.44003	0.0916795	\\
0.441192	0.0909121	\\
0.444417	0.0888159	\\
0.446955	0.0871996	\\
0.44909	0.0858625	\\
0.451835	0.0841736	\\
0.453309	0.0832803	\\
0.454813	0.0823777	\\
0.457958	0.0805226	\\
0.459862	0.0794191	\\
0.462247	0.0780578	\\
0.464619	0.0767268	\\
0.465952	0.0759886	\\
0.467532	0.0751222	\\
0.468681	0.0744982	\\
0.471989	0.0727299	\\
0.47471	0.0713058	\\
0.476291	0.0704906	\\
0.477826	0.069708	\\
0.479499	0.0688642	\\
0.482866	0.067196	\\
0.485005	0.0661562	\\
0.486881	0.065257	\\
0.489435	0.0640519	\\
0.491926	0.0628967	\\
0.495021	0.0614886	\\
0.498525	0.0599309	\\
0.50063	0.059013	\\
0.504166	0.0575009	\\
0.505665	0.056871	\\
0.507841	0.0559679	\\
0.510741	0.0547852	\\
0.512662	0.0540148	\\
0.514519	0.0532796	\\
0.517737	0.0520274	\\
0.520304	0.0510482	\\
0.522653	0.0501673	\\
0.525222	0.0492196	\\
0.527856	0.0482654	\\
0.529053	0.0478374	\\
0.530345	0.0473793	\\
0.532671	0.0465646	\\
0.537014	0.0450777	\\
0.539905	0.044112	\\
0.542539	0.0432485	\\
0.545665	0.0422438	\\
0.548282	0.0414188	\\
0.55136	0.0404671	\\
0.553736	0.0397461	\\
0.557423	0.0386497	\\
0.559872	0.0379368	\\
0.563057	0.0370268	\\
0.566709	0.0360072	\\
0.568692	0.0354642	\\
0.570944	0.0348562	\\
0.572341	0.0344836	\\
0.576713	0.0333405	\\
0.578029	0.0330029	\\
0.58042	0.0323973	\\
0.583278	0.0316863	\\
0.58787	0.030572	\\
0.591899	0.0296227	\\
0.596812	0.0284994	\\
0.598832	0.0280485	\\
0.602077	0.0273368	\\
0.603788	0.0269678	\\
0.605351	0.0266343	\\
0.609567	0.0257525	\\
0.612397	0.0251747	\\
0.613894	0.0248735	\\
0.617595	0.0241418	\\
0.618871	0.0238938	\\
0.621349	0.0234183	\\
0.623221	0.0230645	\\
0.625753	0.0225929	\\
0.630458	0.021738	\\
0.631941	0.0214742	\\
0.636069	0.0207538	\\
0.638582	0.0203252	\\
0.642931	0.0196006	\\
0.646184	0.0190725	\\
0.649208	0.0185923	\\
0.651934	0.0181677	\\
0.655829	0.0175749	\\
0.659188	0.0170764	\\
0.661905	0.0166815	\\
0.664688	0.0162848	\\
0.668699	0.0157263	\\
0.673064	0.0151361	\\
0.675007	0.014879	\\
0.677923	0.0144998	\\
0.680482	0.0141734	\\
0.683106	0.0138447	\\
0.686506	0.0134278	\\
0.68944	0.013076	\\
0.695155	0.0124115	\\
0.699714	0.0119005	\\
0.701674	0.0116858	\\
0.705204	0.0113068	\\
0.708283	0.0109841	\\
0.710861	0.0107193	\\
0.713441	0.0104592	\\
0.716795	0.0101283	\\
0.718886	0.00992605	\\
0.722192	0.00961258	\\
0.725998	0.00926101	\\
0.730187	0.00888535	\\
0.73199	0.00872717	\\
0.736732	0.00832126	\\
0.741189	0.00795266	\\
0.743401	0.00777423	\\
0.74612	0.00755902	\\
0.750475	0.00722347	\\
0.753352	0.0070079	\\
0.755676	0.00683713	\\
0.757967	0.00667186	\\
0.762502	0.00635313	\\
0.766831	0.00605927	\\
0.771437	0.00575743	\\
0.77395	0.00559732	\\
0.776972	0.00540902	\\
0.781398	0.00514138	\\
0.784609	0.00495318	\\
0.787748	0.00477385	\\
0.791038	0.00459088	\\
0.794143	0.00442269	\\
0.79783	0.00422855	\\
0.802238	0.00400427	\\
0.806834	0.00377927	\\
0.809551	0.00365041	\\
0.812533	0.00351238	\\
0.81617	0.00334894	\\
0.820285	0.00317024	\\
0.824309	0.00300179	\\
0.826707	0.0029043	\\
0.833319	0.00264638	\\
0.835586	0.00256156	\\
0.838401	0.00245875	\\
0.842981	0.00229728	\\
0.847488	0.00214526	\\
0.851709	0.00200891	\\
0.855601	0.00188822	\\
0.860051	0.00175605	\\
0.864038	0.00164271	\\
0.867779	0.00154065	\\
0.870037	0.00148104	\\
0.87366	0.00138843	\\
0.880088	0.00123312	\\
0.884857	0.00112515	\\
0.889037	0.00103544	\\
0.893547	0.000943658	\\
0.896011	0.000895672	\\
0.900526	0.000811603	\\
0.903659	0.000756148	\\
0.907433	0.000692409	\\
0.912219	0.000616308	\\
0.913714	0.000593579	\\
0.915829	0.000562299	\\
0.918211	0.000528225	\\
0.922562	0.000469152	\\
0.926574	0.000418226	\\
0.93001	0.000377252	\\
0.939074	0.000280472	\\
0.94474	0.000228013	\\
0.947393	0.000205506	\\
0.951158	0.000175757	\\
0.957049	0.000134272	\\
0.963964	9.31804e-05	\\
0.967383	7.58013e-05	\\
0.969213	6.72817e-05	\\
0.974442	4.58767e-05	\\
0.979112	3.0353e-05	\\
0.985362	1.47187e-05	\\
0.987692	1.0356e-05	\\
0.991722	4.64753e-06	\\
0.996869	6.5798e-07	\\
1.00069	3.21558e-08	\\
1.00686	3.11871e-06	\\
1.0114	8.56842e-06	\\
1.01564	1.60721e-05	\\
1.01883	2.32069e-05	\\
1.02234	3.25485e-05	\\
1.02492	4.03981e-05	\\
1.02894	5.42797e-05	\\
1.03158	6.44661e-05	\\
1.0349	7.84759e-05	\\
1.04362	0.000121524	\\
1.04669	0.00013883	\\
1.05053	0.000162015	\\
1.05595	0.000197575	\\
1.06253	0.000245234	\\
1.06742	0.00028374	\\
1.07338	0.000334198	\\
1.07781	0.000374101	\\
1.08253	0.000418999	\\
1.08601	0.000453567	\\
1.09	0.000494772	\\
1.09294	0.000526044	\\
1.09738	0.00057507	\\
1.102	0.000628214	\\
1.10895	0.00071199	\\
1.11633	0.000806112	\\
1.12186	0.000879896	\\
1.12619	0.00093978	\\
1.1329	0.00103573	\\
1.13615	0.00108373	\\
1.14014	0.00114396	\\
1.14299	0.00118776	\\
1.14711	0.00125234	\\
1.15176	0.00132714	\\
1.15845	0.00143763	\\
1.16166	0.00149207	\\
1.16882	0.00161639	\\
1.1752	0.00173082	\\
1.17957	0.00181084	\\
1.18216	0.00185913	\\
1.19114	0.00203026	\\
1.19497	0.00210506	\\
1.20001	0.00220507	\\
1.20526	0.00231133	\\
1.21056	0.00242058	\\
1.21672	0.00255009	\\
1.22073	0.00263567	\\
1.23028	0.00284424	\\
1.23784	0.00301352	\\
1.24459	0.00316794	\\
1.24788	0.00324412	\\
1.25128	0.00332362	\\
1.25545	0.00342233	\\
1.25793	0.00348141	\\
1.26242	0.00358925	\\
1.26925	0.00375578	\\
1.27407	0.00387512	\\
1.27764	0.00396428	\\
1.28445	0.00413627	\\
1.28897	0.00425202	\\
1.29525	0.00441454	\\
1.3049	0.00466852	\\
1.30871	0.00477022	\\
1.31864	0.00503846	\\
1.32815	0.0053003	\\
1.34101	0.005661	\\
1.3439	0.00574316	\\
1.34954	0.00590473	\\
1.35348	0.00601823	\\
1.35862	0.00616767	\\
1.36647	0.00639807	\\
1.37219	0.00656759	\\
1.3777	0.00673217	\\
1.38335	0.00690208	\\
1.38883	0.00706831	\\
1.39808	0.00735153	\\
1.40462	0.00755391	\\
1.40889	0.00768692	\\
1.41678	0.00793424	\\
1.42489	0.00819104	\\
1.4286	0.00830931	\\
1.43541	0.00852745	\\
1.44356	0.00879073	\\
1.447	0.00890263	\\
1.45273	0.00908968	\\
1.45783	0.00925701	\\
1.46261	0.00941469	\\
1.46644	0.00954141	\\
1.47306	0.00976172	\\
1.48112	0.0100318	\\
1.48631	0.0102066	\\
1.49152	0.0103831	\\
1.498	0.0106033	\\
1.50889	0.0109761	\\
1.51635	0.0112336	\\
1.52411	0.0115027	\\
1.53298	0.0118122	\\
1.53867	0.0120115	\\
1.54339	0.0121776	\\
1.55362	0.0125396	\\
1.56628	0.0129903	\\
1.57436	0.01328	\\
1.58248	0.0135724	\\
1.58694	0.0137337	\\
1.59426	0.0139989	\\
1.6016	0.0142659	\\
1.60963	0.0145594	\\
1.61845	0.014883	\\
1.62336	0.0150636	\\
1.63086	0.0153405	\\
1.63692	0.0155648	\\
1.64619	0.0159091	\\
1.65239	0.01614	\\
1.65937	0.0164008	\\
1.665	0.0166117	\\
1.67189	0.0168703	\\
1.67556	0.0170083	\\
1.68271	0.017278	\\
1.68987	0.0175484	\\
1.69877	0.0178856	\\
1.70866	0.0182614	\\
1.71442	0.0184807	\\
1.72011	0.018698	\\
1.72704	0.0189627	\\
1.73769	0.0193706	\\
1.74769	0.019755	\\
1.75454	0.0200186	\\
1.76234	0.0203194	\\
1.77118	0.0206607	\\
1.78279	0.0211106	\\
1.79008	0.0213933	\\
1.80116	0.0218236	\\
1.80927	0.0221396	\\
1.81677	0.022432	\\
1.8269	0.0228273	\\
1.83609	0.0231863	\\
1.84246	0.0234357	\\
1.84909	0.0236952	\\
1.86067	0.0241492	\\
1.86797	0.024436	\\
1.87509	0.0247157	\\
1.88462	0.0250901	\\
1.89612	0.0255428	\\
1.90382	0.0258457	\\
1.91835	0.0264184	\\
1.92937	0.0268531	\\
1.94517	0.0274767	\\
1.95358	0.027809	\\
1.96322	0.0281898	\\
1.97067	0.0284841	\\
1.97847	0.0287922	\\
1.99118	0.0292945	\\
1.99933	0.0296168	\\
2.00577	0.0298712	\\
2.01732	0.0303276	\\
2.02421	0.0305999	\\
2.03162	0.0308927	\\
2.0416	0.0312872	\\
2.05159	0.0316815	\\
2.06066	0.0320397	\\
2.07203	0.0324885	\\
2.07753	0.0327055	\\
2.08707	0.0330817	\\
2.09452	0.0333756	\\
2.10783	0.0338998	\\
2.1198	0.0343712	\\
2.12924	0.0347425	\\
2.14501	0.0353624	\\
2.15928	0.0359224	\\
2.17265	0.0364467	\\
2.19081	0.0371578	\\
2.2014	0.0375719	\\
2.21006	0.0379101	\\
2.21576	0.0381326	\\
2.22677	0.0385619	\\
2.23651	0.0389413	\\
2.25202	0.0395442	\\
2.26277	0.0399615	\\
2.27758	0.0405356	\\
2.2885	0.0409582	\\
2.2988	0.0413563	\\
2.30861	0.0417345	\\
2.31522	0.0419896	\\
2.32688	0.0424381	\\
2.34351	0.0430768	\\
2.35324	0.0434496	\\
2.36576	0.0439289	\\
2.36978	0.0440825	\\
2.3798	0.0444651	\\
2.39165	0.0449166	\\
2.40418	0.0453929	\\
2.41219	0.0456969	\\
2.42786	0.0462906	\\
2.43743	0.0466521	\\
2.44472	0.0469273	\\
2.45551	0.0473338	\\
2.46508	0.0476938	\\
2.47591	0.0481005	\\
2.49047	0.0486457	\\
2.50494	0.0491862	\\
2.51629	0.0496094	\\
2.53182	0.0501862	\\
2.54063	0.050513	\\
2.55441	0.0510227	\\
2.56632	0.051462	\\
2.57583	0.0518123	\\
2.59765	0.0526128	\\
2.61636	0.0532968	\\
2.62939	0.0537711	\\
2.64229	0.0542397	\\
2.65063	0.0545421	\\
2.66488	0.055057	\\
2.67204	0.0553153	\\
2.68853	0.0559084	\\
2.70159	0.056377	\\
2.71485	0.0568507	\\
2.72522	0.0572208	\\
2.73163	0.0574487	\\
2.74114	0.0577865	\\
2.75287	0.0582023	\\
2.75762	0.0583701	\\
2.76946	0.0587882	\\
2.79324	0.0596243	\\
2.80245	0.0599469	\\
2.82032	0.0605707	\\
2.83198	0.0609766	\\
2.84039	0.0612685	\\
2.85568	0.0617976	\\
2.87357	0.0624143	\\
2.88426	0.0627815	\\
2.90311	0.0634268	\\
2.91433	0.0638098	\\
2.91981	0.063996	\\
2.93285	0.0644389	\\
2.94845	0.0649669	\\
2.95951	0.0653399	\\
2.97174	0.0657513	\\
2.98385	0.0661576	\\
3.00019	0.0667035	\\
3.00996	0.0670288	\\
3.02438	0.0675077	\\
3.03676	0.0679175	\\
3.05065	0.0683756	\\
3.06015	0.0686881	\\
3.07312	0.0691134	\\
3.09063	0.0696854	\\
3.10426	0.0701289	\\
3.11856	0.0705924	\\
3.13097	0.0709932	\\
3.14182	0.071343	\\
3.15581	0.071792	\\
3.17855	0.0725187	\\
3.1944	0.0730228	\\
3.20509	0.0733618	\\
3.21843	0.073783	\\
3.23677	0.07436	\\
3.25536	0.0749421	\\
3.27116	0.0754347	\\
3.28145	0.0757544	\\
3.28776	0.07595	\\
3.29755	0.0762527	\\
3.31668	0.0768421	\\
3.33641	0.0774471	\\
3.35121	0.0778987	\\
3.36002	0.0781666	\\
3.37035	0.0784802	\\
3.38202	0.0788332	\\
3.40333	0.0794751	\\
3.41024	0.0796827	\\
3.42871	0.0802349	\\
3.4392	0.0805477	\\
3.45	0.0808686	\\
3.47698	0.0816661	\\
3.48591	0.0819287	\\
3.49916	0.0823174	\\
3.51073	0.0826559	\\
3.53144	0.0832589	\\
3.54278	0.0835874	\\
3.55771	0.0840187	\\
3.56828	0.0843231	\\
3.59019	0.0849512	\\
3.61177	0.0855661	\\
3.62231	0.0858653	\\
3.63224	0.0861462	\\
3.64821	0.0865965	\\
3.67865	0.0874495	\\
3.6901	0.0877687	\\
3.70816	0.0882702	\\
3.72825	0.0888248	\\
3.74526	0.0892925	\\
3.75968	0.0896868	\\
3.784	0.0903489	\\
3.79747	0.0907137	\\
3.81243	0.0911174	\\
3.82783	0.0915313	\\
3.84763	0.0920611	\\
3.85952	0.0923777	\\
3.87565	0.0928057	\\
3.90046	0.0934604	\\
3.91193	0.0937616	\\
3.93007	0.0942361	\\
3.9543	0.0948663	\\
3.97897	0.0955037	\\
3.99471	0.0959083	\\
4.01354	0.0963899	\\
4.02971	0.0968019	\\
4.0409	0.0970857	\\
4.05891	0.097541	\\
4.07359	0.0979102	\\
4.08099	0.0980959	\\
4.10528	0.0987029	\\
4.1273	0.0992497	\\
4.14039	0.0995732	\\
4.1655	0.100191	\\
4.18578	0.100686	\\
4.21003	0.101276	\\
4.23799	0.10195	\\
4.25882	0.10245	\\
4.27918	0.102935	\\
4.2957	0.103327	\\
4.31762	0.103845	\\
4.33062	0.104151	\\
4.35793	0.104789	\\
4.37829	0.105262	\\
4.4087	0.105964	\\
4.42985	0.106449	\\
4.44854	0.106875	\\
4.46135	0.107166	\\
4.48603	0.107724	\\
4.51	0.108262	\\
4.53141	0.108741	\\
4.55237	0.109206	\\
4.5749	0.109704	\\
4.59724	0.110194	\\
4.60872	0.110445	\\
4.62886	0.110883	\\
4.64609	0.111257	\\
4.70087	0.112433	\\
4.72547	0.112955	\\
4.74041	0.113271	\\
4.75	0.113474	\\
4.7787	0.114075	\\
4.78755	0.11426	\\
4.80747	0.114674	\\
4.83792	0.115304	\\
4.85137	0.11558	\\
4.87628	0.116089	\\
4.89058	0.11638	\\
4.901	0.116592	\\
4.91963	0.116968	\\
4.94708	0.11752	\\
4.96503	0.117879	\\
4.98023	0.118181	\\
5.00062	0.118585	\\
5.03209	0.119204	\\
5.04733	0.119503	\\
5.05957	0.119741	\\
5.089	0.120312	\\
5.11273	0.12077	\\
5.139	0.121273	\\
5.17036	0.121869	\\
5.20238	0.122473	\\
5.22068	0.122816	\\
5.24171	0.123209	\\
5.27082	0.123749	\\
5.29698	0.124231	\\
5.3122	0.12451	\\
5.33363	0.124901	\\
5.35787	0.125341	\\
5.38709	0.125868	\\
5.41346	0.12634	\\
5.4475	0.126945	\\
5.48522	0.12761	\\
5.50637	0.12798	\\
5.5248	0.128302	\\
5.55313	0.128792	\\
5.57634	0.129192	\\
5.60366	0.129659	\\
5.62191	0.12997	\\
5.65958	0.130607	\\
5.69471	0.131196	\\
5.71707	0.131569	\\
5.73437	0.131855	\\
5.75999	0.132278	\\
5.78241	0.132646	\\
5.80886	0.133077	\\
5.82328	0.133311	\\
5.85852	0.13388	\\
5.88786	0.13435	\\
5.917	0.134813	\\
5.93246	0.135058	\\
5.96588	0.135584	\\
5.99341	0.136014	\\
6.011	0.136288	\\
6.03999	0.136736	\\
6.06491	0.137119	\\
6.09937	0.137645	\\
6.12948	0.138102	\\
6.16636	0.138656	\\
6.19125	0.139028	\\
6.21746	0.139417	\\
6.24477	0.13982	\\
6.26611	0.140134	\\
6.28719	0.140442	\\
6.3129	0.140816	\\
6.34627	0.141298	\\
6.37139	0.141658	\\
6.39363	0.141976	\\
6.41025	0.142212	\\
6.43865	0.142614	\\
6.47916	0.143182	\\
6.5102	0.143615	\\
6.52663	0.143842	\\
6.57432	0.144499	\\
6.61072	0.144995	\\
6.63516	0.145326	\\
6.68209	0.145957	\\
6.70497	0.146262	\\
6.7332	0.146637	\\
6.76038	0.146995	\\
6.79327	0.147426	\\
6.82495	0.147839	\\
6.86837	0.1484	\\
6.88989	0.148676	\\
6.91276	0.148968	\\
6.93727	0.149279	\\
6.99222	0.149971	\\
7.0143	0.150247	\\
7.03642	0.150522	\\
7.07477	0.150996	\\
7.09945	0.151299	\\
7.12214	0.151577	\\
7.15202	0.15194	\\
7.18136	0.152294	\\
7.22839	0.152858	\\
7.29291	0.153623	\\
7.32937	0.154051	\\
7.34602	0.154245	\\
7.39887	0.154857	\\
7.42534	0.155162	\\
7.45516	0.155503	\\
7.48223	0.15581	\\
7.51391	0.156168	\\
7.56059	0.156692	\\
7.6048	0.157183	\\
7.62181	0.157371	\\
7.64773	0.157656	\\
7.69234	0.158143	\\
7.73359	0.15859	\\
7.77596	0.159045	\\
7.82277	0.159543	\\
7.86155	0.159953	\\
7.89303	0.160283	\\
7.93264	0.160696	\\
7.97912	0.161176	\\
8.00993	0.161492	\\
8.05018	0.161902	\\
8.09962	0.162402	\\
8.13586	0.162765	\\
8.17892	0.163193	\\
8.2152	0.163551	\\
8.24147	0.163809	\\
8.27699	0.164156	\\
8.31588	0.164533	\\
8.36978	0.165051	\\
8.39427	0.165284	\\
8.43448	0.165666	\\
8.47906	0.166085	\\
8.50451	0.166323	\\
8.52412	0.166506	\\
8.57725	0.166997	\\
8.60789	0.167278	\\
8.6428	0.167597	\\
8.70304	0.168142	\\
8.76292	0.168678	\\
8.79102	0.168927	\\
8.8193	0.169177	\\
8.86518	0.16958	\\
8.91599	0.170022	\\
8.96533	0.170448	\\
8.99896	0.170736	\\
9.03091	0.171008	\\
9.06498	0.171296	\\
9.11212	0.171693	\\
9.15748	0.172071	\\
9.20151	0.172435	\\
9.2463	0.172803	\\
9.30386	0.173272	\\
9.35036	0.173647	\\
9.41429	0.174158	\\
9.45702	0.174496	\\
9.49771	0.174816	\\
9.55248	0.175244	\\
9.61597	0.175734	\\
9.68052	0.176228	\\
9.70768	0.176434	\\
9.7569	0.176805	\\
9.79248	0.177071	\\
9.83031	0.177353	\\
9.87647	0.177694	\\
9.90852	0.177929	\\
9.95018	0.178233	\\
};
\addplot [color=blue,dashed,forget plot]
  table[row sep=crcr]{
0.100016	0.981797	\\
0.100732	0.972638	\\
0.101143	0.967446	\\
0.101868	0.958381	\\
0.102291	0.953152	\\
0.102765	0.947343	\\
0.103121	0.943017	\\
0.103483	0.938659	\\
0.103876	0.933948	\\
0.104476	0.926841	\\
0.104665	0.924621	\\
0.105362	0.9165	\\
0.105744	0.912088	\\
0.106062	0.908446	\\
0.106298	0.905755	\\
0.106648	0.901789	\\
0.10715	0.896156	\\
0.107607	0.891065	\\
0.108043	0.886254	\\
0.108317	0.883254	\\
0.108894	0.876982	\\
0.109224	0.873419	\\
0.109604	0.86935	\\
0.110172	0.863324	\\
0.110505	0.859828	\\
0.110927	0.855416	\\
0.111227	0.852308	\\
0.111839	0.846006	\\
0.112341	0.840894	\\
0.112781	0.836452	\\
0.113469	0.829579	\\
0.113904	0.825278	\\
0.114661	0.817873	\\
0.115014	0.81446	\\
0.115804	0.806885	\\
0.11616	0.80351	\\
0.11676	0.797867	\\
0.1172	0.793765	\\
0.117616	0.78992	\\
0.118117	0.785324	\\
0.118705	0.779983	\\
0.119312	0.774524	\\
0.119771	0.770435	\\
0.120376	0.765102	\\
0.121135	0.758479	\\
0.121529	0.755079	\\
0.122052	0.750595	\\
0.12263	0.74569	\\
0.123087	0.741852	\\
0.123626	0.737351	\\
0.124085	0.733562	\\
0.124676	0.728711	\\
0.125112	0.725169	\\
0.125592	0.721296	\\
0.126054	0.717602	\\
0.126291	0.715711	\\
0.126904	0.710868	\\
0.127492	0.706269	\\
0.127792	0.703942	\\
0.128534	0.698228	\\
0.129167	0.693407	\\
0.129774	0.688831	\\
0.130294	0.68495	\\
0.130818	0.68107	\\
0.131487	0.67616	\\
0.132056	0.672029	\\
0.132644	0.667796	\\
0.133001	0.665248	\\
0.133666	0.660531	\\
0.134361	0.655663	\\
0.13507	0.650748	\\
0.135588	0.647185	\\
0.135991	0.644439	\\
0.13657	0.640521	\\
0.13714	0.636701	\\
0.13769	0.633044	\\
0.138229	0.629488	\\
0.138825	0.625593	\\
0.139339	0.622256	\\
0.140146	0.617082	\\
0.140895	0.612326	\\
0.141373	0.609324	\\
0.141897	0.606052	\\
0.142535	0.60211	\\
0.143589	0.59567	\\
0.144197	0.592005	\\
0.144743	0.588737	\\
0.145248	0.585741	\\
0.145703	0.583054	\\
0.146385	0.579072	\\
0.147035	0.575303	\\
0.147801	0.570916	\\
0.148257	0.568324	\\
0.149005	0.564114	\\
0.149751	0.559956	\\
0.150789	0.554243	\\
0.151369	0.551089	\\
0.15236	0.545758	\\
0.153206	0.541264	\\
0.153923	0.537497	\\
0.154616	0.53389	\\
0.155702	0.528308	\\
0.156276	0.525391	\\
0.157109	0.521199	\\
0.157913	0.517198	\\
0.158792	0.512873	\\
0.159332	0.510238	\\
0.160076	0.506647	\\
0.160613	0.504072	\\
0.161543	0.499662	\\
0.161952	0.497738	\\
0.162943	0.493119	\\
0.163795	0.489198	\\
0.164577	0.485639	\\
0.165563	0.481198	\\
0.166526	0.476918	\\
0.167275	0.473628	\\
0.167903	0.470892	\\
0.168356	0.46893	\\
0.169313	0.464826	\\
0.170372	0.460343	\\
0.171347	0.456269	\\
0.172157	0.452922	\\
0.172836	0.450143	\\
0.173425	0.447751	\\
0.173801	0.446235	\\
0.174706	0.442609	\\
0.175346	0.440071	\\
0.176423	0.435841	\\
0.177117	0.433148	\\
0.178062	0.429515	\\
0.178898	0.426336	\\
0.180102	0.421815	\\
0.180862	0.418993	\\
0.182027	0.41472	\\
0.184089	0.4073	\\
0.18487	0.404537	\\
0.185944	0.400777	\\
0.18661	0.39847	\\
0.187504	0.395402	\\
0.188082	0.393436	\\
0.188911	0.390637	\\
0.189884	0.387385	\\
0.190693	0.384711	\\
0.191648	0.381587	\\
0.19297	0.377315	\\
0.194005	0.374016	\\
0.195319	0.369882	\\
0.19641	0.366496	\\
0.197237	0.363957	\\
0.198188	0.361066	\\
0.198714	0.359481	\\
0.199929	0.355851	\\
0.200903	0.352979	\\
0.201795	0.350376	\\
0.202715	0.347715	\\
0.203536	0.345363	\\
0.204572	0.342428	\\
0.205426	0.340032	\\
0.207216	0.33508	\\
0.207983	0.332986	\\
0.208602	0.331308	\\
0.209736	0.328266	\\
0.210653	0.325832	\\
0.211348	0.324003	\\
0.212512	0.320972	\\
0.213482	0.318471	\\
0.214593	0.315639	\\
0.215334	0.313768	\\
0.216055	0.311963	\\
0.217124	0.309307	\\
0.21783	0.307571	\\
0.219062	0.304572	\\
0.219841	0.302693	\\
0.220808	0.300384	\\
0.222028	0.297501	\\
0.222801	0.295692	\\
0.223564	0.293921	\\
0.224473	0.291829	\\
0.225435	0.289636	\\
0.226929	0.286272	\\
0.228105	0.283659	\\
0.229383	0.280853	\\
0.230274	0.278917	\\
0.231502	0.276278	\\
0.23291	0.27329	\\
0.233592	0.271857	\\
0.234564	0.269831	\\
0.235642	0.267609	\\
0.237129	0.264579	\\
0.238486	0.261851	\\
0.239473	0.259889	\\
0.240467	0.257932	\\
0.241136	0.256625	\\
0.242537	0.253916	\\
0.244035	0.251057	\\
0.245076	0.249094	\\
0.246009	0.247352	\\
0.247	0.245516	\\
0.247885	0.243891	\\
0.249225	0.241458	\\
0.250428	0.239298	\\
0.251459	0.237466	\\
0.252329	0.235934	\\
0.253517	0.23386	\\
0.254847	0.231566	\\
0.255904	0.229763	\\
0.257098	0.227745	\\
0.258282	0.225767	\\
0.259441	0.22385	\\
0.260833	0.221575	\\
0.262131	0.219478	\\
0.262987	0.218109	\\
0.263852	0.216736	\\
0.264951	0.215005	\\
0.266916	0.211955	\\
0.268033	0.210243	\\
0.269649	0.207796	\\
0.271025	0.20574	\\
0.272281	0.203884	\\
0.274193	0.201098	\\
0.275028	0.199896	\\
0.276161	0.198278	\\
0.27739	0.196541	\\
0.278326	0.19523	\\
0.279306	0.193868	\\
0.280599	0.192088	\\
0.281796	0.190458	\\
0.282404	0.189638	\\
0.282916	0.188948	\\
0.284345	0.187042	\\
0.286539	0.18416	\\
0.288465	0.181671	\\
0.289577	0.180254	\\
0.290789	0.178723	\\
0.291891	0.177343	\\
0.292729	0.176303	\\
0.293549	0.175292	\\
0.294692	0.173894	\\
0.29527	0.173192	\\
0.296798	0.171352	\\
0.298342	0.169515	\\
0.299756	0.167854	\\
0.301579	0.16574	\\
0.302548	0.164628	\\
0.304394	0.162534	\\
0.306557	0.16012	\\
0.308453	0.158038	\\
0.310287	0.156052	\\
0.311845	0.154389	\\
0.313014	0.153154	\\
0.314818	0.15127	\\
0.316266	0.149777	\\
0.317626	0.148389	\\
0.319309	0.146693	\\
0.320928	0.145081	\\
0.322521	0.143515	\\
0.323438	0.142622	\\
0.324975	0.141139	\\
0.32696	0.13925	\\
0.328393	0.137904	\\
0.329228	0.137126	\\
0.330537	0.135917	\\
0.331616	0.134929	\\
0.332773	0.133879	\\
0.334735	0.132118	\\
0.335838	0.131139	\\
0.336532	0.130527	\\
0.338708	0.12863	\\
0.339764	0.127721	\\
0.340906	0.126745	\\
0.342263	0.125596	\\
0.344496	0.123731	\\
0.346563	0.12203	\\
0.348333	0.120595	\\
0.349845	0.119384	\\
0.352231	0.117498	\\
0.353715	0.116342	\\
0.355412	0.115035	\\
0.35655	0.114168	\\
0.358077	0.113015	\\
0.360099	0.111508	\\
0.361818	0.110244	\\
0.364139	0.10856	\\
0.36546	0.107615	\\
0.367669	0.106054	\\
0.369975	0.104449	\\
0.371557	0.103363	\\
0.372694	0.10259	\\
0.373774	0.101862	\\
0.375262	0.100868	\\
0.377306	0.0995175	\\
0.380459	0.0974729	\\
0.382933	0.0958993	\\
0.384678	0.0948059	\\
0.385902	0.0940465	\\
0.388727	0.0923178	\\
0.391005	0.0909477	\\
0.393414	0.0895223	\\
0.395144	0.0885126	\\
0.396611	0.0876661	\\
0.398773	0.0864334	\\
0.400663	0.0853705	\\
0.403071	0.0840356	\\
0.404874	0.0830497	\\
0.406304	0.0822767	\\
0.407502	0.0816344	\\
0.408969	0.0808549	\\
0.410986	0.0797961	\\
0.413545	0.0784724	\\
0.415279	0.0775883	\\
0.418128	0.0761574	\\
0.420064	0.0752	\\
0.421823	0.0743405	\\
0.424593	0.0730077	\\
0.426398	0.0721519	\\
0.428145	0.0713329	\\
0.430117	0.07042	\\
0.431719	0.0696866	\\
0.434004	0.0686539	\\
0.435639	0.067924	\\
0.438002	0.0668831	\\
0.44003	0.0660024	\\
0.441192	0.0655028	\\
0.444417	0.0641357	\\
0.446955	0.0630792	\\
0.44909	0.0622037	\\
0.451835	0.0610957	\\
0.453309	0.0605087	\\
0.454813	0.0599149	\\
0.457958	0.0586923	\\
0.459862	0.0579636	\\
0.462247	0.0570632	\\
0.464619	0.0561812	\\
0.465952	0.0556913	\\
0.467532	0.0551157	\\
0.468681	0.0547007	\\
0.471989	0.0535227	\\
0.47471	0.0525717	\\
0.476291	0.0520264	\\
0.477826	0.0515023	\\
0.479499	0.0509365	\\
0.482866	0.0498157	\\
0.485005	0.0491157	\\
0.486881	0.0485094	\\
0.489435	0.0476953	\\
0.491926	0.0469135	\\
0.495021	0.0459584	\\
0.498525	0.0448992	\\
0.50063	0.0442736	\\
0.504166	0.0432409	\\
0.505665	0.0428098	\\
0.507841	0.0421909	\\
0.510741	0.0413789	\\
0.512662	0.0408489	\\
0.514519	0.0403424	\\
0.517737	0.039478	\\
0.520304	0.0388005	\\
0.522653	0.0381898	\\
0.525222	0.0375317	\\
0.527856	0.0368676	\\
0.529053	0.0365694	\\
0.530345	0.0362498	\\
0.532671	0.0356806	\\
0.537014	0.0346392	\\
0.539905	0.0339609	\\
0.542539	0.0333531	\\
0.545665	0.0326443	\\
0.548282	0.032061	\\
0.55136	0.0313866	\\
0.553736	0.0308746	\\
0.557423	0.0300942	\\
0.559872	0.0295856	\\
0.563057	0.0289349	\\
0.566709	0.0282039	\\
0.568692	0.0278137	\\
0.570944	0.0273762	\\
0.572341	0.0271076	\\
0.576713	0.026282	\\
0.578029	0.0260375	\\
0.58042	0.0255985	\\
0.583278	0.025082	\\
0.58787	0.0242702	\\
0.591899	0.0235763	\\
0.596812	0.0227523	\\
0.598832	0.0224206	\\
0.602077	0.0218961	\\
0.603788	0.0216236	\\
0.605351	0.021377	\\
0.609567	0.0207236	\\
0.612397	0.0202943	\\
0.613894	0.0200701	\\
0.617595	0.0195244	\\
0.618871	0.0193391	\\
0.621349	0.0189832	\\
0.623221	0.018718	\\
0.625753	0.0183639	\\
0.630458	0.0177202	\\
0.631941	0.017521	\\
0.636069	0.0169761	\\
0.638582	0.0166511	\\
0.642931	0.0161001	\\
0.646184	0.0156974	\\
0.649208	0.0153302	\\
0.651934	0.0150049	\\
0.655829	0.0145496	\\
0.659188	0.0141655	\\
0.661905	0.0138606	\\
0.664688	0.0135537	\\
0.668699	0.0131204	\\
0.673064	0.012661	\\
0.675007	0.0124605	\\
0.677923	0.0121641	\\
0.680482	0.0119084	\\
0.683106	0.0116504	\\
0.686506	0.0113224	\\
0.68944	0.0110449	\\
0.695155	0.0105191	\\
0.699714	0.0101132	\\
0.701674	0.0099422	\\
0.705204	0.00963971	\\
0.708283	0.00938146	\\
0.710861	0.00916915	\\
0.713441	0.00896015	\\
0.716795	0.00869363	\\
0.718886	0.00853043	\\
0.722192	0.00827693	\\
0.725998	0.00799184	\\
0.730187	0.00768627	\\
0.73199	0.00755731	\\
0.736732	0.00722555	\\
0.741189	0.00692321	\\
0.743401	0.00677648	\\
0.74612	0.00659917	\\
0.750475	0.00632196	\\
0.753352	0.00614338	\\
0.755676	0.00600163	\\
0.757967	0.00586419	\\
0.762502	0.00559846	\\
0.766831	0.00535263	\\
0.771437	0.00509925	\\
0.77395	0.00496448	\\
0.776972	0.00480564	\\
0.781398	0.00457924	\\
0.784609	0.00441958	\\
0.787748	0.00426706	\\
0.791038	0.00411106	\\
0.794143	0.00396732	\\
0.79783	0.00380098	\\
0.802238	0.0036082	\\
0.806834	0.00341414	\\
0.809551	0.00330269	\\
0.812533	0.00318305	\\
0.81617	0.00304103	\\
0.820285	0.00288528	\\
0.824309	0.00273801	\\
0.826707	0.00265258	\\
0.833319	0.00242576	\\
0.835586	0.00235091	\\
0.838401	0.00226002	\\
0.842981	0.00211687	\\
0.847488	0.00198162	\\
0.851709	0.00185991	\\
0.855601	0.00175185	\\
0.860051	0.00163313	\\
0.864038	0.001531	\\
0.867779	0.00143877	\\
0.870037	0.00138477	\\
0.87366	0.00130069	\\
0.880088	0.00115917	\\
0.884857	0.00106035	\\
0.889037	0.000977969	\\
0.893547	0.000893414	\\
0.896011	0.000849087	\\
0.900526	0.000771223	\\
0.903659	0.000719713	\\
0.907433	0.000660351	\\
0.912219	0.000589249	\\
0.913714	0.000567963	\\
0.915829	0.000538626	\\
0.918211	0.000506616	\\
0.922562	0.000450981	\\
0.926574	0.000402867	\\
0.93001	0.000364046	\\
0.939074	0.000271926	\\
0.94474	0.000221711	\\
0.947393	0.000200099	\\
0.951158	0.000171464	\\
0.957049	0.000131387	\\
0.963964	9.1501e-05	\\
0.967383	7.45647e-05	\\
0.969213	6.62456e-05	\\
0.974442	4.52902e-05	\\
0.979112	3.00359e-05	\\
0.985362	1.46109e-05	\\
0.987692	1.02923e-05	\\
0.991722	4.62829e-06	\\
0.996869	6.56949e-07	\\
1.00069	3.21446e-08	\\
1.00686	3.10809e-06	\\
1.0114	8.52015e-06	\\
1.01564	1.59484e-05	\\
1.01883	2.29925e-05	\\
1.02234	3.2193e-05	\\
1.02492	3.99072e-05	\\
1.02894	5.35166e-05	\\
1.03158	6.34797e-05	\\
1.0349	7.71531e-05	\\
1.04362	0.000118985	\\
1.04669	0.000135735	\\
1.05053	0.00015812	\\
1.05595	0.000192342	\\
1.06253	0.00023802	\\
1.06742	0.000274782	\\
1.07338	0.000322777	\\
1.07781	0.000360603	\\
1.08253	0.000403033	\\
1.08601	0.000435612	\\
1.09	0.000474351	\\
1.09294	0.000503685	\\
1.09738	0.000549563	\\
1.102	0.00059915	\\
1.10895	0.000677026	\\
1.11633	0.000764123	\\
1.12186	0.000832124	\\
1.12619	0.000887145	\\
1.1329	0.000975003	\\
1.13615	0.00101882	\\
1.14014	0.00107368	\\
1.14299	0.00111349	\\
1.14711	0.00117207	\\
1.15176	0.00123973	\\
1.15845	0.00133934	\\
1.16166	0.00138829	\\
1.16882	0.0014997	\\
1.1752	0.00160184	\\
1.17957	0.00167305	\\
1.18216	0.00171594	\\
1.19114	0.00186741	\\
1.19497	0.00193338	\\
1.20001	0.00202136	\\
1.20526	0.00211457	\\
1.21056	0.00221013	\\
1.21672	0.00232305	\\
1.22073	0.00239746	\\
1.23028	0.00257813	\\
1.23784	0.0027241	\\
1.24459	0.00285675	\\
1.24788	0.00292202	\\
1.25128	0.00299	\\
1.25545	0.00307426	\\
1.25793	0.00312461	\\
1.26242	0.00321633	\\
1.26925	0.00335755	\\
1.27407	0.00345845	\\
1.27764	0.00353368	\\
1.28445	0.00367842	\\
1.28897	0.00377555	\\
1.29525	0.00391155	\\
1.3049	0.00412328	\\
1.30871	0.00420778	\\
1.31864	0.0044299	\\
1.32815	0.00464572	\\
1.34101	0.00494145	\\
1.3439	0.00500857	\\
1.34954	0.00514029	\\
1.35348	0.00523261	\\
1.35862	0.00535392	\\
1.36647	0.00554039	\\
1.37219	0.00567718	\\
1.3777	0.00580963	\\
1.38335	0.00594604	\\
1.38883	0.00607917	\\
1.39808	0.00630525	\\
1.40462	0.00646625	\\
1.40889	0.00657182	\\
1.41678	0.00676759	\\
1.42489	0.00697017	\\
1.4286	0.00706325	\\
1.43541	0.00723452	\\
1.44356	0.0074406	\\
1.447	0.00752798	\\
1.45273	0.00767377	\\
1.45783	0.00780389	\\
1.46261	0.00792627	\\
1.46644	0.00802445	\\
1.47306	0.00819478	\\
1.48112	0.00840296	\\
1.48631	0.00853735	\\
1.49152	0.00867278	\\
1.498	0.00884134	\\
1.50889	0.00912579	\\
1.51635	0.00932158	\\
1.52411	0.00952556	\\
1.53298	0.0097594	\\
1.53867	0.00990965	\\
1.54339	0.0100346	\\
1.55362	0.0103061	\\
1.56628	0.0106427	\\
1.57436	0.0108583	\\
1.58248	0.0110753	\\
1.58694	0.0111947	\\
1.59426	0.0113907	\\
1.6016	0.0115874	\\
1.60963	0.0118032	\\
1.61845	0.0120403	\\
1.62336	0.0121724	\\
1.63086	0.0123744	\\
1.63692	0.0125377	\\
1.64619	0.0127876	\\
1.65239	0.0129549	\\
1.65937	0.0131433	\\
1.665	0.0132954	\\
1.67189	0.0134815	\\
1.67556	0.0135807	\\
1.68271	0.0137741	\\
1.68987	0.0139676	\\
1.69877	0.0142083	\\
1.70866	0.0144757	\\
1.71442	0.0146313	\\
1.72011	0.0147853	\\
1.72704	0.0149725	\\
1.73769	0.0152603	\\
1.74769	0.0155306	\\
1.75454	0.0157155	\\
1.76234	0.015926	\\
1.77118	0.0161643	\\
1.78279	0.0164774	\\
1.79008	0.0166737	\\
1.80116	0.0169716	\\
1.80927	0.0171897	\\
1.81677	0.0173912	\\
1.8269	0.0176628	\\
1.83609	0.0179089	\\
1.84246	0.0180795	\\
1.84909	0.0182566	\\
1.86067	0.0185658	\\
1.86797	0.0187606	\\
1.87509	0.0189502	\\
1.88462	0.0192035	\\
1.89612	0.0195089	\\
1.90382	0.0197127	\\
1.91835	0.020097	\\
1.92937	0.0203877	\\
1.94517	0.0208033	\\
1.95358	0.0210241	\\
1.96322	0.0212766	\\
1.97067	0.0214713	\\
1.97847	0.0216747	\\
1.99118	0.0220056	\\
1.99933	0.0222174	\\
2.00577	0.0223843	\\
2.01732	0.0226831	\\
2.02421	0.0228609	\\
2.03162	0.0230518	\\
2.0416	0.0233085	\\
2.05159	0.0235645	\\
2.06066	0.0237966	\\
2.07203	0.0240867	\\
2.07753	0.0242267	\\
2.08707	0.024469	\\
2.09452	0.0246579	\\
2.10783	0.0249941	\\
2.1198	0.0252957	\\
2.12924	0.0255326	\\
2.14501	0.0259271	\\
2.15928	0.0262824	\\
2.17265	0.0266141	\\
2.19081	0.0270625	\\
2.2014	0.0273229	\\
2.21006	0.027535	\\
2.21576	0.0276744	\\
2.22677	0.027943	\\
2.23651	0.0281799	\\
2.25202	0.0285553	\\
2.26277	0.0288145	\\
2.27758	0.0291703	\\
2.2885	0.0294315	\\
2.2988	0.029677	\\
2.30861	0.0299099	\\
2.31522	0.0300667	\\
2.32688	0.030342	\\
2.34351	0.030733	\\
2.35324	0.0309607	\\
2.36576	0.0312528	\\
2.36978	0.0313463	\\
2.3798	0.0315788	\\
2.39165	0.0318527	\\
2.40418	0.0321411	\\
2.41219	0.0323248	\\
2.42786	0.0326828	\\
2.43743	0.0329004	\\
2.44472	0.0330657	\\
2.45551	0.0333097	\\
2.46508	0.0335253	\\
2.47591	0.0337685	\\
2.49047	0.0340939	\\
2.50494	0.0344156	\\
2.51629	0.034667	\\
2.53182	0.035009	\\
2.54063	0.0352024	\\
2.55441	0.0355034	\\
2.56632	0.0357624	\\
2.57583	0.0359686	\\
2.59765	0.0364385	\\
2.61636	0.0368389	\\
2.62939	0.0371159	\\
2.64229	0.037389	\\
2.65063	0.0375649	\\
2.66488	0.0378641	\\
2.67204	0.0380139	\\
2.68853	0.0383574	\\
2.70159	0.0386281	\\
2.71485	0.0389014	\\
2.72522	0.0391145	\\
2.73163	0.0392456	\\
2.74114	0.0394397	\\
2.75287	0.0396782	\\
2.75762	0.0397744	\\
2.76946	0.0400137	\\
2.79324	0.0404912	\\
2.80245	0.040675	\\
2.82032	0.0410298	\\
2.83198	0.0412602	\\
2.84039	0.0414258	\\
2.85568	0.0417253	\\
2.87357	0.0420737	\\
2.88426	0.0422807	\\
2.90311	0.0426439	\\
2.91433	0.0428591	\\
2.91981	0.0429636	\\
2.93285	0.0432119	\\
2.94845	0.0435074	\\
2.95951	0.0437158	\\
2.97174	0.0439453	\\
2.98385	0.0441717	\\
3.00019	0.0444753	\\
3.00996	0.044656	\\
3.02438	0.0449216	\\
3.03676	0.0451485	\\
3.05065	0.0454018	\\
3.06015	0.0455744	\\
3.07312	0.0458089	\\
3.09063	0.0461238	\\
3.10426	0.0463676	\\
3.11856	0.0466219	\\
3.13097	0.0468415	\\
3.14182	0.0470329	\\
3.15581	0.0472784	\\
3.17855	0.0476748	\\
3.1944	0.0479492	\\
3.20509	0.0481334	\\
3.21843	0.0483622	\\
3.23677	0.0486749	\\
3.25536	0.0489899	\\
3.27116	0.0492559	\\
3.28145	0.0494284	\\
3.28776	0.0495338	\\
3.29755	0.0496968	\\
3.31668	0.0500138	\\
3.33641	0.0503385	\\
3.35121	0.0505805	\\
3.36002	0.0507239	\\
3.37035	0.0508916	\\
3.38202	0.0510802	\\
3.40333	0.0514225	\\
3.41024	0.0515331	\\
3.42871	0.0518269	\\
3.4392	0.0519931	\\
3.45	0.0521635	\\
3.47698	0.0525861	\\
3.48591	0.052725	\\
3.49916	0.0529305	\\
3.51073	0.0531092	\\
3.53144	0.0534271	\\
3.54278	0.0536001	\\
3.55771	0.0538269	\\
3.56828	0.0539868	\\
3.59019	0.0543163	\\
3.61177	0.0546383	\\
3.62231	0.0547948	\\
3.63224	0.0549415	\\
3.64821	0.0551766	\\
3.67865	0.055621	\\
3.6901	0.0557869	\\
3.70816	0.0560474	\\
3.72825	0.0563351	\\
3.74526	0.0565773	\\
3.75968	0.0567813	\\
3.784	0.0571233	\\
3.79747	0.0573114	\\
3.81243	0.0575194	\\
3.82783	0.0577324	\\
3.84763	0.0580046	\\
3.85952	0.0581672	\\
3.87565	0.0583867	\\
3.90046	0.0587219	\\
3.91193	0.0588759	\\
3.93007	0.0591183	\\
3.9543	0.0594397	\\
3.97897	0.0597642	\\
3.99471	0.05997	\\
4.01354	0.0602145	\\
4.02971	0.0604235	\\
4.0409	0.0605673	\\
4.05891	0.0607978	\\
4.07359	0.0609845	\\
4.08099	0.0610783	\\
4.10528	0.0613847	\\
4.1273	0.0616603	\\
4.14039	0.0618232	\\
4.1655	0.0621336	\\
4.18578	0.0623825	\\
4.21003	0.0626779	\\
4.23799	0.0630156	\\
4.25882	0.0632653	\\
4.27918	0.0635076	\\
4.2957	0.0637031	\\
4.31762	0.0639609	\\
4.33062	0.0641129	\\
4.35793	0.0644302	\\
4.37829	0.0646649	\\
4.4087	0.0650127	\\
4.42985	0.0652526	\\
4.44854	0.0654632	\\
4.46135	0.0656069	\\
4.48603	0.065882	\\
4.51	0.0661471	\\
4.53141	0.0663823	\\
4.55237	0.0666109	\\
4.5749	0.0668551	\\
4.59724	0.0670954	\\
4.60872	0.0672184	\\
4.62886	0.0674329	\\
4.64609	0.0676154	\\
4.70087	0.0681891	\\
4.72547	0.0684435	\\
4.74041	0.0685972	\\
4.75	0.0686955	\\
4.7787	0.0689877	\\
4.78755	0.0690774	\\
4.80747	0.0692782	\\
4.83792	0.0695829	\\
4.85137	0.0697166	\\
4.87628	0.0699628	\\
4.89058	0.0701033	\\
4.901	0.0702053	\\
4.91963	0.0703868	\\
4.94708	0.0706525	\\
4.96503	0.070825	\\
4.98023	0.0709704	\\
5.00062	0.0711646	\\
5.03209	0.0714617	\\
5.04733	0.0716046	\\
5.05957	0.0717189	\\
5.089	0.0719922	\\
5.11273	0.0722108	\\
5.139	0.072451	\\
5.17036	0.0727354	\\
5.20238	0.0730231	\\
5.22068	0.0731864	\\
5.24171	0.0733729	\\
5.27082	0.0736294	\\
5.29698	0.0738579	\\
5.3122	0.0739901	\\
5.33363	0.0741753	\\
5.35787	0.0743834	\\
5.38709	0.0746324	\\
5.41346	0.0748554	\\
5.4475	0.0751407	\\
5.48522	0.0754538	\\
5.50637	0.0756279	\\
5.5248	0.0757788	\\
5.55313	0.0760092	\\
5.57634	0.0761966	\\
5.60366	0.0764157	\\
5.62191	0.0765612	\\
5.65958	0.0768592	\\
5.69471	0.0771344	\\
5.71707	0.0773081	\\
5.73437	0.0774418	\\
5.75999	0.0776387	\\
5.78241	0.0778099	\\
5.80886	0.0780106	\\
5.82328	0.0781193	\\
5.85852	0.0783834	\\
5.88786	0.0786015	\\
5.917	0.0788163	\\
5.93246	0.0789296	\\
5.96588	0.079173	\\
5.99341	0.0793719	\\
6.011	0.0794983	\\
6.03999	0.0797052	\\
6.06491	0.0798819	\\
6.09937	0.0801243	\\
6.12948	0.0803344	\\
6.16636	0.0805894	\\
6.19125	0.0807603	\\
6.21746	0.0809389	\\
6.24477	0.0811238	\\
6.26611	0.0812675	\\
6.28719	0.0814086	\\
6.3129	0.0815796	\\
6.34627	0.0818001	\\
6.37139	0.0819648	\\
6.39363	0.0821097	\\
6.41025	0.0822176	\\
6.43865	0.0824007	\\
6.47916	0.0826598	\\
6.5102	0.0828566	\\
6.52663	0.0829601	\\
6.57432	0.0832583	\\
6.61072	0.0834836	\\
6.63516	0.0836338	\\
6.68209	0.0839196	\\
6.70497	0.0840578	\\
6.7332	0.0842272	\\
6.76038	0.0843893	\\
6.79327	0.084584	\\
6.82495	0.0847701	\\
6.86837	0.0850229	\\
6.88989	0.0851473	\\
6.91276	0.0852788	\\
6.93727	0.0854189	\\
6.99222	0.0857301	\\
7.0143	0.085854	\\
7.03642	0.0859775	\\
7.07477	0.0861902	\\
7.09945	0.086326	\\
7.12214	0.0864503	\\
7.15202	0.0866129	\\
7.18136	0.0867714	\\
7.22839	0.0870234	\\
7.29291	0.0873649	\\
7.32937	0.0875556	\\
7.34602	0.0876421	\\
7.39887	0.0879149	\\
7.42534	0.0880502	\\
7.45516	0.0882018	\\
7.48223	0.0883385	\\
7.51391	0.0884975	\\
7.56059	0.0887297	\\
7.6048	0.0889474	\\
7.62181	0.0890306	\\
7.64773	0.0891569	\\
7.69234	0.0893725	\\
7.73359	0.08957	\\
7.77596	0.0897711	\\
7.82277	0.0899911	\\
7.86155	0.0901717	\\
7.89303	0.0903172	\\
7.93264	0.0904989	\\
7.97912	0.0907103	\\
8.00993	0.0908493	\\
8.05018	0.0910294	\\
8.09962	0.0912487	\\
8.13586	0.091408	\\
8.17892	0.0915958	\\
8.2152	0.0917527	\\
8.24147	0.0918655	\\
8.27699	0.0920172	\\
8.31588	0.092182	\\
8.36978	0.0924083	\\
8.39427	0.0925103	\\
8.43448	0.0926767	\\
8.47906	0.0928596	\\
8.50451	0.0929633	\\
8.52412	0.0930428	\\
8.57725	0.0932567	\\
8.60789	0.0933791	\\
8.6428	0.0935176	\\
8.70304	0.0937544	\\
8.76292	0.093987	\\
8.79102	0.0940953	\\
8.8193	0.0942036	\\
8.86518	0.0943781	\\
8.91599	0.0945696	\\
8.96533	0.0947538	\\
8.99896	0.0948783	\\
9.03091	0.0949959	\\
9.06498	0.0951205	\\
9.11212	0.0952916	\\
9.15748	0.0954548	\\
9.20151	0.0956118	\\
9.2463	0.0957703	\\
9.30386	0.095972	\\
9.35036	0.0961335	\\
9.41429	0.0963531	\\
9.45702	0.0964985	\\
9.49771	0.0966359	\\
9.55248	0.0968193	\\
9.61597	0.0970296	\\
9.68052	0.0972409	\\
9.70768	0.0973291	\\
9.7569	0.0974879	\\
9.79248	0.0976018	\\
9.83031	0.0977221	\\
9.87647	0.0978679	\\
9.90852	0.0979684	\\
9.95018	0.0980981	\\
};
\end{axis}
\end{tikzpicture}%

        \end{subfigure}
        \hfill
        \begin{subfigure}[b]{0.45\textwidth}
            ~
            % % This file was created by matlab2tikz v0.4.6 running on MATLAB 8.1.
% Copyright (c) 2008--2014, Nico Schlömer <nico.schloemer@gmail.com>
% All rights reserved.
% Minimal pgfplots version: 1.3
%
% The latest updates can be retrieved from
%   http://www.mathworks.com/matlabcentral/fileexchange/22022-matlab2tikz
% where you can also make suggestions and rate matlab2tikz.
%
\begin{tikzpicture}

\begin{axis}[%
width=\figurewidth,
height=\figureheight,
scale only axis,
xmin=0,
xmax=10,
ymode=log,
ymin=1e-06,
ymax=1,
yminorticks=false
]
\addplot [color=red,solid,forget plot]
  table[row sep=crcr]{
0.100016	2.72923e-15	\\
0.100732	2.4432e-05	\\
0.101143	5.97004e-05	\\
0.101868	0.000157632	\\
0.102291	0.000234753	\\
0.102765	0.000337776	\\
0.103121	0.000426202	\\
0.103483	0.000525205	\\
0.103876	0.00064331	\\
0.104476	0.000842849	\\
0.104665	0.000910395	\\
0.105362	0.00117809	\\
0.105744	0.00133691	\\
0.106062	0.001475	\\
0.106298	0.00158105	\\
0.106648	0.00174342	\\
0.10715	0.00198647	\\
0.107607	0.00221838	\\
0.108043	0.00244802	\\
0.108317	0.00259629	\\
0.108894	0.00291869	\\
0.109224	0.00310916	\\
0.109604	0.00333306	\\
0.110172	0.00367696	\\
0.110505	0.00388308	\\
0.110927	0.00414996	\\
0.111227	0.00434241	\\
0.111839	0.00474374	\\
0.112341	0.00507999	\\
0.112781	0.00537979	\\
0.113469	0.00585709	\\
0.113904	0.00616397	\\
0.114661	0.00670654	\\
0.115014	0.00696254	\\
0.115804	0.0075436	\\
0.11616	0.00780804	\\
0.11676	0.00825764	\\
0.1172	0.00859018	\\
0.117616	0.00890606	\\
0.118117	0.00928888	\\
0.118705	0.00974073	\\
0.119312	0.01021	\\
0.119771	0.0105662	\\
0.120376	0.0110367	\\
0.121135	0.0116302	\\
0.121529	0.0119384	\\
0.122052	0.0123487	\\
0.12263	0.0128021	\\
0.123087	0.0131602	\\
0.123626	0.0135836	\\
0.124085	0.0139428	\\
0.124676	0.0144062	\\
0.125112	0.0147471	\\
0.125592	0.015122	\\
0.126054	0.0154817	\\
0.126291	0.0156666	\\
0.126904	0.0161424	\\
0.127492	0.016597	\\
0.127792	0.0168281	\\
0.128534	0.017398	\\
0.129167	0.0178818	\\
0.129774	0.018343	\\
0.130294	0.0187357	\\
0.130818	0.0191296	\\
0.131487	0.0196296	\\
0.132056	0.0200517	\\
0.132644	0.0204853	\\
0.133001	0.0207466	\\
0.133666	0.0212314	\\
0.134361	0.0217326	\\
0.13507	0.0222394	\\
0.135588	0.0226071	\\
0.135991	0.0228907	\\
0.13657	0.0232953	\\
0.13714	0.0236899	\\
0.13769	0.0240677	\\
0.138229	0.0244347	\\
0.138825	0.0248366	\\
0.139339	0.0251806	\\
0.140146	0.0257133	\\
0.140895	0.0262019	\\
0.141373	0.0265097	\\
0.141897	0.0268448	\\
0.142535	0.0272474	\\
0.143589	0.0279031	\\
0.144197	0.0282747	\\
0.144743	0.028605	\\
0.145248	0.028907	\\
0.145703	0.029177	\\
0.146385	0.0295758	\\
0.147035	0.0299516	\\
0.147801	0.0303869	\\
0.148257	0.0306429	\\
0.149005	0.0310569	\\
0.149751	0.0314632	\\
0.150789	0.0320173	\\
0.151369	0.032321	\\
0.15236	0.0328304	\\
0.153206	0.033256	\\
0.153923	0.0336099	\\
0.154616	0.0339461	\\
0.155702	0.0344612	\\
0.156276	0.0347279	\\
0.157109	0.0351078	\\
0.157913	0.0354667	\\
0.158792	0.0358505	\\
0.159332	0.0360821	\\
0.160076	0.036395	\\
0.160613	0.0366173	\\
0.161543	0.0369941	\\
0.161952	0.0371568	\\
0.162943	0.0375434	\\
0.163795	0.0378669	\\
0.164577	0.0381565	\\
0.165563	0.0385128	\\
0.166526	0.0388505	\\
0.167275	0.0391062	\\
0.167903	0.0393161	\\
0.168356	0.0394652	\\
0.169313	0.0397729	\\
0.170372	0.0401025	\\
0.171347	0.0403959	\\
0.172157	0.0406326	\\
0.172836	0.040826	\\
0.173425	0.0409902	\\
0.173801	0.0410932	\\
0.174706	0.0413359	\\
0.175346	0.0415029	\\
0.176423	0.0417755	\\
0.177117	0.0419453	\\
0.178062	0.0421698	\\
0.178898	0.0423618	\\
0.180102	0.0426275	\\
0.180862	0.0427889	\\
0.182027	0.0430268	\\
0.184089	0.0434205	\\
0.18487	0.0435608	\\
0.185944	0.0437459	\\
0.18661	0.0438561	\\
0.187504	0.0439989	\\
0.188082	0.0440879	\\
0.188911	0.0442115	\\
0.189884	0.0443502	\\
0.190693	0.0444603	\\
0.191648	0.0445843	\\
0.19297	0.0447457	\\
0.194005	0.0448639	\\
0.195319	0.0450038	\\
0.19641	0.0451116	\\
0.197237	0.0451883	\\
0.198188	0.0452713	\\
0.198714	0.0453148	\\
0.199929	0.0454092	\\
0.200903	0.0454785	\\
0.201795	0.0455373	\\
0.202715	0.0455932	\\
0.203536	0.0456393	\\
0.204572	0.0456921	\\
0.205426	0.0457314	\\
0.207216	0.0458016	\\
0.207983	0.0458267	\\
0.208602	0.0458449	\\
0.209736	0.0458734	\\
0.210653	0.045892	\\
0.211348	0.0459035	\\
0.212512	0.0459179	\\
0.213482	0.0459252	\\
0.214593	0.0459286	\\
0.215334	0.045928	\\
0.216055	0.0459251	\\
0.217124	0.045917	\\
0.21783	0.0459091	\\
0.219062	0.0458906	\\
0.219841	0.0458759	\\
0.220808	0.0458546	\\
0.222028	0.0458227	\\
0.222801	0.0457997	\\
0.223564	0.0457749	\\
0.224473	0.0457428	\\
0.225435	0.0457058	\\
0.226929	0.0456423	\\
0.228105	0.0455873	\\
0.229383	0.0455226	\\
0.230274	0.0454746	\\
0.231502	0.0454047	\\
0.23291	0.0453192	\\
0.233592	0.0452758	\\
0.234564	0.0452118	\\
0.235642	0.045138	\\
0.237129	0.0450311	\\
0.238486	0.0449288	\\
0.239473	0.0448517	\\
0.240467	0.0447716	\\
0.241136	0.0447165	\\
0.242537	0.0445979	\\
0.244035	0.0444663	\\
0.245076	0.0443721	\\
0.246009	0.0442858	\\
0.247	0.0441923	\\
0.247885	0.0441072	\\
0.249225	0.0439756	\\
0.250428	0.0438546	\\
0.251459	0.043749	\\
0.252329	0.0436585	\\
0.253517	0.0435327	\\
0.254847	0.0433894	\\
0.255904	0.0432735	\\
0.257098	0.0431405	\\
0.258282	0.0430067	\\
0.259441	0.0428738	\\
0.260833	0.0427119	\\
0.262131	0.0425586	\\
0.262987	0.0424564	\\
0.263852	0.0423521	\\
0.264951	0.0422184	\\
0.266916	0.0419761	\\
0.268033	0.0418364	\\
0.269649	0.0416321	\\
0.271025	0.0414561	\\
0.272281	0.041294	\\
0.274193	0.0410446	\\
0.275028	0.0409347	\\
0.276161	0.0407846	\\
0.27739	0.0406208	\\
0.278326	0.0404952	\\
0.279306	0.0403632	\\
0.280599	0.0401878	\\
0.281796	0.0400245	\\
0.282404	0.0399413	\\
0.282916	0.0398709	\\
0.284345	0.039674	\\
0.286539	0.0393693	\\
0.288465	0.0390998	\\
0.289577	0.0389435	\\
0.290789	0.0387725	\\
0.291891	0.0386164	\\
0.292729	0.0384975	\\
0.293549	0.0383808	\\
0.294692	0.0382178	\\
0.29527	0.0381352	\\
0.296798	0.0379163	\\
0.298342	0.0376944	\\
0.299756	0.0374906	\\
0.301579	0.0372274	\\
0.302548	0.0370871	\\
0.304394	0.0368194	\\
0.306557	0.036505	\\
0.308453	0.0362289	\\
0.310287	0.0359615	\\
0.311845	0.0357342	\\
0.313014	0.0355635	\\
0.314818	0.0353001	\\
0.316266	0.0350885	\\
0.317626	0.0348898	\\
0.319309	0.0346441	\\
0.320928	0.0344077	\\
0.322521	0.0341753	\\
0.323438	0.0340416	\\
0.324975	0.0338176	\\
0.32696	0.0335287	\\
0.328393	0.0333205	\\
0.329228	0.0331993	\\
0.330537	0.0330094	\\
0.331616	0.0328531	\\
0.332773	0.0326858	\\
0.334735	0.0324025	\\
0.335838	0.0322434	\\
0.336532	0.0321435	\\
0.338708	0.0318309	\\
0.339764	0.0316796	\\
0.340906	0.0315162	\\
0.342263	0.0313224	\\
0.344496	0.0310045	\\
0.346563	0.0307113	\\
0.348333	0.0304612	\\
0.349845	0.0302482	\\
0.352231	0.0299134	\\
0.353715	0.029706	\\
0.355412	0.0294697	\\
0.35655	0.0293116	\\
0.358077	0.0291003	\\
0.360099	0.0288217	\\
0.361818	0.0285858	\\
0.364139	0.0282689	\\
0.36546	0.0280893	\\
0.367669	0.0277906	\\
0.369975	0.0274805	\\
0.371557	0.0272688	\\
0.372694	0.0271173	\\
0.373774	0.0269739	\\
0.375262	0.026777	\\
0.377306	0.0265077	\\
0.380459	0.0260957	\\
0.382933	0.0257751	\\
0.384678	0.0255505	\\
0.385902	0.0253936	\\
0.388727	0.0250339	\\
0.391005	0.0247461	\\
0.393414	0.0244442	\\
0.395144	0.0242288	\\
0.396611	0.0240472	\\
0.398773	0.0237812	\\
0.400663	0.0235503	\\
0.403071	0.0232582	\\
0.404874	0.0230411	\\
0.406304	0.0228699	\\
0.407502	0.0227271	\\
0.408969	0.0225532	\\
0.410986	0.0223156	\\
0.413545	0.0220165	\\
0.415279	0.0218155	\\
0.418128	0.021488	\\
0.420064	0.0212674	\\
0.421823	0.0210683	\\
0.424593	0.0207577	\\
0.426398	0.020557	\\
0.428145	0.0203641	\\
0.430117	0.020148	\\
0.431719	0.0199736	\\
0.434004	0.0197268	\\
0.435639	0.0195515	\\
0.438002	0.0193003	\\
0.44003	0.0190866	\\
0.441192	0.018965	\\
0.444417	0.0186303	\\
0.446955	0.01837	\\
0.44909	0.0181531	\\
0.451835	0.0178772	\\
0.453309	0.0177303	\\
0.454813	0.0175813	\\
0.457958	0.0172729	\\
0.459862	0.0170881	\\
0.462247	0.0168588	\\
0.464619	0.0166331	\\
0.465952	0.0165073	\\
0.467532	0.0163591	\\
0.468681	0.0162519	\\
0.471989	0.0159464	\\
0.47471	0.0156984	\\
0.476291	0.0155556	\\
0.477826	0.015418	\\
0.479499	0.0152691	\\
0.482866	0.0149726	\\
0.485005	0.0147866	\\
0.486881	0.0146249	\\
0.489435	0.014407	\\
0.491926	0.0141969	\\
0.495021	0.013939	\\
0.498525	0.0136515	\\
0.50063	0.013481	\\
0.504166	0.0131983	\\
0.505665	0.0130799	\\
0.507841	0.0129093	\\
0.510741	0.0126848	\\
0.512662	0.0125377	\\
0.514519	0.0123968	\\
0.517737	0.0121554	\\
0.520304	0.0119655	\\
0.522653	0.0117938	\\
0.525222	0.0116082	\\
0.527856	0.0114202	\\
0.529053	0.0113356	\\
0.530345	0.0112447	\\
0.532671	0.0110826	\\
0.537014	0.0107848	\\
0.539905	0.01059	\\
0.542539	0.0104148	\\
0.545665	0.0102099	\\
0.548282	0.0100408	\\
0.55136	0.00984459	\\
0.553736	0.0096952	\\
0.557423	0.00946677	\\
0.559872	0.0093174	\\
0.563057	0.00912579	\\
0.566709	0.00890979	\\
0.568692	0.00879418	\\
0.570944	0.00866428	\\
0.572341	0.00858441	\\
0.576713	0.0083382	\\
0.578029	0.00826512	\\
0.58042	0.00813363	\\
0.583278	0.0079786	\\
0.58787	0.00773413	\\
0.591899	0.00752442	\\
0.596812	0.00727448	\\
0.598832	0.00717359	\\
0.602077	0.00701373	\\
0.603788	0.00693051	\\
0.605351	0.00685512	\\
0.609567	0.00665491	\\
0.612397	0.00652302	\\
0.613894	0.00645406	\\
0.617595	0.00628585	\\
0.618871	0.00622864	\\
0.621349	0.00611863	\\
0.623221	0.00603653	\\
0.625753	0.00592673	\\
0.630458	0.00572668	\\
0.631941	0.00566468	\\
0.636069	0.00549472	\\
0.638582	0.00539313	\\
0.642931	0.00522058	\\
0.646184	0.00509418	\\
0.649208	0.00497875	\\
0.651934	0.00487631	\\
0.655829	0.00473266	\\
0.659188	0.00461128	\\
0.661905	0.00451476	\\
0.664688	0.00441747	\\
0.668699	0.00427989	\\
0.673064	0.00413373	\\
0.675007	0.00406983	\\
0.677923	0.00397529	\\
0.680482	0.00389363	\\
0.683106	0.00381115	\\
0.686506	0.00370615	\\
0.68944	0.00361721	\\
0.695155	0.00344837	\\
0.699714	0.00331776	\\
0.701674	0.00326268	\\
0.705204	0.00316513	\\
0.708283	0.00308176	\\
0.710861	0.00301314	\\
0.713441	0.00294554	\\
0.716795	0.00285925	\\
0.718886	0.00280636	\\
0.722192	0.00272414	\\
0.725998	0.00263157	\\
0.730187	0.00253223	\\
0.73199	0.00249027	\\
0.736732	0.00238222	\\
0.741189	0.00228363	\\
0.743401	0.00223575	\\
0.74612	0.00217784	\\
0.750475	0.00208723	\\
0.753352	0.0020288	\\
0.755676	0.0019824	\\
0.757967	0.00193739	\\
0.762502	0.00185029	\\
0.766831	0.00176964	\\
0.771437	0.00168643	\\
0.77395	0.00164215	\\
0.776972	0.00158993	\\
0.781398	0.00151544	\\
0.784609	0.00146288	\\
0.787748	0.00141264	\\
0.791038	0.00136124	\\
0.794143	0.00131384	\\
0.79783	0.00125897	\\
0.802238	0.00119534	\\
0.806834	0.00113125	\\
0.809551	0.00109443	\\
0.812533	0.00105488	\\
0.81617	0.00100793	\\
0.820285	0.000956406	\\
0.824309	0.000907675	\\
0.826707	0.000879395	\\
0.833319	0.000804288	\\
0.835586	0.000779495	\\
0.838401	0.00074938	\\
0.842981	0.000701939	\\
0.847488	0.000657106	\\
0.851709	0.000616751	\\
0.855601	0.000580913	\\
0.860051	0.000541534	\\
0.864038	0.000507654	\\
0.867779	0.000477053	\\
0.870037	0.000459136	\\
0.87366	0.000431237	\\
0.880088	0.00038427	\\
0.884857	0.000351477	\\
0.889037	0.000324136	\\
0.893547	0.000296075	\\
0.896011	0.000281365	\\
0.900526	0.000255526	\\
0.903659	0.000238434	\\
0.907433	0.000218739	\\
0.912219	0.000195151	\\
0.913714	0.00018809	\\
0.915829	0.000178359	\\
0.918211	0.000167742	\\
0.922562	0.000149292	\\
0.926574	0.00013334	\\
0.93001	0.000120471	\\
0.939074	8.99435e-05	\\
0.94474	7.33111e-05	\\
0.947393	6.61546e-05	\\
0.951158	5.66747e-05	\\
0.957049	4.34123e-05	\\
0.963964	3.02196e-05	\\
0.967383	2.46204e-05	\\
0.969213	2.18708e-05	\\
0.974442	1.49469e-05	\\
0.979112	9.90925e-06	\\
0.985362	4.81808e-06	\\
0.987692	3.39335e-06	\\
0.991722	1.52546e-06	\\
0.996869	2.16438e-07	\\
1.00069	1.05944e-08	\\
1.00686	1.03015e-06	\\
1.0114	2.83556e-06	\\
1.01564	5.32801e-06	\\
1.01883	7.7032e-06	\\
1.02234	1.08193e-05	\\
1.02492	1.34425e-05	\\
1.02894	1.80908e-05	\\
1.03158	2.15084e-05	\\
1.0349	2.62171e-05	\\
1.04362	4.07376e-05	\\
1.04669	4.65948e-05	\\
1.05053	5.44572e-05	\\
1.05595	6.65483e-05	\\
1.06253	8.28089e-05	\\
1.06742	9.59884e-05	\\
1.07338	0.000113311	\\
1.07781	0.000127049	\\
1.08253	0.000142546	\\
1.08601	0.000154504	\\
1.09	0.000168786	\\
1.09294	0.000179645	\\
1.09738	0.000196704	\\
1.102	0.000215239	\\
1.10895	0.000244547	\\
1.11633	0.000277598	\\
1.12186	0.000303592	\\
1.12619	0.000324743	\\
1.1329	0.000358727	\\
1.13615	0.000375769	\\
1.14014	0.000397194	\\
1.14299	0.000412799	\\
1.14711	0.000435849	\\
1.15176	0.000462601	\\
1.15845	0.000502231	\\
1.16166	0.000521806	\\
1.16882	0.000566611	\\
1.1752	0.000607984	\\
1.17957	0.000636992	\\
1.18216	0.000654525	\\
1.19114	0.000716822	\\
1.19497	0.000744135	\\
1.20001	0.000780722	\\
1.20526	0.000819688	\\
1.21056	0.000859848	\\
1.21672	0.000907574	\\
1.22073	0.000939183	\\
1.23028	0.00101644	\\
1.23784	0.00107939	\\
1.24459	0.00113698	\\
1.24788	0.00116545	\\
1.25128	0.00119521	\\
1.25545	0.00123221	\\
1.25793	0.00125439	\\
1.26242	0.00129494	\\
1.26925	0.00135769	\\
1.27407	0.00140277	\\
1.27764	0.0014365	\\
1.28445	0.00150172	\\
1.28897	0.00154571	\\
1.29525	0.00160761	\\
1.3049	0.00170465	\\
1.30871	0.00174361	\\
1.31864	0.00184664	\\
1.32815	0.00194759	\\
1.34101	0.00208725	\\
1.3439	0.00211916	\\
1.34954	0.002182	\\
1.35348	0.00222622	\\
1.35862	0.00228455	\\
1.36647	0.00237469	\\
1.37219	0.00244117	\\
1.3777	0.00250585	\\
1.38335	0.00257275	\\
1.38883	0.00263834	\\
1.39808	0.00275037	\\
1.40462	0.00283065	\\
1.40889	0.00288351	\\
1.41678	0.002982	\\
1.42489	0.00308455	\\
1.4286	0.00313187	\\
1.43541	0.00321931	\\
1.44356	0.0033251	\\
1.447	0.00337016	\\
1.45273	0.00344558	\\
1.45783	0.00351317	\\
1.46261	0.00357696	\\
1.46644	0.00362831	\\
1.47306	0.00371771	\\
1.48112	0.00382757	\\
1.48631	0.00389882	\\
1.49152	0.00397089	\\
1.498	0.00406097	\\
1.50889	0.00421389	\\
1.51635	0.00431982	\\
1.52411	0.00443076	\\
1.53298	0.00455865	\\
1.53867	0.00464124	\\
1.54339	0.00471014	\\
1.55362	0.00486066	\\
1.56628	0.00504866	\\
1.57436	0.00516988	\\
1.58248	0.00529253	\\
1.58694	0.0053603	\\
1.59426	0.00547191	\\
1.6016	0.00558451	\\
1.60963	0.00570858	\\
1.61845	0.00584563	\\
1.62336	0.00592232	\\
1.63086	0.00604002	\\
1.63692	0.00613558	\\
1.64619	0.00628251	\\
1.65239	0.00638126	\\
1.65937	0.00649301	\\
1.665	0.00658349	\\
1.67189	0.00669468	\\
1.67556	0.00675407	\\
1.68271	0.00687033	\\
1.68987	0.00698707	\\
1.69877	0.00713302	\\
1.70866	0.007296	\\
1.71442	0.00739131	\\
1.72011	0.00748588	\\
1.72704	0.00760127	\\
1.73769	0.00777947	\\
1.74769	0.00794782	\\
1.75454	0.00806351	\\
1.76234	0.00819576	\\
1.77118	0.00834612	\\
1.78279	0.00854477	\\
1.79008	0.00866991	\\
1.80116	0.00886075	\\
1.80927	0.00900117	\\
1.81677	0.00913139	\\
1.8269	0.00930776	\\
1.83609	0.00946831	\\
1.84246	0.00958002	\\
1.84909	0.00969643	\\
1.86067	0.0099005	\\
1.86797	0.0100297	\\
1.87509	0.0101558	\\
1.88462	0.0103251	\\
1.89612	0.0105301	\\
1.90382	0.0106676	\\
1.91835	0.0109281	\\
1.92937	0.0111264	\\
1.94517	0.0114117	\\
1.95358	0.011564	\\
1.96322	0.011739	\\
1.97067	0.0118744	\\
1.97847	0.0120164	\\
1.99118	0.0122484	\\
1.99933	0.0123975	\\
2.00577	0.0125154	\\
2.01732	0.0127273	\\
2.02421	0.0128539	\\
2.03162	0.0129903	\\
2.0416	0.0131743	\\
2.05159	0.0133586	\\
2.06066	0.0135263	\\
2.07203	0.0137368	\\
2.07753	0.0138388	\\
2.08707	0.0140157	\\
2.09452	0.0141542	\\
2.10783	0.0144017	\\
2.1198	0.0146247	\\
2.12924	0.0148007	\\
2.14501	0.0150952	\\
2.15928	0.015362	\\
2.17265	0.0156123	\\
2.19081	0.0159527	\\
2.2014	0.0161515	\\
2.21006	0.016314	\\
2.21576	0.0164211	\\
2.22677	0.016628	\\
2.23651	0.0168112	\\
2.25202	0.0171028	\\
2.26277	0.0173051	\\
2.27758	0.017584	\\
2.2885	0.0177897	\\
2.2988	0.0179838	\\
2.30861	0.0181684	\\
2.31522	0.0182932	\\
2.32688	0.0185128	\\
2.34351	0.0188262	\\
2.35324	0.0190095	\\
2.36576	0.0192455	\\
2.36978	0.0193213	\\
2.3798	0.0195101	\\
2.39165	0.0197334	\\
2.40418	0.0199693	\\
2.41219	0.0201201	\\
2.42786	0.020415	\\
2.43743	0.020595	\\
2.44472	0.0207321	\\
2.45551	0.020935	\\
2.46508	0.0211149	\\
2.47591	0.0213184	\\
2.49047	0.0215917	\\
2.50494	0.0218632	\\
2.51629	0.0220762	\\
2.53182	0.022367	\\
2.54063	0.022532	\\
2.55441	0.0227897	\\
2.56632	0.0230122	\\
2.57583	0.0231899	\\
2.59765	0.0235968	\\
2.61636	0.0239453	\\
2.62939	0.0241875	\\
2.64229	0.0244272	\\
2.65063	0.0245821	\\
2.66488	0.0248461	\\
2.67204	0.0249788	\\
2.68853	0.0252838	\\
2.70159	0.0255252	\\
2.71485	0.0257696	\\
2.72522	0.0259609	\\
2.73163	0.0260788	\\
2.74114	0.0262537	\\
2.75287	0.0264692	\\
2.75762	0.0265563	\\
2.76946	0.0267734	\\
2.79324	0.0272085	\\
2.80245	0.0273768	\\
2.82032	0.0277026	\\
2.83198	0.0279148	\\
2.84039	0.0280677	\\
2.85568	0.0283452	\\
2.87357	0.0286691	\\
2.88426	0.0288623	\\
2.90311	0.0292024	\\
2.91433	0.0294045	\\
2.91981	0.0295029	\\
2.93285	0.0297371	\\
2.94845	0.0300167	\\
2.95951	0.0302146	\\
2.97174	0.030433	\\
2.98385	0.030649	\\
3.00019	0.0309397	\\
3.00996	0.0311131	\\
3.02438	0.0313687	\\
3.03676	0.0315878	\\
3.05065	0.0318329	\\
3.06015	0.0320004	\\
3.07312	0.0322285	\\
3.09063	0.0325358	\\
3.10426	0.0327744	\\
3.11856	0.0330241	\\
3.13097	0.0332403	\\
3.14182	0.0334292	\\
3.15581	0.033672	\\
3.17855	0.0340655	\\
3.1944	0.034339	\\
3.20509	0.0345231	\\
3.21843	0.0347522	\\
3.23677	0.0350664	\\
3.25536	0.0353839	\\
3.27116	0.035653	\\
3.28145	0.0358279	\\
3.28776	0.0359349	\\
3.29755	0.0361007	\\
3.31668	0.036424	\\
3.33641	0.0367563	\\
3.35121	0.0370047	\\
3.36002	0.0371522	\\
3.37035	0.0373251	\\
3.38202	0.0375198	\\
3.40333	0.0378744	\\
3.41024	0.0379892	\\
3.42871	0.0382949	\\
3.4392	0.0384683	\\
3.45	0.0386463	\\
3.47698	0.0390893	\\
3.48591	0.0392355	\\
3.49916	0.0394519	\\
3.51073	0.0396406	\\
3.53144	0.0399771	\\
3.54278	0.0401607	\\
3.55771	0.040402	\\
3.56828	0.0405724	\\
3.59019	0.0409245	\\
3.61177	0.0412698	\\
3.62231	0.041438	\\
3.63224	0.041596	\\
3.64821	0.0418496	\\
3.67865	0.0423309	\\
3.6901	0.0425112	\\
3.70816	0.0427948	\\
3.72825	0.0431089	\\
3.74526	0.0433742	\\
3.75968	0.0435981	\\
3.784	0.0439745	\\
3.79747	0.0441821	\\
3.81243	0.0444122	\\
3.82783	0.0446482	\\
3.84763	0.0449508	\\
3.85952	0.0451318	\\
3.87565	0.0453768	\\
3.90046	0.0457519	\\
3.91193	0.0459247	\\
3.93007	0.0461972	\\
3.9543	0.0465595	\\
3.97897	0.0469266	\\
3.99471	0.04716	\\
4.01354	0.047438	\\
4.02971	0.0476761	\\
4.0409	0.0478403	\\
4.05891	0.0481039	\\
4.07359	0.0483178	\\
4.08099	0.0484255	\\
4.10528	0.0487779	\\
4.1273	0.0490958	\\
4.14039	0.049284	\\
4.1655	0.0496436	\\
4.18578	0.0499329	\\
4.21003	0.050277	\\
4.23799	0.0506718	\\
4.25882	0.0509644	\\
4.27918	0.0512492	\\
4.2957	0.0514793	\\
4.31762	0.0517836	\\
4.33062	0.0519633	\\
4.35793	0.0523394	\\
4.37829	0.0526184	\\
4.4087	0.0530329	\\
4.42985	0.0533197	\\
4.44854	0.0535721	\\
4.46135	0.0537445	\\
4.48603	0.0540753	\\
4.51	0.0543949	\\
4.53141	0.0546792	\\
4.55237	0.0549562	\\
4.5749	0.0552527	\\
4.59724	0.0555451	\\
4.60872	0.055695	\\
4.62886	0.0559569	\\
4.64609	0.0561802	\\
4.70087	0.0568846	\\
4.72547	0.0571983	\\
4.74041	0.0573881	\\
4.75	0.0575096	\\
4.7787	0.0578717	\\
4.78755	0.057983	\\
4.80747	0.0582326	\\
4.83792	0.0586122	\\
4.85137	0.0587791	\\
4.87628	0.0590871	\\
4.89058	0.0592631	\\
4.901	0.059391	\\
4.91963	0.059619	\\
4.94708	0.0599535	\\
4.96503	0.0601711	\\
4.98023	0.0603548	\\
5.00062	0.0606004	\\
5.03209	0.0609773	\\
5.04733	0.0611589	\\
5.05957	0.0613043	\\
5.089	0.0616526	\\
5.11273	0.0619319	\\
5.139	0.0622395	\\
5.17036	0.0626045	\\
5.20238	0.0629747	\\
5.22068	0.0631853	\\
5.24171	0.0634262	\\
5.27082	0.063758	\\
5.29698	0.0640545	\\
5.3122	0.0642263	\\
5.33363	0.0644672	\\
5.35787	0.0647386	\\
5.38709	0.0650638	\\
5.41346	0.0653557	\\
5.4475	0.0657302	\\
5.48522	0.0661421	\\
5.50637	0.0663717	\\
5.5248	0.066571	\\
5.55313	0.0668758	\\
5.57634	0.0671243	\\
5.60366	0.0674152	\\
5.62191	0.0676087	\\
5.65958	0.0680058	\\
5.69471	0.0683736	\\
5.71707	0.0686062	\\
5.73437	0.0687855	\\
5.75999	0.0690499	\\
5.78241	0.0692801	\\
5.80886	0.0695505	\\
5.82328	0.0696972	\\
5.85852	0.0700541	\\
5.88786	0.0703495	\\
5.917	0.070641	\\
5.93246	0.070795	\\
5.96588	0.0711262	\\
5.99341	0.0713975	\\
6.011	0.0715701	\\
6.03999	0.0718532	\\
6.06491	0.0720953	\\
6.09937	0.072428	\\
6.12948	0.072717	\\
6.16636	0.0730686	\\
6.19125	0.0733045	\\
6.21746	0.0735516	\\
6.24477	0.0738078	\\
6.26611	0.0740071	\\
6.28719	0.074203	\\
6.3129	0.0744409	\\
6.34627	0.0747481	\\
6.37139	0.074978	\\
6.39363	0.0751806	\\
6.41025	0.0753315	\\
6.43865	0.0755881	\\
6.47916	0.0759517	\\
6.5102	0.0762285	\\
6.52663	0.0763743	\\
6.57432	0.076795	\\
6.61072	0.0771136	\\
6.63516	0.0773263	\\
6.68209	0.0777319	\\
6.70497	0.0779284	\\
6.7332	0.0781695	\\
6.76038	0.0784006	\\
6.79327	0.0786786	\\
6.82495	0.0789448	\\
6.86837	0.0793071	\\
6.88989	0.0794856	\\
6.91276	0.0796745	\\
6.93727	0.079876	\\
6.99222	0.0803245	\\
7.0143	0.0805034	\\
7.03642	0.080682	\\
7.07477	0.0809899	\\
7.09945	0.0811868	\\
7.12214	0.0813672	\\
7.15202	0.0816034	\\
7.18136	0.0818342	\\
7.22839	0.0822016	\\
7.29291	0.0827005	\\
7.32937	0.0829798	\\
7.34602	0.0831068	\\
7.39887	0.0835074	\\
7.42534	0.0837065	\\
7.45516	0.0839298	\\
7.48223	0.0841314	\\
7.51391	0.0843662	\\
7.56059	0.0847097	\\
7.6048	0.0850324	\\
7.62181	0.0851559	\\
7.64773	0.0853434	\\
7.69234	0.0856641	\\
7.73359	0.0859584	\\
7.77596	0.0862585	\\
7.82277	0.0865874	\\
7.86155	0.0868579	\\
7.89303	0.0870761	\\
7.93264	0.0873491	\\
7.97912	0.087667	\\
8.00993	0.0878764	\\
8.05018	0.0881481	\\
8.09962	0.0884795	\\
8.13586	0.0887206	\\
8.17892	0.0890051	\\
8.2152	0.0892433	\\
8.24147	0.0894148	\\
8.27699	0.0896455	\\
8.31588	0.0898965	\\
8.36978	0.0902417	\\
8.39427	0.0903976	\\
8.43448	0.0906521	\\
8.47906	0.0909322	\\
8.50451	0.0910913	\\
8.52412	0.0912133	\\
8.57725	0.0915421	\\
8.60789	0.0917303	\\
8.6428	0.0919437	\\
8.70304	0.092309	\\
8.76292	0.0926686	\\
8.79102	0.0928362	\\
8.8193	0.093004	\\
8.86518	0.0932747	\\
8.91599	0.0935722	\\
8.96533	0.0938587	\\
8.99896	0.0940527	\\
9.03091	0.0942361	\\
9.06498	0.0944306	\\
9.11212	0.0946979	\\
9.15748	0.0949534	\\
9.20151	0.0951995	\\
9.2463	0.0954482	\\
9.30386	0.0957652	\\
9.35036	0.0960192	\\
9.41429	0.0963655	\\
9.45702	0.096595	\\
9.49771	0.0968121	\\
9.55248	0.0971023	\\
9.61597	0.0974356	\\
9.68052	0.0977712	\\
9.70768	0.0979114	\\
9.7569	0.098164	\\
9.79248	0.0983455	\\
9.83031	0.0985373	\\
9.87647	0.09877	\\
9.90852	0.0989305	\\
9.95018	0.099138	\\
};
\addplot [color=blue,dashed,forget plot]
  table[row sep=crcr]{
0.100016	0	\\
0.100732	2.99995e-06	\\
0.101143	7.35764e-06	\\
0.101868	1.95536e-05	\\
0.102291	2.92302e-05	\\
0.102765	4.22355e-05	\\
0.103121	5.34604e-05	\\
0.103483	6.60889e-05	\\
0.103876	8.12309e-05	\\
0.104476	0.000106986	\\
0.104665	0.00011575	\\
0.105362	0.000150694	\\
0.105744	0.000171574	\\
0.106062	0.000189815	\\
0.106298	0.000203875	\\
0.106648	0.000225487	\\
0.10715	0.000258023	\\
0.107607	0.000289267	\\
0.108043	0.00032039	\\
0.108317	0.000340579	\\
0.108894	0.000384731	\\
0.109224	0.000410972	\\
0.109604	0.000441966	\\
0.110172	0.000489872	\\
0.110505	0.000518756	\\
0.110927	0.000556343	\\
0.111227	0.000583578	\\
0.111839	0.000640716	\\
0.112341	0.000688944	\\
0.112781	0.000732211	\\
0.113469	0.000801614	\\
0.113904	0.000846567	\\
0.114661	0.000926676	\\
0.115014	0.00096475	\\
0.115804	0.00105183	\\
0.11616	0.00109175	\\
0.11676	0.00116007	\\
0.1172	0.00121095	\\
0.117616	0.00125955	\\
0.118117	0.00131881	\\
0.118705	0.00138925	\\
0.119312	0.00146299	\\
0.119771	0.00151937	\\
0.120376	0.00159435	\\
0.121135	0.00168977	\\
0.121529	0.00173971	\\
0.122052	0.0018066	\\
0.12263	0.00188105	\\
0.123087	0.00194025	\\
0.123626	0.00201071	\\
0.124085	0.00207089	\\
0.124676	0.00214906	\\
0.125112	0.00220695	\\
0.125592	0.00227101	\\
0.126054	0.00233286	\\
0.126291	0.0023648	\\
0.126904	0.00244745	\\
0.127492	0.00252707	\\
0.127792	0.00256776	\\
0.128534	0.00266886	\\
0.129167	0.00275545	\\
0.129774	0.0028387	\\
0.130294	0.00291011	\\
0.130818	0.00298226	\\
0.131487	0.00307456	\\
0.132056	0.00315313	\\
0.132644	0.00323446	\\
0.133001	0.0032838	\\
0.133666	0.00337597	\\
0.134361	0.00347213	\\
0.13507	0.0035703	\\
0.135588	0.00364212	\\
0.135991	0.00369786	\\
0.13657	0.00377792	\\
0.13714	0.00385663	\\
0.13769	0.00393255	\\
0.138229	0.00400688	\\
0.138825	0.0040889	\\
0.139339	0.00415964	\\
0.140146	0.00427019	\\
0.140895	0.0043727	\\
0.141373	0.00443786	\\
0.141897	0.00450927	\\
0.142535	0.00459578	\\
0.143589	0.00473834	\\
0.144197	0.00482011	\\
0.144743	0.0048934	\\
0.145248	0.00496089	\\
0.145703	0.00502168	\\
0.146385	0.00511219	\\
0.147035	0.0051983	\\
0.147801	0.00529907	\\
0.148257	0.00535888	\\
0.149005	0.00545644	\\
0.149751	0.00555328	\\
0.150789	0.00568709	\\
0.151369	0.00576133	\\
0.15236	0.00588736	\\
0.153206	0.00599413	\\
0.153923	0.00608398	\\
0.154616	0.0061703	\\
0.155702	0.00630439	\\
0.156276	0.00637474	\\
0.157109	0.00647605	\\
0.157913	0.00657306	\\
0.158792	0.00667819	\\
0.159332	0.0067424	\\
0.160076	0.00683006	\\
0.160613	0.00689302	\\
0.161543	0.00700102	\\
0.161952	0.00704821	\\
0.162943	0.00716165	\\
0.163795	0.00725812	\\
0.164577	0.00734576	\\
0.165563	0.00745526	\\
0.166526	0.0075609	\\
0.167275	0.00764214	\\
0.167903	0.00770972	\\
0.168356	0.00775821	\\
0.169313	0.00785959	\\
0.170372	0.00797037	\\
0.171347	0.00807098	\\
0.172157	0.00815361	\\
0.172836	0.00822215	\\
0.173425	0.00828112	\\
0.173801	0.0083185	\\
0.174706	0.00840776	\\
0.175346	0.00847019	\\
0.176423	0.00857408	\\
0.177117	0.0086401	\\
0.178062	0.00872903	\\
0.178898	0.0088067	\\
0.180102	0.00891686	\\
0.180862	0.00898543	\\
0.182027	0.00908897	\\
0.184089	0.00926782	\\
0.18487	0.00933409	\\
0.185944	0.00942393	\\
0.18661	0.00947884	\\
0.187504	0.00955166	\\
0.188082	0.00959815	\\
0.188911	0.00966414	\\
0.189884	0.00974049	\\
0.190693	0.00980296	\\
0.191648	0.00987565	\\
0.19297	0.00997441	\\
0.194005	0.0100502	\\
0.195319	0.0101444	\\
0.19641	0.010221	\\
0.197237	0.0102781	\\
0.198188	0.0103428	\\
0.198714	0.010378	\\
0.199929	0.0104582	\\
0.200903	0.0105212	\\
0.201795	0.0105779	\\
0.202715	0.0106354	\\
0.203536	0.0106858	\\
0.204572	0.0107483	\\
0.205426	0.010799	\\
0.207216	0.0109023	\\
0.207983	0.0109456	\\
0.208602	0.0109799	\\
0.209736	0.0110418	\\
0.210653	0.0110908	\\
0.211348	0.0111273	\\
0.212512	0.0111872	\\
0.213482	0.011236	\\
0.214593	0.0112907	\\
0.215334	0.0113265	\\
0.216055	0.0113607	\\
0.217124	0.0114105	\\
0.21783	0.0114427	\\
0.219062	0.0114977	\\
0.219841	0.0115317	\\
0.220808	0.011573	\\
0.222028	0.0116238	\\
0.222801	0.0116552	\\
0.223564	0.0116857	\\
0.224473	0.0117212	\\
0.225435	0.011758	\\
0.226929	0.0118133	\\
0.228105	0.0118554	\\
0.229383	0.0118997	\\
0.230274	0.0119296	\\
0.231502	0.0119698	\\
0.23291	0.0120142	\\
0.233592	0.012035	\\
0.234564	0.0120641	\\
0.235642	0.0120953	\\
0.237129	0.0121368	\\
0.238486	0.012173	\\
0.239473	0.0121984	\\
0.240467	0.0122232	\\
0.241136	0.0122394	\\
0.242537	0.0122722	\\
0.244035	0.0123056	\\
0.245076	0.0123278	\\
0.246009	0.0123469	\\
0.247	0.0123665	\\
0.247885	0.0123834	\\
0.249225	0.0124079	\\
0.250428	0.0124288	\\
0.251459	0.0124458	\\
0.252329	0.0124596	\\
0.253517	0.0124776	\\
0.254847	0.0124965	\\
0.255904	0.0125107	\\
0.257098	0.0125258	\\
0.258282	0.0125399	\\
0.259441	0.0125528	\\
0.260833	0.0125671	\\
0.262131	0.0125793	\\
0.262987	0.0125868	\\
0.263852	0.0125939	\\
0.264951	0.0126022	\\
0.266916	0.0126154	\\
0.268033	0.0126218	\\
0.269649	0.0126298	\\
0.271025	0.0126355	\\
0.272281	0.0126397	\\
0.274193	0.0126445	\\
0.275028	0.0126459	\\
0.276161	0.0126473	\\
0.27739	0.012648	\\
0.278326	0.012648	\\
0.279306	0.0126476	\\
0.280599	0.0126463	\\
0.281796	0.0126443	\\
0.282404	0.012643	\\
0.282916	0.0126418	\\
0.284345	0.0126377	\\
0.286539	0.0126296	\\
0.288465	0.0126206	\\
0.289577	0.0126147	\\
0.290789	0.0126076	\\
0.291891	0.0126006	\\
0.292729	0.0125949	\\
0.293549	0.0125891	\\
0.294692	0.0125804	\\
0.29527	0.0125759	\\
0.296798	0.0125631	\\
0.298342	0.0125493	\\
0.299756	0.0125358	\\
0.301579	0.0125173	\\
0.302548	0.0125069	\\
0.304394	0.0124862	\\
0.306557	0.0124604	\\
0.308453	0.0124364	\\
0.310287	0.0124119	\\
0.311845	0.0123903	\\
0.313014	0.0123736	\\
0.314818	0.0123469	\\
0.316266	0.0123247	\\
0.317626	0.0123033	\\
0.319309	0.012276	\\
0.320928	0.012249	\\
0.322521	0.0122217	\\
0.323438	0.0122056	\\
0.324975	0.0121782	\\
0.32696	0.0121418	\\
0.328393	0.0121149	\\
0.329228	0.012099	\\
0.330537	0.0120737	\\
0.331616	0.0120525	\\
0.332773	0.0120294	\\
0.334735	0.0119896	\\
0.335838	0.0119669	\\
0.336532	0.0119524	\\
0.338708	0.0119063	\\
0.339764	0.0118836	\\
0.340906	0.0118587	\\
0.342263	0.0118288	\\
0.344496	0.0117787	\\
0.346563	0.0117315	\\
0.348333	0.0116904	\\
0.349845	0.0116549	\\
0.352231	0.0115979	\\
0.353715	0.0115619	\\
0.355412	0.0115203	\\
0.35655	0.0114922	\\
0.358077	0.0114541	\\
0.360099	0.0114031	\\
0.361818	0.0113592	\\
0.364139	0.0112992	\\
0.36546	0.0112648	\\
0.367669	0.0112066	\\
0.369975	0.0111452	\\
0.371557	0.0111026	\\
0.372694	0.0110719	\\
0.373774	0.0110425	\\
0.375262	0.0110018	\\
0.377306	0.0109455	\\
0.380459	0.0108577	\\
0.382933	0.0107881	\\
0.384678	0.0107387	\\
0.385902	0.0107038	\\
0.388727	0.0106228	\\
0.391005	0.010557	\\
0.393414	0.010487	\\
0.395144	0.0104363	\\
0.396611	0.0103933	\\
0.398773	0.0103295	\\
0.400663	0.0102735	\\
0.403071	0.0102018	\\
0.404874	0.0101479	\\
0.406304	0.010105	\\
0.407502	0.010069	\\
0.408969	0.0100248	\\
0.410986	0.00996386	\\
0.413545	0.00988624	\\
0.415279	0.00983349	\\
0.418128	0.00974655	\\
0.420064	0.00968728	\\
0.421823	0.00963331	\\
0.424593	0.00954818	\\
0.426398	0.00949257	\\
0.428145	0.00943866	\\
0.430117	0.00937775	\\
0.431719	0.00932818	\\
0.434004	0.00925741	\\
0.435639	0.00920671	\\
0.438002	0.0091334	\\
0.44003	0.00907043	\\
0.441192	0.00903432	\\
0.444417	0.00893408	\\
0.446955	0.00885513	\\
0.44909	0.00878872	\\
0.451835	0.00870335	\\
0.453309	0.00865753	\\
0.454813	0.00861074	\\
0.457958	0.00851302	\\
0.459862	0.00845389	\\
0.462247	0.00837987	\\
0.464619	0.00830635	\\
0.465952	0.00826507	\\
0.467532	0.00821616	\\
0.468681	0.00818063	\\
0.471989	0.00807846	\\
0.47471	0.00799458	\\
0.476291	0.00794591	\\
0.477826	0.00789874	\\
0.479499	0.00784736	\\
0.482866	0.00774422	\\
0.485005	0.00767885	\\
0.486881	0.00762163	\\
0.489435	0.00754394	\\
0.491926	0.00746835	\\
0.495021	0.0073747	\\
0.498525	0.00726913	\\
0.50063	0.00720592	\\
0.504166	0.00710013	\\
0.505665	0.00705545	\\
0.507841	0.00699074	\\
0.510741	0.00690481	\\
0.512662	0.0068481	\\
0.514519	0.00679344	\\
0.517737	0.00669908	\\
0.520304	0.00662416	\\
0.522653	0.00655589	\\
0.525222	0.00648152	\\
0.527856	0.00640564	\\
0.529053	0.00637128	\\
0.530345	0.00633426	\\
0.532671	0.00626783	\\
0.537014	0.00614458	\\
0.539905	0.0060631	\\
0.542539	0.00598927	\\
0.545665	0.00590217	\\
0.548282	0.00582967	\\
0.55136	0.00574491	\\
0.553736	0.00567987	\\
0.557423	0.00557959	\\
0.559872	0.00551346	\\
0.563057	0.00542797	\\
0.566709	0.00533071	\\
0.568692	0.00527825	\\
0.570944	0.00521898	\\
0.572341	0.00518236	\\
0.576713	0.00506861	\\
0.578029	0.0050346	\\
0.58042	0.0049731	\\
0.583278	0.00490011	\\
0.58787	0.00478389	\\
0.591899	0.00468311	\\
0.596812	0.00456164	\\
0.598832	0.00451219	\\
0.602077	0.00443331	\\
0.603788	0.004392	\\
0.605351	0.00435443	\\
0.609567	0.00425394	\\
0.612397	0.00418719	\\
0.613894	0.0041521	\\
0.617595	0.00406599	\\
0.618871	0.00403653	\\
0.621349	0.00397963	\\
0.623221	0.00393695	\\
0.625753	0.00387958	\\
0.630458	0.00377417	\\
0.631941	0.00374127	\\
0.636069	0.00365049	\\
0.638582	0.00359582	\\
0.642931	0.00350225	\\
0.646184	0.00343313	\\
0.649208	0.00336955	\\
0.651934	0.00331278	\\
0.655829	0.00323258	\\
0.659188	0.00316428	\\
0.661905	0.00310962	\\
0.664688	0.00305418	\\
0.668699	0.00297523	\\
0.673064	0.0028906	\\
0.675007	0.00285336	\\
0.677923	0.00279798	\\
0.680482	0.00274987	\\
0.683106	0.00270101	\\
0.686506	0.00263843	\\
0.68944	0.00258507	\\
0.695155	0.00248289	\\
0.699714	0.00240302	\\
0.701674	0.00236912	\\
0.705204	0.00230875	\\
0.708283	0.00225681	\\
0.710861	0.00221383	\\
0.713441	0.00217127	\\
0.716795	0.00211662	\\
0.718886	0.00208295	\\
0.722192	0.00203033	\\
0.725998	0.00197068	\\
0.730187	0.00190619	\\
0.73199	0.00187879	\\
0.736732	0.0018078	\\
0.741189	0.00174246	\\
0.743401	0.00171052	\\
0.74612	0.00167173	\\
0.750475	0.00161063	\\
0.753352	0.00157097	\\
0.755676	0.00153932	\\
0.757967	0.00150849	\\
0.762502	0.00144846	\\
0.766831	0.00139244	\\
0.771437	0.00133417	\\
0.77395	0.00130296	\\
0.776972	0.00126598	\\
0.781398	0.00121288	\\
0.784609	0.00117516	\\
0.787748	0.0011389	\\
0.791038	0.0011016	\\
0.794143	0.00106701	\\
0.79783	0.00102674	\\
0.802238	0.000979711	\\
0.806834	0.00093198	\\
0.809551	0.000904385	\\
0.812533	0.00087461	\\
0.81617	0.000839054	\\
0.820285	0.000799793	\\
0.824309	0.000762404	\\
0.826707	0.00074059	\\
0.833319	0.000682229	\\
0.835586	0.000662824	\\
0.838401	0.000639155	\\
0.842981	0.000601644	\\
0.847488	0.000565935	\\
0.851709	0.000533569	\\
0.855601	0.000504637	\\
0.860051	0.000472638	\\
0.864038	0.000444922	\\
0.867779	0.000419737	\\
0.870037	0.000404921	\\
0.87366	0.000381747	\\
0.880088	0.000342429	\\
0.884857	0.000314738	\\
0.889037	0.000291492	\\
0.893547	0.000267476	\\
0.896011	0.000254819	\\
0.900526	0.000232471	\\
0.903659	0.000217603	\\
0.907433	0.00020038	\\
0.912219	0.000179623	\\
0.913714	0.00017338	\\
0.915829	0.000164753	\\
0.918211	0.00015531	\\
0.922562	0.000138819	\\
0.926574	0.000124472	\\
0.93001	0.000112835	\\
0.939074	8.49833e-05	\\
0.94474	6.96453e-05	\\
0.947393	6.30058e-05	\\
0.951158	5.41708e-05	\\
0.957049	4.17261e-05	\\
0.963964	2.92352e-05	\\
0.967383	2.38946e-05	\\
0.969213	2.12623e-05	\\
0.974442	1.46018e-05	\\
0.979112	9.72228e-06	\\
0.985362	4.7544e-06	\\
0.987692	3.35565e-06	\\
0.991722	1.51406e-06	\\
0.996869	2.15826e-07	\\
1.00069	1.05936e-08	\\
1.00686	1.02947e-06	\\
1.0114	2.83242e-06	\\
1.01564	5.31994e-06	\\
1.01883	7.68918e-06	\\
1.02234	1.0796e-05	\\
1.02492	1.34103e-05	\\
1.02894	1.80404e-05	\\
1.03158	2.14432e-05	\\
1.0349	2.61294e-05	\\
1.04362	4.05681e-05	\\
1.04669	4.63876e-05	\\
1.05053	5.41956e-05	\\
1.05595	6.61954e-05	\\
1.06253	8.23197e-05	\\
1.06742	9.53786e-05	\\
1.07338	0.00011253	\\
1.07781	0.000126123	\\
1.08253	0.000141446	\\
1.08601	0.000153263	\\
1.09	0.000167371	\\
1.09294	0.000178093	\\
1.09738	0.000194926	\\
1.102	0.000213207	\\
1.10895	0.00024209	\\
1.11633	0.00027463	\\
1.12186	0.000300203	\\
1.12619	0.000320996	\\
1.1329	0.000354383	\\
1.13615	0.000371115	\\
1.14014	0.00039214	\\
1.14299	0.000407448	\\
1.14711	0.000430048	\\
1.15176	0.000456263	\\
1.15845	0.000495071	\\
1.16166	0.000514228	\\
1.16882	0.000558049	\\
1.1752	0.000598478	\\
1.17957	0.000626807	\\
1.18216	0.000643922	\\
1.19114	0.00070469	\\
1.19497	0.000731312	\\
1.20001	0.000766956	\\
1.20526	0.000804892	\\
1.21056	0.000843967	\\
1.21672	0.000890372	\\
1.22073	0.000921088	\\
1.23028	0.000996106	\\
1.23784	0.00105716	\\
1.24459	0.00111298	\\
1.24788	0.00114056	\\
1.25128	0.00116937	\\
1.25545	0.00120519	\\
1.25793	0.00122665	\\
1.26242	0.00126585	\\
1.26925	0.0013265	\\
1.27407	0.00137004	\\
1.27764	0.00140261	\\
1.28445	0.00146554	\\
1.28897	0.00150795	\\
1.29525	0.0015676	\\
1.3049	0.00166103	\\
1.30871	0.00169852	\\
1.31864	0.00179757	\\
1.32815	0.00189453	\\
1.34101	0.0020285	\\
1.3439	0.00205908	\\
1.34954	0.00211928	\\
1.35348	0.00216163	\\
1.35862	0.00221745	\\
1.36647	0.00230366	\\
1.37219	0.00236721	\\
1.3777	0.00242899	\\
1.38335	0.00249286	\\
1.38883	0.00255544	\\
1.39808	0.00266226	\\
1.40462	0.00273874	\\
1.40889	0.00278907	\\
1.41678	0.00288279	\\
1.42489	0.0029803	\\
1.4286	0.00302527	\\
1.43541	0.00310832	\\
1.44356	0.00320873	\\
1.447	0.00325146	\\
1.45273	0.00332298	\\
1.45783	0.00338703	\\
1.46261	0.00344746	\\
1.46644	0.00349607	\\
1.47306	0.00358068	\\
1.48112	0.00368457	\\
1.48631	0.00375191	\\
1.49152	0.00381999	\\
1.498	0.00390504	\\
1.50889	0.0040493	\\
1.51635	0.00414914	\\
1.52411	0.00425364	\\
1.53298	0.00437402	\\
1.53867	0.0044517	\\
1.54339	0.00451648	\\
1.55362	0.0046579	\\
1.56628	0.00483436	\\
1.57436	0.00494803	\\
1.58248	0.00506296	\\
1.58694	0.00512642	\\
1.59426	0.0052309	\\
1.6016	0.00533624	\\
1.60963	0.00545223	\\
1.61845	0.00558027	\\
1.62336	0.00565187	\\
1.63086	0.0057617	\\
1.63692	0.00585082	\\
1.64619	0.00598778	\\
1.65239	0.00607976	\\
1.65937	0.00618379	\\
1.665	0.00626798	\\
1.67189	0.00637139	\\
1.67556	0.0064266	\\
1.68271	0.00653463	\\
1.68987	0.00664304	\\
1.69877	0.00677849	\\
1.70866	0.00692963	\\
1.71442	0.00701796	\\
1.72011	0.00710557	\\
1.72704	0.00721242	\\
1.73769	0.0073773	\\
1.74769	0.00753295	\\
1.75454	0.00763984	\\
1.76234	0.00776196	\\
1.77118	0.00790073	\\
1.78279	0.00808391	\\
1.79008	0.00819923	\\
1.80116	0.00837496	\\
1.80927	0.00850418	\\
1.81677	0.00862393	\\
1.8269	0.00878603	\\
1.83609	0.00893349	\\
1.84246	0.00903603	\\
1.84909	0.00914284	\\
1.86067	0.00932995	\\
1.86797	0.0094483	\\
1.87509	0.00956385	\\
1.88462	0.00971875	\\
1.89612	0.00990628	\\
1.90382	0.0100319	\\
1.91835	0.0102699	\\
1.92937	0.0104508	\\
1.94517	0.0107109	\\
1.95358	0.0108497	\\
1.96322	0.0110089	\\
1.97067	0.0111322	\\
1.97847	0.0112613	\\
1.99118	0.0114721	\\
1.99933	0.0116076	\\
2.00577	0.0117146	\\
2.01732	0.0119068	\\
2.02421	0.0120216	\\
2.03162	0.0121452	\\
2.0416	0.0123119	\\
2.05159	0.0124787	\\
2.06066	0.0126304	\\
2.07203	0.0128208	\\
2.07753	0.0129129	\\
2.08707	0.0130727	\\
2.09452	0.0131978	\\
2.10783	0.013421	\\
2.1198	0.013622	\\
2.12924	0.0137806	\\
2.14501	0.0140457	\\
2.15928	0.0142857	\\
2.17265	0.0145106	\\
2.19081	0.0148163	\\
2.2014	0.0149946	\\
2.21006	0.0151403	\\
2.21576	0.0152363	\\
2.22677	0.0154217	\\
2.23651	0.0155857	\\
2.25202	0.0158466	\\
2.26277	0.0160275	\\
2.27758	0.0162766	\\
2.2885	0.0164603	\\
2.2988	0.0166335	\\
2.30861	0.0167981	\\
2.31522	0.0169093	\\
2.32688	0.017105	\\
2.34351	0.017384	\\
2.35324	0.0175471	\\
2.36576	0.017757	\\
2.36978	0.0178243	\\
2.3798	0.0179921	\\
2.39165	0.0181903	\\
2.40418	0.0183997	\\
2.41219	0.0185334	\\
2.42786	0.0187949	\\
2.43743	0.0189543	\\
2.44472	0.0190758	\\
2.45551	0.0192553	\\
2.46508	0.0194145	\\
2.47591	0.0195944	\\
2.49047	0.019836	\\
2.50494	0.0200757	\\
2.51629	0.0202636	\\
2.53182	0.0205201	\\
2.54063	0.0206655	\\
2.55441	0.0208926	\\
2.56632	0.0210885	\\
2.57583	0.0212449	\\
2.59765	0.0216027	\\
2.61636	0.0219089	\\
2.62939	0.0221215	\\
2.64229	0.0223318	\\
2.65063	0.0224676	\\
2.66488	0.0226991	\\
2.67204	0.0228153	\\
2.68853	0.0230824	\\
2.70159	0.0232937	\\
2.71485	0.0235075	\\
2.72522	0.0236747	\\
2.73163	0.0237777	\\
2.74114	0.0239305	\\
2.75287	0.0241187	\\
2.75762	0.0241947	\\
2.76946	0.0243842	\\
2.79324	0.0247636	\\
2.80245	0.0249102	\\
2.82032	0.0251939	\\
2.83198	0.0253787	\\
2.84039	0.0255117	\\
2.85568	0.025753	\\
2.87357	0.0260345	\\
2.88426	0.0262023	\\
2.90311	0.0264975	\\
2.91433	0.0266728	\\
2.91981	0.0267582	\\
2.93285	0.0269612	\\
2.94845	0.0272035	\\
2.95951	0.0273748	\\
2.97174	0.0275639	\\
2.98385	0.0277508	\\
3.00019	0.0280022	\\
3.00996	0.0281521	\\
3.02438	0.0283729	\\
3.03676	0.0285621	\\
3.05065	0.0287737	\\
3.06015	0.0289182	\\
3.07312	0.0291149	\\
3.09063	0.0293798	\\
3.10426	0.0295854	\\
3.11856	0.0298003	\\
3.13097	0.0299864	\\
3.14182	0.0301489	\\
3.15581	0.0303577	\\
3.17855	0.0306959	\\
3.1944	0.0309308	\\
3.20509	0.0310888	\\
3.21843	0.0312854	\\
3.23677	0.0315549	\\
3.25536	0.031827	\\
3.27116	0.0320576	\\
3.28145	0.0322073	\\
3.28776	0.0322989	\\
3.29755	0.0324408	\\
3.31668	0.0327173	\\
3.33641	0.0330014	\\
3.35121	0.0332137	\\
3.36002	0.0333397	\\
3.37035	0.0334872	\\
3.38202	0.0336535	\\
3.40333	0.033956	\\
3.41024	0.0340538	\\
3.42871	0.0343144	\\
3.4392	0.0344621	\\
3.45	0.0346138	\\
3.47698	0.0349909	\\
3.48591	0.0351153	\\
3.49916	0.0352994	\\
3.51073	0.0354598	\\
3.53144	0.0357458	\\
3.54278	0.0359017	\\
3.55771	0.0361066	\\
3.56828	0.0362512	\\
3.59019	0.03655	\\
3.61177	0.0368427	\\
3.62231	0.0369853	\\
3.63224	0.0371192	\\
3.64821	0.0373339	\\
3.67865	0.0377413	\\
3.6901	0.0378938	\\
3.70816	0.0381336	\\
3.72825	0.0383991	\\
3.74526	0.0386231	\\
3.75968	0.0388122	\\
3.784	0.0391299	\\
3.79747	0.039305	\\
3.81243	0.039499	\\
3.82783	0.039698	\\
3.84763	0.0399528	\\
3.85952	0.0401053	\\
3.87565	0.0403115	\\
3.90046	0.0406271	\\
3.91193	0.0407725	\\
3.93007	0.0410015	\\
3.9543	0.041306	\\
3.97897	0.0416143	\\
3.99471	0.0418102	\\
4.01354	0.0420434	\\
4.02971	0.0422431	\\
4.0409	0.0423808	\\
4.05891	0.0426017	\\
4.07359	0.0427809	\\
4.08099	0.0428711	\\
4.10528	0.0431662	\\
4.1273	0.0434321	\\
4.14039	0.0435896	\\
4.1655	0.0438903	\\
4.18578	0.044132	\\
4.21003	0.0444194	\\
4.23799	0.044749	\\
4.25882	0.0449931	\\
4.27918	0.0452306	\\
4.2957	0.0454225	\\
4.31762	0.045676	\\
4.33062	0.0458257	\\
4.35793	0.0461388	\\
4.37829	0.046371	\\
4.4087	0.0467158	\\
4.42985	0.0469542	\\
4.44854	0.0471639	\\
4.46135	0.0473071	\\
4.48603	0.0475818	\\
4.51	0.0478471	\\
4.53141	0.0480829	\\
4.55237	0.0483126	\\
4.5749	0.0485584	\\
4.59724	0.0488007	\\
4.60872	0.0489248	\\
4.62886	0.0491417	\\
4.64609	0.0493265	\\
4.70087	0.0499091	\\
4.72547	0.0501684	\\
4.74041	0.0503252	\\
4.75	0.0504256	\\
4.7787	0.0507246	\\
4.78755	0.0508165	\\
4.80747	0.0510225	\\
4.83792	0.0513356	\\
4.85137	0.0514732	\\
4.87628	0.0517271	\\
4.89058	0.0518721	\\
4.901	0.0519775	\\
4.91963	0.0521653	\\
4.94708	0.0524407	\\
4.96503	0.0526198	\\
4.98023	0.0527709	\\
5.00062	0.0529729	\\
5.03209	0.0532828	\\
5.04733	0.053432	\\
5.05957	0.0535515	\\
5.089	0.0538376	\\
5.11273	0.0540668	\\
5.139	0.0543193	\\
5.17036	0.0546186	\\
5.20238	0.0549222	\\
5.22068	0.0550947	\\
5.24171	0.0552921	\\
5.27082	0.0555639	\\
5.29698	0.0558066	\\
5.3122	0.0559471	\\
5.33363	0.0561443	\\
5.35787	0.0563662	\\
5.38709	0.0566321	\\
5.41346	0.0568706	\\
5.4475	0.0571764	\\
5.48522	0.0575127	\\
5.50637	0.0577001	\\
5.5248	0.0578626	\\
5.55313	0.0581112	\\
5.57634	0.0583138	\\
5.60366	0.0585509	\\
5.62191	0.0587085	\\
5.65958	0.0590319	\\
5.69471	0.0593312	\\
5.71707	0.0595205	\\
5.73437	0.0596663	\\
5.75999	0.0598813	\\
5.78241	0.0600684	\\
5.80886	0.0602882	\\
5.82328	0.0604073	\\
5.85852	0.0606972	\\
5.88786	0.060937	\\
5.917	0.0611735	\\
5.93246	0.0612985	\\
5.96588	0.0615671	\\
5.99341	0.0617871	\\
6.011	0.061927	\\
6.03999	0.0621564	\\
6.06491	0.0623524	\\
6.09937	0.0626218	\\
6.12948	0.0628557	\\
6.16636	0.0631401	\\
6.19125	0.0633309	\\
6.21746	0.0635307	\\
6.24477	0.0637378	\\
6.26611	0.0638988	\\
6.28719	0.0640571	\\
6.3129	0.0642492	\\
6.34627	0.0644972	\\
6.37139	0.0646827	\\
6.39363	0.0648462	\\
6.41025	0.0649679	\\
6.43865	0.0651749	\\
6.47916	0.065468	\\
6.5102	0.065691	\\
6.52663	0.0658085	\\
6.57432	0.0661473	\\
6.61072	0.0664038	\\
6.63516	0.0665749	\\
6.68209	0.0669012	\\
6.70497	0.0670592	\\
6.7332	0.0672531	\\
6.76038	0.0674388	\\
6.79327	0.0676622	\\
6.82495	0.067876	\\
6.86837	0.0681668	\\
6.88989	0.0683101	\\
6.91276	0.0684616	\\
6.93727	0.0686233	\\
6.99222	0.068983	\\
7.0143	0.0691264	\\
7.03642	0.0692695	\\
7.07477	0.0695162	\\
7.09945	0.069674	\\
7.12214	0.0698184	\\
7.15202	0.0700075	\\
7.18136	0.0701922	\\
7.22839	0.0704862	\\
7.29291	0.0708852	\\
7.32937	0.0711085	\\
7.34602	0.07121	\\
7.39887	0.07153	\\
7.42534	0.071689	\\
7.45516	0.0718673	\\
7.48223	0.0720283	\\
7.51391	0.0722156	\\
7.56059	0.0724897	\\
7.6048	0.072747	\\
7.62181	0.0728455	\\
7.64773	0.0729949	\\
7.69234	0.0732505	\\
7.73359	0.0734849	\\
7.77596	0.0737239	\\
7.82277	0.0739858	\\
7.86155	0.074201	\\
7.89303	0.0743746	\\
7.93264	0.0745917	\\
7.97912	0.0748445	\\
8.00993	0.0750109	\\
8.05018	0.0752269	\\
8.09962	0.0754901	\\
8.13586	0.0756815	\\
8.17892	0.0759075	\\
8.2152	0.0760965	\\
8.24147	0.0762325	\\
8.27699	0.0764156	\\
8.31588	0.0766147	\\
8.36978	0.0768884	\\
8.39427	0.0770119	\\
8.43448	0.0772136	\\
8.47906	0.0774355	\\
8.50451	0.0775614	\\
8.52412	0.0776581	\\
8.57725	0.0779183	\\
8.60789	0.0780674	\\
8.6428	0.0782362	\\
8.70304	0.0785252	\\
8.76292	0.0788095	\\
8.79102	0.078942	\\
8.8193	0.0790746	\\
8.86518	0.0792885	\\
8.91599	0.0795235	\\
8.96533	0.0797498	\\
8.99896	0.0799029	\\
9.03091	0.0800476	\\
9.06498	0.0802011	\\
9.11212	0.080412	\\
9.15748	0.0806135	\\
9.20151	0.0808076	\\
9.2463	0.0810036	\\
9.30386	0.0812534	\\
9.35036	0.0814535	\\
9.41429	0.0817262	\\
9.45702	0.0819069	\\
9.49771	0.0820779	\\
9.55248	0.0823062	\\
9.61597	0.0825684	\\
9.68052	0.0828323	\\
9.70768	0.0829425	\\
9.7569	0.0831411	\\
9.79248	0.0832837	\\
9.83031	0.0834344	\\
9.87647	0.0836172	\\
9.90852	0.0837433	\\
9.95018	0.0839062	\\
};
\end{axis}
\end{tikzpicture}%

        \end{subfigure}

        \begin{subfigure}[b]{0.45\textwidth}
            % % This file was created by matlab2tikz v0.4.6 running on MATLAB 8.1.
% Copyright (c) 2008--2014, Nico Schlömer <nico.schloemer@gmail.com>
% All rights reserved.
% Minimal pgfplots version: 1.3
%
% The latest updates can be retrieved from
%   http://www.mathworks.com/matlabcentral/fileexchange/22022-matlab2tikz
% where you can also make suggestions and rate matlab2tikz.
%
\begin{tikzpicture}

\begin{axis}[%
width=\figurewidth,
height=\figureheight,
scale only axis,
xmin=0,
xmax=10,
ymode=log,
ymin=1e-12,
ymax=1e-06,
yminorticks=false
]
\addplot [color=red,solid,forget plot]
  table[row sep=crcr]{
0.100016	2.31354e-15	\\
0.100732	1.71568e-10	\\
0.101143	4.18882e-10	\\
0.101868	1.10442e-09	\\
0.102291	1.64338e-09	\\
0.102765	2.36239e-09	\\
0.103121	2.97878e-09	\\
0.103483	3.66815e-09	\\
0.103876	4.4896e-09	\\
0.104476	5.8754e-09	\\
0.104665	6.34396e-09	\\
0.105362	8.19852e-09	\\
0.105744	9.297e-09	\\
0.106062	1.02512e-08	\\
0.106298	1.09834e-08	\\
0.106648	1.21034e-08	\\
0.10715	1.37779e-08	\\
0.107607	1.53734e-08	\\
0.108043	1.69512e-08	\\
0.108317	1.79688e-08	\\
0.108894	2.0179e-08	\\
0.109224	2.14829e-08	\\
0.109604	2.30142e-08	\\
0.110172	2.53628e-08	\\
0.110505	2.67686e-08	\\
0.110927	2.85869e-08	\\
0.111227	2.98967e-08	\\
0.111839	3.26245e-08	\\
0.112341	3.49062e-08	\\
0.112781	3.69378e-08	\\
0.113469	4.0167e-08	\\
0.113904	4.22398e-08	\\
0.114661	4.58983e-08	\\
0.115014	4.76216e-08	\\
0.115804	5.15268e-08	\\
0.11616	5.33011e-08	\\
0.11676	5.63135e-08	\\
0.1172	5.85383e-08	\\
0.117616	6.0649e-08	\\
0.118117	6.32035e-08	\\
0.118705	6.62139e-08	\\
0.119312	6.93348e-08	\\
0.119771	7.17004e-08	\\
0.120376	7.48201e-08	\\
0.121135	7.87467e-08	\\
0.121529	8.07828e-08	\\
0.122052	8.34894e-08	\\
0.12263	8.64759e-08	\\
0.123087	8.88308e-08	\\
0.123626	9.16108e-08	\\
0.124085	9.39662e-08	\\
0.124676	9.70001e-08	\\
0.125112	9.92283e-08	\\
0.125592	1.01675e-07	\\
0.126054	1.0402e-07	\\
0.126291	1.05224e-07	\\
0.126904	1.08318e-07	\\
0.127492	1.1127e-07	\\
0.127792	1.12768e-07	\\
0.128534	1.16457e-07	\\
0.129167	1.19582e-07	\\
0.129774	1.22556e-07	\\
0.130294	1.25083e-07	\\
0.130818	1.27615e-07	\\
0.131487	1.30822e-07	\\
0.132056	1.33524e-07	\\
0.132644	1.36294e-07	\\
0.133001	1.37962e-07	\\
0.133666	1.4105e-07	\\
0.134361	1.44236e-07	\\
0.13507	1.47451e-07	\\
0.135588	1.49778e-07	\\
0.135991	1.51571e-07	\\
0.13657	1.54124e-07	\\
0.13714	1.5661e-07	\\
0.13769	1.58985e-07	\\
0.138229	1.61289e-07	\\
0.138825	1.63807e-07	\\
0.139339	1.65958e-07	\\
0.140146	1.69281e-07	\\
0.140895	1.72321e-07	\\
0.141373	1.74233e-07	\\
0.141897	1.7631e-07	\\
0.142535	1.78801e-07	\\
0.143589	1.82845e-07	\\
0.144197	1.8513e-07	\\
0.144743	1.87158e-07	\\
0.145248	1.89007e-07	\\
0.145703	1.90659e-07	\\
0.146385	1.93092e-07	\\
0.147035	1.9538e-07	\\
0.147801	1.98022e-07	\\
0.148257	1.99573e-07	\\
0.149005	2.02074e-07	\\
0.149751	2.04522e-07	\\
0.150789	2.07849e-07	\\
0.151369	2.09666e-07	\\
0.15236	2.12705e-07	\\
0.153206	2.15234e-07	\\
0.153923	2.17329e-07	\\
0.154616	2.19314e-07	\\
0.155702	2.22344e-07	\\
0.156276	2.23907e-07	\\
0.157109	2.26125e-07	\\
0.157913	2.28214e-07	\\
0.158792	2.30437e-07	\\
0.159332	2.31775e-07	\\
0.160076	2.33576e-07	\\
0.160613	2.34852e-07	\\
0.161543	2.37005e-07	\\
0.161952	2.37932e-07	\\
0.162943	2.40126e-07	\\
0.163795	2.41952e-07	\\
0.164577	2.4358e-07	\\
0.165563	2.45572e-07	\\
0.166526	2.47449e-07	\\
0.167275	2.48862e-07	\\
0.167903	2.50017e-07	\\
0.168356	2.50835e-07	\\
0.169313	2.52514e-07	\\
0.170372	2.54301e-07	\\
0.171347	2.5588e-07	\\
0.172157	2.57145e-07	\\
0.172836	2.58172e-07	\\
0.173425	2.5904e-07	\\
0.173801	2.59582e-07	\\
0.174706	2.60853e-07	\\
0.175346	2.61722e-07	\\
0.176423	2.63128e-07	\\
0.177117	2.63997e-07	\\
0.178062	2.65136e-07	\\
0.178898	2.66101e-07	\\
0.180102	2.67422e-07	\\
0.180862	2.68215e-07	\\
0.182027	2.69369e-07	\\
0.184089	2.71239e-07	\\
0.18487	2.7189e-07	\\
0.185944	2.72737e-07	\\
0.18661	2.73234e-07	\\
0.187504	2.73867e-07	\\
0.188082	2.74257e-07	\\
0.188911	2.74789e-07	\\
0.189884	2.75374e-07	\\
0.190693	2.75827e-07	\\
0.191648	2.76326e-07	\\
0.19297	2.76952e-07	\\
0.194005	2.77392e-07	\\
0.195319	2.77887e-07	\\
0.19641	2.78247e-07	\\
0.197237	2.78489e-07	\\
0.198188	2.78736e-07	\\
0.198714	2.78858e-07	\\
0.199929	2.79102e-07	\\
0.200903	2.7926e-07	\\
0.201795	2.79375e-07	\\
0.202715	2.79466e-07	\\
0.203536	2.79524e-07	\\
0.204572	2.79565e-07	\\
0.205426	2.79574e-07	\\
0.207216	2.7952e-07	\\
0.207983	2.79467e-07	\\
0.208602	2.79412e-07	\\
0.209736	2.79283e-07	\\
0.210653	2.79153e-07	\\
0.211348	2.79039e-07	\\
0.212512	2.78819e-07	\\
0.213482	2.78609e-07	\\
0.214593	2.78339e-07	\\
0.215334	2.78142e-07	\\
0.216055	2.77938e-07	\\
0.217124	2.77612e-07	\\
0.21783	2.77383e-07	\\
0.219062	2.76956e-07	\\
0.219841	2.76668e-07	\\
0.220808	2.76294e-07	\\
0.222028	2.75793e-07	\\
0.222801	2.7546e-07	\\
0.223564	2.7512e-07	\\
0.224473	2.74701e-07	\\
0.225435	2.7424e-07	\\
0.226929	2.7349e-07	\\
0.228105	2.72873e-07	\\
0.229383	2.72174e-07	\\
0.230274	2.71672e-07	\\
0.231502	2.70958e-07	\\
0.23291	2.70112e-07	\\
0.233592	2.69691e-07	\\
0.234564	2.69079e-07	\\
0.235642	2.68386e-07	\\
0.237129	2.67403e-07	\\
0.238486	2.66481e-07	\\
0.239473	2.65796e-07	\\
0.240467	2.65094e-07	\\
0.241136	2.64615e-07	\\
0.242537	2.63595e-07	\\
0.244035	2.62481e-07	\\
0.245076	2.61692e-07	\\
0.246009	2.60976e-07	\\
0.247	2.60206e-07	\\
0.247885	2.5951e-07	\\
0.249225	2.58444e-07	\\
0.250428	2.57472e-07	\\
0.251459	2.5663e-07	\\
0.252329	2.55913e-07	\\
0.253517	2.54923e-07	\\
0.254847	2.53803e-07	\\
0.255904	2.52904e-07	\\
0.257098	2.51878e-07	\\
0.258282	2.50852e-07	\\
0.259441	2.49839e-07	\\
0.260833	2.48612e-07	\\
0.262131	2.47457e-07	\\
0.262987	2.46691e-07	\\
0.263852	2.45913e-07	\\
0.264951	2.44918e-07	\\
0.266916	2.43127e-07	\\
0.268033	2.421e-07	\\
0.269649	2.40606e-07	\\
0.271025	2.39326e-07	\\
0.272281	2.38152e-07	\\
0.274193	2.36355e-07	\\
0.275028	2.35567e-07	\\
0.276161	2.34494e-07	\\
0.27739	2.33326e-07	\\
0.278326	2.32434e-07	\\
0.279306	2.31498e-07	\\
0.280599	2.3026e-07	\\
0.281796	2.29111e-07	\\
0.282404	2.28527e-07	\\
0.282916	2.28033e-07	\\
0.284345	2.26656e-07	\\
0.286539	2.24535e-07	\\
0.288465	2.22668e-07	\\
0.289577	2.2159e-07	\\
0.290789	2.20412e-07	\\
0.291891	2.19341e-07	\\
0.292729	2.18526e-07	\\
0.293549	2.17728e-07	\\
0.294692	2.16616e-07	\\
0.29527	2.16053e-07	\\
0.296798	2.14566e-07	\\
0.298342	2.13062e-07	\\
0.299756	2.11686e-07	\\
0.301579	2.09913e-07	\\
0.302548	2.08972e-07	\\
0.304394	2.07178e-07	\\
0.306557	2.05081e-07	\\
0.308453	2.03245e-07	\\
0.310287	2.01473e-07	\\
0.311845	1.99972e-07	\\
0.313014	1.98847e-07	\\
0.314818	1.97114e-07	\\
0.316266	1.95727e-07	\\
0.317626	1.94427e-07	\\
0.319309	1.92823e-07	\\
0.320928	1.91284e-07	\\
0.322521	1.89775e-07	\\
0.323438	1.88908e-07	\\
0.324975	1.87459e-07	\\
0.32696	1.85595e-07	\\
0.328393	1.84255e-07	\\
0.329228	1.83476e-07	\\
0.330537	1.82257e-07	\\
0.331616	1.81256e-07	\\
0.332773	1.80186e-07	\\
0.334735	1.78378e-07	\\
0.335838	1.77365e-07	\\
0.336532	1.76729e-07	\\
0.338708	1.74744e-07	\\
0.339764	1.73786e-07	\\
0.340906	1.72752e-07	\\
0.342263	1.71528e-07	\\
0.344496	1.69525e-07	\\
0.346563	1.67683e-07	\\
0.348333	1.66115e-07	\\
0.349845	1.64783e-07	\\
0.352231	1.62693e-07	\\
0.353715	1.61402e-07	\\
0.355412	1.59933e-07	\\
0.35655	1.58952e-07	\\
0.358077	1.57643e-07	\\
0.360099	1.55921e-07	\\
0.361818	1.54466e-07	\\
0.364139	1.52516e-07	\\
0.36546	1.51413e-07	\\
0.367669	1.49582e-07	\\
0.369975	1.47686e-07	\\
0.371557	1.46395e-07	\\
0.372694	1.45472e-07	\\
0.373774	1.44599e-07	\\
0.375262	1.43402e-07	\\
0.377306	1.41769e-07	\\
0.380459	1.39277e-07	\\
0.382933	1.37344e-07	\\
0.384678	1.35993e-07	\\
0.385902	1.3505e-07	\\
0.388727	1.32893e-07	\\
0.391005	1.31172e-07	\\
0.393414	1.29371e-07	\\
0.395144	1.28089e-07	\\
0.396611	1.2701e-07	\\
0.398773	1.25431e-07	\\
0.400663	1.24063e-07	\\
0.403071	1.22337e-07	\\
0.404874	1.21057e-07	\\
0.406304	1.20049e-07	\\
0.407502	1.19209e-07	\\
0.408969	1.18187e-07	\\
0.410986	1.16794e-07	\\
0.413545	1.15043e-07	\\
0.415279	1.13869e-07	\\
0.418128	1.11961e-07	\\
0.420064	1.10678e-07	\\
0.421823	1.09522e-07	\\
0.424593	1.07723e-07	\\
0.426398	1.06562e-07	\\
0.428145	1.05449e-07	\\
0.430117	1.04203e-07	\\
0.431719	1.032e-07	\\
0.434004	1.01782e-07	\\
0.435639	1.00777e-07	\\
0.438002	9.93393e-08	\\
0.44003	9.81185e-08	\\
0.441192	9.74243e-08	\\
0.444417	9.55187e-08	\\
0.446955	9.40399e-08	\\
0.44909	9.28104e-08	\\
0.451835	9.12493e-08	\\
0.453309	9.04199e-08	\\
0.454813	8.95793e-08	\\
0.457958	8.78432e-08	\\
0.459862	8.68053e-08	\\
0.462247	8.55194e-08	\\
0.464619	8.42564e-08	\\
0.465952	8.35533e-08	\\
0.467532	8.27259e-08	\\
0.468681	8.21284e-08	\\
0.471989	8.04282e-08	\\
0.47471	7.90513e-08	\\
0.476291	7.82602e-08	\\
0.477826	7.74985e-08	\\
0.479499	7.66749e-08	\\
0.482866	7.50395e-08	\\
0.485005	7.40154e-08	\\
0.486881	7.31267e-08	\\
0.489435	7.19313e-08	\\
0.491926	7.07808e-08	\\
0.495021	6.93719e-08	\\
0.498525	6.78053e-08	\\
0.50063	6.68781e-08	\\
0.504166	6.53441e-08	\\
0.505665	6.47026e-08	\\
0.507841	6.37805e-08	\\
0.510741	6.25685e-08	\\
0.512662	6.17761e-08	\\
0.514519	6.1018e-08	\\
0.517737	5.97222e-08	\\
0.520304	5.87048e-08	\\
0.522653	5.77866e-08	\\
0.525222	5.67955e-08	\\
0.527856	5.57941e-08	\\
0.529053	5.53439e-08	\\
0.530345	5.48611e-08	\\
0.532671	5.40007e-08	\\
0.537014	5.24238e-08	\\
0.539905	5.13951e-08	\\
0.542539	5.04721e-08	\\
0.545665	4.93947e-08	\\
0.548282	4.8507e-08	\\
0.55136	4.74795e-08	\\
0.553736	4.66988e-08	\\
0.557423	4.55075e-08	\\
0.559872	4.47303e-08	\\
0.563057	4.37353e-08	\\
0.566709	4.26164e-08	\\
0.568692	4.20187e-08	\\
0.570944	4.13482e-08	\\
0.572341	4.09365e-08	\\
0.576713	3.96698e-08	\\
0.578029	3.92946e-08	\\
0.58042	3.86204e-08	\\
0.583278	3.7827e-08	\\
0.58787	3.6579e-08	\\
0.591899	3.55118e-08	\\
0.596812	3.42438e-08	\\
0.598832	3.37333e-08	\\
0.602077	3.29257e-08	\\
0.603788	3.2506e-08	\\
0.605351	3.21262e-08	\\
0.609567	3.11197e-08	\\
0.612397	3.04583e-08	\\
0.613894	3.0113e-08	\\
0.617595	2.92722e-08	\\
0.618871	2.89868e-08	\\
0.621349	2.84385e-08	\\
0.623221	2.803e-08	\\
0.625753	2.74844e-08	\\
0.630458	2.64929e-08	\\
0.631941	2.61863e-08	\\
0.636069	2.53473e-08	\\
0.638582	2.48469e-08	\\
0.642931	2.3999e-08	\\
0.646184	2.33795e-08	\\
0.649208	2.28149e-08	\\
0.651934	2.23148e-08	\\
0.655829	2.16151e-08	\\
0.659188	2.10254e-08	\\
0.661905	2.05574e-08	\\
0.664688	2.00865e-08	\\
0.668699	1.94221e-08	\\
0.673064	1.87182e-08	\\
0.675007	1.84112e-08	\\
0.677923	1.79576e-08	\\
0.680482	1.75665e-08	\\
0.683106	1.71722e-08	\\
0.686506	1.66712e-08	\\
0.68944	1.62477e-08	\\
0.695155	1.54461e-08	\\
0.699714	1.4828e-08	\\
0.701674	1.4568e-08	\\
0.705204	1.41082e-08	\\
0.708283	1.37161e-08	\\
0.710861	1.3394e-08	\\
0.713441	1.30772e-08	\\
0.716795	1.26736e-08	\\
0.718886	1.24267e-08	\\
0.722192	1.20434e-08	\\
0.725998	1.1613e-08	\\
0.730187	1.11522e-08	\\
0.73199	1.0958e-08	\\
0.736732	1.04589e-08	\\
0.741189	1.00049e-08	\\
0.743401	9.78479e-09	\\
0.74612	9.5191e-09	\\
0.750475	9.1043e-09	\\
0.753352	8.83747e-09	\\
0.755676	8.62591e-09	\\
0.757967	8.42097e-09	\\
0.762502	8.0253e-09	\\
0.766831	7.65996e-09	\\
0.771437	7.28413e-09	\\
0.77395	7.08456e-09	\\
0.776972	6.84963e-09	\\
0.781398	6.51535e-09	\\
0.784609	6.28002e-09	\\
0.787748	6.05556e-09	\\
0.791038	5.82634e-09	\\
0.794143	5.61545e-09	\\
0.79783	5.37181e-09	\\
0.802238	5.09001e-09	\\
0.806834	4.80698e-09	\\
0.809551	4.64474e-09	\\
0.812533	4.47084e-09	\\
0.81617	4.26475e-09	\\
0.820285	4.03921e-09	\\
0.824309	3.82642e-09	\\
0.826707	3.70318e-09	\\
0.833319	3.3768e-09	\\
0.835586	3.26937e-09	\\
0.838401	3.13909e-09	\\
0.842981	2.93432e-09	\\
0.847488	2.74136e-09	\\
0.851709	2.56816e-09	\\
0.855601	2.41473e-09	\\
0.860051	2.24659e-09	\\
0.864038	2.10231e-09	\\
0.867779	1.97231e-09	\\
0.870037	1.89634e-09	\\
0.87366	1.77826e-09	\\
0.880088	1.5801e-09	\\
0.884857	1.44223e-09	\\
0.889037	1.3276e-09	\\
0.893547	1.21027e-09	\\
0.896011	1.1489e-09	\\
0.900526	1.04134e-09	\\
0.903659	9.70363e-10	\\
0.907433	8.88748e-10	\\
0.912219	7.91262e-10	\\
0.913714	7.62137e-10	\\
0.915829	7.22047e-10	\\
0.918211	6.78368e-10	\\
0.922562	6.02621e-10	\\
0.926574	5.37297e-10	\\
0.93001	4.84724e-10	\\
0.939074	3.60489e-10	\\
0.94474	2.93116e-10	\\
0.947393	2.64203e-10	\\
0.951158	2.2598e-10	\\
0.957049	1.72665e-10	\\
0.963964	1.19841e-10	\\
0.967383	9.74951e-11	\\
0.969213	8.65398e-11	\\
0.974442	5.90126e-11	\\
0.979112	3.90461e-11	\\
0.985362	1.89354e-11	\\
0.987692	1.33231e-11	\\
0.991722	5.97938e-12	\\
0.996869	8.46815e-13	\\
1.00069	4.15916e-14	\\
1.00686	4.01249e-12	\\
1.0114	1.10234e-11	\\
1.01564	2.06762e-11	\\
1.01883	2.9854e-11	\\
1.02234	4.18696e-11	\\
1.02492	5.19657e-11	\\
1.02894	6.98182e-11	\\
1.03158	8.29175e-11	\\
1.0349	1.00932e-10	\\
1.04362	1.56272e-10	\\
1.04669	1.78516e-10	\\
1.05053	2.0831e-10	\\
1.05595	2.53997e-10	\\
1.06253	3.15211e-10	\\
1.06742	3.64651e-10	\\
1.07338	4.29417e-10	\\
1.07781	4.80618e-10	\\
1.08253	5.38209e-10	\\
1.08601	5.82536e-10	\\
1.09	6.3536e-10	\\
1.09294	6.75438e-10	\\
1.09738	7.38252e-10	\\
1.102	8.06316e-10	\\
1.10895	9.13556e-10	\\
1.11633	1.03396e-09	\\
1.12186	1.12829e-09	\\
1.12619	1.20481e-09	\\
1.1329	1.32734e-09	\\
1.13615	1.3886e-09	\\
1.14014	1.46545e-09	\\
1.14299	1.5213e-09	\\
1.14711	1.60363e-09	\\
1.15176	1.69892e-09	\\
1.15845	1.8396e-09	\\
1.16166	1.90888e-09	\\
1.16882	2.06696e-09	\\
1.1752	2.21233e-09	\\
1.17957	2.31392e-09	\\
1.18216	2.37519e-09	\\
1.19114	2.59213e-09	\\
1.19497	2.68688e-09	\\
1.20001	2.81346e-09	\\
1.20526	2.94785e-09	\\
1.21056	3.08591e-09	\\
1.21672	3.24942e-09	\\
1.22073	3.35738e-09	\\
1.23028	3.62018e-09	\\
1.23784	3.83316e-09	\\
1.24459	4.02721e-09	\\
1.24788	4.12285e-09	\\
1.25128	4.22259e-09	\\
1.25545	4.34637e-09	\\
1.25793	4.4204e-09	\\
1.26242	4.55544e-09	\\
1.26925	4.76375e-09	\\
1.27407	4.91285e-09	\\
1.27764	5.02417e-09	\\
1.28445	5.23868e-09	\\
1.28897	5.38286e-09	\\
1.29525	5.58509e-09	\\
1.3049	5.90062e-09	\\
1.30871	6.02678e-09	\\
1.31864	6.35904e-09	\\
1.32815	6.6827e-09	\\
1.34101	7.12742e-09	\\
1.3439	7.22854e-09	\\
1.34954	7.42719e-09	\\
1.35348	7.56658e-09	\\
1.35862	7.74992e-09	\\
1.36647	8.03213e-09	\\
1.37219	8.23944e-09	\\
1.3777	8.44042e-09	\\
1.38335	8.64763e-09	\\
1.38883	8.85007e-09	\\
1.39808	9.19433e-09	\\
1.40462	9.43984e-09	\\
1.40889	9.60096e-09	\\
1.41678	9.90008e-09	\\
1.42489	1.021e-08	\\
1.4286	1.03525e-08	\\
1.43541	1.0615e-08	\\
1.44356	1.09311e-08	\\
1.447	1.10652e-08	\\
1.45273	1.12892e-08	\\
1.45783	1.14892e-08	\\
1.46261	1.16775e-08	\\
1.46644	1.18286e-08	\\
1.47306	1.20908e-08	\\
1.48112	1.24117e-08	\\
1.48631	1.26189e-08	\\
1.49152	1.28279e-08	\\
1.498	1.30881e-08	\\
1.50889	1.35275e-08	\\
1.51635	1.38302e-08	\\
1.52411	1.41458e-08	\\
1.53298	1.45077e-08	\\
1.53867	1.47403e-08	\\
1.54339	1.49338e-08	\\
1.55362	1.53545e-08	\\
1.56628	1.58763e-08	\\
1.57436	1.62106e-08	\\
1.58248	1.65471e-08	\\
1.58694	1.67323e-08	\\
1.59426	1.70362e-08	\\
1.6016	1.73415e-08	\\
1.60963	1.76762e-08	\\
1.61845	1.8044e-08	\\
1.62336	1.82489e-08	\\
1.63086	1.85622e-08	\\
1.63692	1.88155e-08	\\
1.64619	1.92031e-08	\\
1.65239	1.94623e-08	\\
1.65937	1.97544e-08	\\
1.665	1.999e-08	\\
1.67189	2.02784e-08	\\
1.67556	2.04319e-08	\\
1.68271	2.07314e-08	\\
1.68987	2.10307e-08	\\
1.69877	2.14031e-08	\\
1.70866	2.18164e-08	\\
1.71442	2.20569e-08	\\
1.72011	2.22947e-08	\\
1.72704	2.25837e-08	\\
1.73769	2.30274e-08	\\
1.74769	2.34439e-08	\\
1.75454	2.37285e-08	\\
1.76234	2.40523e-08	\\
1.77118	2.44185e-08	\\
1.78279	2.48991e-08	\\
1.79008	2.52001e-08	\\
1.80116	2.56562e-08	\\
1.80927	2.59898e-08	\\
1.81677	2.62975e-08	\\
1.8269	2.67119e-08	\\
1.83609	2.70867e-08	\\
1.84246	2.73462e-08	\\
1.84909	2.76154e-08	\\
1.86067	2.80845e-08	\\
1.86797	2.83796e-08	\\
1.87509	2.86664e-08	\\
1.88462	2.9049e-08	\\
1.89612	2.95093e-08	\\
1.90382	2.9816e-08	\\
1.91835	3.03928e-08	\\
1.92937	3.08279e-08	\\
1.94517	3.14482e-08	\\
1.95358	3.17768e-08	\\
1.96322	3.21518e-08	\\
1.97067	3.24404e-08	\\
1.97847	3.27413e-08	\\
1.99118	3.32295e-08	\\
1.99933	3.35411e-08	\\
2.00577	3.37862e-08	\\
2.01732	3.42239e-08	\\
2.02421	3.44837e-08	\\
2.03162	3.47622e-08	\\
2.0416	3.51356e-08	\\
2.05159	3.55069e-08	\\
2.06066	3.58425e-08	\\
2.07203	3.62608e-08	\\
2.07753	3.64621e-08	\\
2.08707	3.68096e-08	\\
2.09452	3.708e-08	\\
2.10783	3.75593e-08	\\
2.1198	3.79874e-08	\\
2.12924	3.83226e-08	\\
2.14501	3.88783e-08	\\
2.15928	3.93759e-08	\\
2.17265	3.98381e-08	\\
2.19081	4.04593e-08	\\
2.2014	4.0818e-08	\\
2.21006	4.11092e-08	\\
2.21576	4.13e-08	\\
2.22677	4.16663e-08	\\
2.23651	4.19879e-08	\\
2.25202	4.24951e-08	\\
2.26277	4.28433e-08	\\
2.27758	4.33185e-08	\\
2.2885	4.36654e-08	\\
2.2988	4.39899e-08	\\
2.30861	4.42963e-08	\\
2.31522	4.45018e-08	\\
2.32688	4.4861e-08	\\
2.34351	4.53677e-08	\\
2.35324	4.56609e-08	\\
2.36576	4.60349e-08	\\
2.36978	4.61542e-08	\\
2.3798	4.64496e-08	\\
2.39165	4.67957e-08	\\
2.40418	4.71576e-08	\\
2.41219	4.7387e-08	\\
2.42786	4.7831e-08	\\
2.43743	4.8099e-08	\\
2.44472	4.83017e-08	\\
2.45551	4.85992e-08	\\
2.46508	4.88606e-08	\\
2.47591	4.91537e-08	\\
2.49047	4.95429e-08	\\
2.50494	4.99244e-08	\\
2.51629	5.02201e-08	\\
2.53182	5.06191e-08	\\
2.54063	5.08429e-08	\\
2.55441	5.11888e-08	\\
2.56632	5.14839e-08	\\
2.57583	5.1717e-08	\\
2.59765	5.22431e-08	\\
2.61636	5.26849e-08	\\
2.62939	5.29871e-08	\\
2.64229	5.32823e-08	\\
2.65063	5.3471e-08	\\
2.66488	5.37891e-08	\\
2.67204	5.39472e-08	\\
2.68853	5.43062e-08	\\
2.70159	5.4586e-08	\\
2.71485	5.48654e-08	\\
2.72522	5.50812e-08	\\
2.73163	5.52131e-08	\\
2.74114	5.5407e-08	\\
2.75287	5.56431e-08	\\
2.75762	5.57377e-08	\\
2.76946	5.59714e-08	\\
2.79324	5.64302e-08	\\
2.80245	5.66043e-08	\\
2.82032	5.69362e-08	\\
2.83198	5.71487e-08	\\
2.84039	5.72999e-08	\\
2.85568	5.75705e-08	\\
2.87357	5.78801e-08	\\
2.88426	5.80614e-08	\\
2.90311	5.83748e-08	\\
2.91433	5.85575e-08	\\
2.91981	5.86455e-08	\\
2.93285	5.88524e-08	\\
2.94845	5.90947e-08	\\
2.95951	5.92631e-08	\\
2.97174	5.94462e-08	\\
2.98385	5.96242e-08	\\
3.00019	5.98589e-08	\\
3.00996	5.99963e-08	\\
3.02438	6.01954e-08	\\
3.03676	6.03626e-08	\\
3.05065	6.05461e-08	\\
3.06015	6.06692e-08	\\
3.07312	6.08341e-08	\\
3.09063	6.10508e-08	\\
3.10426	6.12149e-08	\\
3.11856	6.13828e-08	\\
3.13097	6.15249e-08	\\
3.14182	6.16465e-08	\\
3.15581	6.17996e-08	\\
3.17855	6.20397e-08	\\
3.1944	6.22007e-08	\\
3.20509	6.23065e-08	\\
3.21843	6.2435e-08	\\
3.23677	6.26058e-08	\\
3.25536	6.2772e-08	\\
3.27116	6.29078e-08	\\
3.28145	6.29936e-08	\\
3.28776	6.30451e-08	\\
3.29755	6.31236e-08	\\
3.31668	6.32714e-08	\\
3.33641	6.34165e-08	\\
3.35121	6.35203e-08	\\
3.36002	6.35801e-08	\\
3.37035	6.36484e-08	\\
3.38202	6.37231e-08	\\
3.40333	6.3853e-08	\\
3.41024	6.38933e-08	\\
3.42871	6.39966e-08	\\
3.4392	6.40525e-08	\\
3.45	6.4108e-08	\\
3.47698	6.42372e-08	\\
3.48591	6.42772e-08	\\
3.49916	6.43338e-08	\\
3.51073	6.43807e-08	\\
3.53144	6.44589e-08	\\
3.54278	6.44984e-08	\\
3.55771	6.45472e-08	\\
3.56828	6.45795e-08	\\
3.59019	6.46403e-08	\\
3.61177	6.46923e-08	\\
3.62231	6.47149e-08	\\
3.63224	6.47345e-08	\\
3.64821	6.47627e-08	\\
3.67865	6.48049e-08	\\
3.6901	6.4817e-08	\\
3.70816	6.48318e-08	\\
3.72825	6.48423e-08	\\
3.74526	6.48464e-08	\\
3.75968	6.48463e-08	\\
3.784	6.48391e-08	\\
3.79747	6.48313e-08	\\
3.81243	6.48195e-08	\\
3.82783	6.48039e-08	\\
3.84763	6.47788e-08	\\
3.85952	6.4761e-08	\\
3.87565	6.47337e-08	\\
3.90046	6.46845e-08	\\
3.91193	6.46589e-08	\\
3.93007	6.46147e-08	\\
3.9543	6.45487e-08	\\
3.97897	6.44734e-08	\\
3.99471	6.44212e-08	\\
4.01354	6.43545e-08	\\
4.02971	6.42935e-08	\\
4.0409	6.42494e-08	\\
4.05891	6.41751e-08	\\
4.07359	6.41116e-08	\\
4.08099	6.40785e-08	\\
4.10528	6.39653e-08	\\
4.1273	6.38565e-08	\\
4.14039	6.37891e-08	\\
4.1655	6.36542e-08	\\
4.18578	6.35399e-08	\\
4.21003	6.33971e-08	\\
4.23799	6.32243e-08	\\
4.25882	6.30899e-08	\\
4.27918	6.29541e-08	\\
4.2957	6.28406e-08	\\
4.31762	6.26856e-08	\\
4.33062	6.25913e-08	\\
4.35793	6.23877e-08	\\
4.37829	6.22309e-08	\\
4.4087	6.19892e-08	\\
4.42985	6.18159e-08	\\
4.44854	6.16591e-08	\\
4.46135	6.15498e-08	\\
4.48603	6.13349e-08	\\
4.51	6.1121e-08	\\
4.53141	6.09255e-08	\\
4.55237	6.07303e-08	\\
4.5749	6.05162e-08	\\
4.59724	6.02998e-08	\\
4.60872	6.01869e-08	\\
4.62886	5.99863e-08	\\
4.64609	5.9812e-08	\\
4.70087	5.92426e-08	\\
4.72547	5.89795e-08	\\
4.74041	5.88175e-08	\\
4.75	5.87126e-08	\\
4.7787	5.83949e-08	\\
4.78755	5.82957e-08	\\
4.80747	5.80705e-08	\\
4.83792	5.77209e-08	\\
4.85137	5.75646e-08	\\
4.87628	5.72718e-08	\\
4.89058	5.7102e-08	\\
4.901	5.69774e-08	\\
4.91963	5.6753e-08	\\
4.94708	5.64184e-08	\\
4.96503	5.61972e-08	\\
4.98023	5.60083e-08	\\
5.00062	5.57528e-08	\\
5.03209	5.5354e-08	\\
5.04733	5.51589e-08	\\
5.05957	5.50013e-08	\\
5.089	5.4619e-08	\\
5.11273	5.43075e-08	\\
5.139	5.39594e-08	\\
5.17036	5.35394e-08	\\
5.20238	5.31058e-08	\\
5.22068	5.28558e-08	\\
5.24171	5.25669e-08	\\
5.27082	5.21636e-08	\\
5.29698	5.17982e-08	\\
5.3122	5.15844e-08	\\
5.33363	5.12817e-08	\\
5.35787	5.09371e-08	\\
5.38709	5.0519e-08	\\
5.41346	5.01389e-08	\\
5.4475	4.96448e-08	\\
5.48522	4.90929e-08	\\
5.50637	4.87814e-08	\\
5.5248	4.85089e-08	\\
5.55313	4.80882e-08	\\
5.57634	4.77419e-08	\\
5.60366	4.73323e-08	\\
5.62191	4.70576e-08	\\
5.65958	4.64882e-08	\\
5.69471	4.5954e-08	\\
5.71707	4.56128e-08	\\
5.73437	4.5348e-08	\\
5.75999	4.49549e-08	\\
5.78241	4.46099e-08	\\
5.80886	4.42016e-08	\\
5.82328	4.39786e-08	\\
5.85852	4.34322e-08	\\
5.88786	4.29757e-08	\\
5.917	4.25215e-08	\\
5.93246	4.228e-08	\\
5.96588	4.17571e-08	\\
5.99341	4.13255e-08	\\
6.011	4.10493e-08	\\
6.03999	4.05936e-08	\\
6.06491	4.02014e-08	\\
6.09937	3.96585e-08	\\
6.12948	3.91834e-08	\\
6.16636	3.86014e-08	\\
6.19125	3.82081e-08	\\
6.21746	3.7794e-08	\\
6.24477	3.73626e-08	\\
6.26611	3.70252e-08	\\
6.28719	3.66923e-08	\\
6.3129	3.62862e-08	\\
6.34627	3.57593e-08	\\
6.37139	3.5363e-08	\\
6.39363	3.50123e-08	\\
6.41025	3.47503e-08	\\
6.43865	3.43032e-08	\\
6.47916	3.36664e-08	\\
6.5102	3.31792e-08	\\
6.52663	3.29217e-08	\\
6.57432	3.21758e-08	\\
6.61072	3.1608e-08	\\
6.63516	3.12276e-08	\\
6.68209	3.04996e-08	\\
6.70497	3.01456e-08	\\
6.7332	2.97101e-08	\\
6.76038	2.92919e-08	\\
6.79327	2.87874e-08	\\
6.82495	2.83032e-08	\\
6.86837	2.76427e-08	\\
6.88989	2.73166e-08	\\
6.91276	2.69711e-08	\\
6.93727	2.6602e-08	\\
6.99222	2.57791e-08	\\
7.0143	2.54504e-08	\\
7.03642	2.51222e-08	\\
7.07477	2.45558e-08	\\
7.09945	2.41933e-08	\\
7.12214	2.38613e-08	\\
7.15202	2.34262e-08	\\
7.18136	2.30013e-08	\\
7.22839	2.2325e-08	\\
7.29291	2.14073e-08	\\
7.32937	2.08943e-08	\\
7.34602	2.06614e-08	\\
7.39887	1.99276e-08	\\
7.42534	1.95635e-08	\\
7.45516	1.9156e-08	\\
7.48223	1.87886e-08	\\
7.51391	1.83619e-08	\\
7.56059	1.77394e-08	\\
7.6048	1.71571e-08	\\
7.62181	1.69349e-08	\\
7.64773	1.65984e-08	\\
7.69234	1.60251e-08	\\
7.73359	1.55017e-08	\\
7.77596	1.4971e-08	\\
7.82277	1.43931e-08	\\
7.86155	1.39211e-08	\\
7.89303	1.35424e-08	\\
7.93264	1.30719e-08	\\
7.97912	1.25283e-08	\\
8.00993	1.21731e-08	\\
8.05018	1.17155e-08	\\
8.09962	1.11631e-08	\\
8.13586	1.07655e-08	\\
8.17892	1.03007e-08	\\
8.2152	9.91584e-09	\\
8.24147	9.64114e-09	\\
8.27699	9.2749e-09	\\
8.31588	8.88094e-09	\\
8.36978	8.34719e-09	\\
8.39427	8.10943e-09	\\
8.43448	7.72553e-09	\\
8.47906	7.30966e-09	\\
8.50451	7.07677e-09	\\
8.52412	6.89966e-09	\\
8.57725	6.42982e-09	\\
8.60789	6.16564e-09	\\
8.6428	5.87072e-09	\\
8.70304	5.37735e-09	\\
8.76292	4.90655e-09	\\
8.79102	4.69241e-09	\\
8.8193	4.48139e-09	\\
8.86518	4.14859e-09	\\
8.91599	3.7939e-09	\\
8.96533	3.46373e-09	\\
8.99896	3.24677e-09	\\
9.03091	3.04673e-09	\\
9.06498	2.84008e-09	\\
9.11212	2.56547e-09	\\
9.15748	2.31372e-09	\\
9.20151	2.08126e-09	\\
9.2463	1.85687e-09	\\
9.30386	1.5866e-09	\\
9.35036	1.38319e-09	\\
9.41429	1.12561e-09	\\
9.45702	9.67838e-10	\\
9.49771	8.28379e-10	\\
9.55248	6.5738e-10	\\
9.61597	4.83427e-10	\\
9.68052	3.33495e-10	\\
9.70768	2.78598e-10	\\
9.7569	1.91531e-10	\\
9.79248	1.38628e-10	\\
9.83031	9.16727e-11	\\
9.87647	4.7405e-11	\\
9.90852	2.515e-11	\\
9.95018	6.65833e-12	\\
};
\addplot [color=blue,dashed,forget plot]
  table[row sep=crcr]{
0.100016	0	\\
0.100732	3.13367e-11	\\
0.101143	7.68026e-11	\\
0.101868	2.03837e-10	\\
0.102291	3.04453e-10	\\
0.102765	4.39484e-10	\\
0.103121	5.55899e-10	\\
0.103483	6.86735e-10	\\
0.103876	8.43426e-10	\\
0.104476	1.10953e-09	\\
0.104665	1.19998e-09	\\
0.105362	1.56011e-09	\\
0.105744	1.77495e-09	\\
0.106062	1.96245e-09	\\
0.106298	2.10684e-09	\\
0.106648	2.32861e-09	\\
0.10715	2.66203e-09	\\
0.107607	2.98173e-09	\\
0.108043	3.29979e-09	\\
0.108317	3.50588e-09	\\
0.108894	3.95602e-09	\\
0.109224	4.22318e-09	\\
0.109604	4.5384e-09	\\
0.110172	5.0249e-09	\\
0.110505	5.31783e-09	\\
0.110927	5.69859e-09	\\
0.111227	5.97419e-09	\\
0.111839	6.55158e-09	\\
0.112341	7.03809e-09	\\
0.112781	7.47393e-09	\\
0.113469	8.17189e-09	\\
0.113904	8.62317e-09	\\
0.114661	9.42594e-09	\\
0.115014	9.80681e-09	\\
0.115804	1.06764e-08	\\
0.11616	1.10745e-08	\\
0.11676	1.17545e-08	\\
0.1172	1.22601e-08	\\
0.117616	1.27425e-08	\\
0.118117	1.33299e-08	\\
0.118705	1.40269e-08	\\
0.119312	1.47552e-08	\\
0.119771	1.53111e-08	\\
0.120376	1.60493e-08	\\
0.121135	1.69867e-08	\\
0.121529	1.74764e-08	\\
0.122052	1.81313e-08	\\
0.12263	1.88591e-08	\\
0.123087	1.94369e-08	\\
0.123626	2.01234e-08	\\
0.124085	2.07089e-08	\\
0.124676	2.14682e-08	\\
0.125112	2.20296e-08	\\
0.125592	2.26499e-08	\\
0.126054	2.32479e-08	\\
0.126291	2.35564e-08	\\
0.126904	2.43536e-08	\\
0.127492	2.51201e-08	\\
0.127792	2.55113e-08	\\
0.128534	2.64815e-08	\\
0.129167	2.73107e-08	\\
0.129774	2.81064e-08	\\
0.130294	2.87877e-08	\\
0.130818	2.94748e-08	\\
0.131487	3.03522e-08	\\
0.132056	3.10975e-08	\\
0.132644	3.18676e-08	\\
0.133001	3.23341e-08	\\
0.133666	3.3204e-08	\\
0.134361	3.41095e-08	\\
0.13507	3.50319e-08	\\
0.135588	3.57053e-08	\\
0.135991	3.62271e-08	\\
0.13657	3.69754e-08	\\
0.13714	3.77097e-08	\\
0.13769	3.84166e-08	\\
0.138229	3.91075e-08	\\
0.138825	3.98683e-08	\\
0.139339	4.05234e-08	\\
0.140146	4.15446e-08	\\
0.140895	4.24892e-08	\\
0.141373	4.30882e-08	\\
0.141897	4.37436e-08	\\
0.142535	4.45361e-08	\\
0.143589	4.58381e-08	\\
0.144197	4.65828e-08	\\
0.144743	4.72488e-08	\\
0.145248	4.78611e-08	\\
0.145703	4.84116e-08	\\
0.146385	4.92295e-08	\\
0.147035	5.00059e-08	\\
0.147801	5.09121e-08	\\
0.148257	5.14489e-08	\\
0.149005	5.23223e-08	\\
0.149751	5.3187e-08	\\
0.150789	5.43779e-08	\\
0.151369	5.50366e-08	\\
0.15236	5.61516e-08	\\
0.153206	5.7093e-08	\\
0.153923	5.78828e-08	\\
0.154616	5.86395e-08	\\
0.155702	5.9811e-08	\\
0.156276	6.04236e-08	\\
0.157109	6.13035e-08	\\
0.157913	6.21432e-08	\\
0.158792	6.30502e-08	\\
0.159332	6.36026e-08	\\
0.160076	6.43547e-08	\\
0.160613	6.48935e-08	\\
0.161543	6.5815e-08	\\
0.161952	6.62166e-08	\\
0.162943	6.7179e-08	\\
0.163795	6.79942e-08	\\
0.164577	6.87323e-08	\\
0.165563	6.9651e-08	\\
0.166526	7.05334e-08	\\
0.167275	7.12096e-08	\\
0.167903	7.17703e-08	\\
0.168356	7.21716e-08	\\
0.169313	7.30082e-08	\\
0.170372	7.39179e-08	\\
0.171347	7.47403e-08	\\
0.172157	7.5413e-08	\\
0.172836	7.59689e-08	\\
0.173425	7.64458e-08	\\
0.173801	7.67473e-08	\\
0.174706	7.74652e-08	\\
0.175346	7.79654e-08	\\
0.176423	7.87942e-08	\\
0.177117	7.93185e-08	\\
0.178062	8.00218e-08	\\
0.178898	8.06332e-08	\\
0.180102	8.14957e-08	\\
0.180862	8.20298e-08	\\
0.182027	8.2832e-08	\\
0.184089	8.42053e-08	\\
0.18487	8.47099e-08	\\
0.185944	8.53904e-08	\\
0.18661	8.58042e-08	\\
0.187504	8.63503e-08	\\
0.188082	8.66973e-08	\\
0.188911	8.71878e-08	\\
0.189884	8.77521e-08	\\
0.190693	8.82111e-08	\\
0.191648	8.87421e-08	\\
0.19297	8.9458e-08	\\
0.194005	9.00028e-08	\\
0.195319	9.06748e-08	\\
0.19641	9.12165e-08	\\
0.197237	9.16171e-08	\\
0.198188	9.20676e-08	\\
0.198714	9.23118e-08	\\
0.199929	9.28639e-08	\\
0.200903	9.32934e-08	\\
0.201795	9.36769e-08	\\
0.202715	9.4063e-08	\\
0.203536	9.43994e-08	\\
0.204572	9.48124e-08	\\
0.205426	9.51439e-08	\\
0.207216	9.58125e-08	\\
0.207983	9.60883e-08	\\
0.208602	9.63063e-08	\\
0.209736	9.66947e-08	\\
0.210653	9.69988e-08	\\
0.211348	9.72234e-08	\\
0.212512	9.75881e-08	\\
0.213482	9.78818e-08	\\
0.214593	9.82062e-08	\\
0.215334	9.84158e-08	\\
0.216055	9.86144e-08	\\
0.217124	9.88998e-08	\\
0.21783	9.9082e-08	\\
0.219062	9.93886e-08	\\
0.219841	9.95753e-08	\\
0.220808	9.97989e-08	\\
0.222028	1.00069e-07	\\
0.222801	1.00233e-07	\\
0.223564	1.0039e-07	\\
0.224473	1.0057e-07	\\
0.225435	1.00752e-07	\\
0.226929	1.0102e-07	\\
0.228105	1.01217e-07	\\
0.229383	1.01419e-07	\\
0.230274	1.01552e-07	\\
0.231502	1.01724e-07	\\
0.23291	1.01907e-07	\\
0.233592	1.0199e-07	\\
0.234564	1.02102e-07	\\
0.235642	1.02217e-07	\\
0.237129	1.02363e-07	\\
0.238486	1.02482e-07	\\
0.239473	1.0256e-07	\\
0.240467	1.02632e-07	\\
0.241136	1.02676e-07	\\
0.242537	1.02759e-07	\\
0.244035	1.02833e-07	\\
0.245076	1.02875e-07	\\
0.246009	1.02908e-07	\\
0.247	1.02936e-07	\\
0.247885	1.02956e-07	\\
0.249225	1.02976e-07	\\
0.250428	1.02986e-07	\\
0.251459	1.02987e-07	\\
0.252329	1.02983e-07	\\
0.253517	1.02971e-07	\\
0.254847	1.02947e-07	\\
0.255904	1.02921e-07	\\
0.257098	1.02885e-07	\\
0.258282	1.02841e-07	\\
0.259441	1.02791e-07	\\
0.260833	1.02722e-07	\\
0.262131	1.02648e-07	\\
0.262987	1.02595e-07	\\
0.263852	1.02538e-07	\\
0.264951	1.0246e-07	\\
0.266916	1.02306e-07	\\
0.268033	1.02211e-07	\\
0.269649	1.02063e-07	\\
0.271025	1.01928e-07	\\
0.272281	1.01798e-07	\\
0.274193	1.01587e-07	\\
0.275028	1.0149e-07	\\
0.276161	1.01354e-07	\\
0.27739	1.01201e-07	\\
0.278326	1.0108e-07	\\
0.279306	1.00951e-07	\\
0.280599	1.00774e-07	\\
0.281796	1.00605e-07	\\
0.282404	1.00518e-07	\\
0.282916	1.00443e-07	\\
0.284345	1.00229e-07	\\
0.286539	9.98869e-08	\\
0.288465	9.95738e-08	\\
0.289577	9.93878e-08	\\
0.290789	9.91806e-08	\\
0.291891	9.89883e-08	\\
0.292729	9.88397e-08	\\
0.293549	9.86923e-08	\\
0.294692	9.84837e-08	\\
0.29527	9.83768e-08	\\
0.296798	9.80897e-08	\\
0.298342	9.77931e-08	\\
0.299756	9.75161e-08	\\
0.301579	9.71517e-08	\\
0.302548	9.69546e-08	\\
0.304394	9.65728e-08	\\
0.306557	9.61154e-08	\\
0.308453	9.5706e-08	\\
0.310287	9.53025e-08	\\
0.311845	9.49545e-08	\\
0.313014	9.46902e-08	\\
0.314818	9.42771e-08	\\
0.316266	9.39411e-08	\\
0.317626	9.3622e-08	\\
0.319309	9.32229e-08	\\
0.320928	9.28343e-08	\\
0.322521	9.24478e-08	\\
0.323438	9.22235e-08	\\
0.324975	9.18448e-08	\\
0.32696	9.13505e-08	\\
0.328393	9.09903e-08	\\
0.329228	9.07791e-08	\\
0.330537	9.0446e-08	\\
0.331616	9.01699e-08	\\
0.332773	8.98723e-08	\\
0.334735	8.93639e-08	\\
0.335838	8.9076e-08	\\
0.336532	8.88941e-08	\\
0.338708	8.83208e-08	\\
0.339764	8.8041e-08	\\
0.340906	8.77368e-08	\\
0.342263	8.73739e-08	\\
0.344496	8.67731e-08	\\
0.346563	8.62129e-08	\\
0.348333	8.57307e-08	\\
0.349845	8.53168e-08	\\
0.352231	8.46602e-08	\\
0.353715	8.42499e-08	\\
0.355412	8.37791e-08	\\
0.35655	8.34624e-08	\\
0.358077	8.30363e-08	\\
0.360099	8.24704e-08	\\
0.361818	8.19876e-08	\\
0.364139	8.13335e-08	\\
0.36546	8.09602e-08	\\
0.367669	8.03349e-08	\\
0.369975	7.96802e-08	\\
0.371557	7.92301e-08	\\
0.372694	7.89064e-08	\\
0.373774	7.85985e-08	\\
0.375262	7.8174e-08	\\
0.377306	7.75899e-08	\\
0.380459	7.66881e-08	\\
0.382933	7.59795e-08	\\
0.384678	7.54798e-08	\\
0.385902	7.5129e-08	\\
0.388727	7.43193e-08	\\
0.391005	7.36665e-08	\\
0.393414	7.29766e-08	\\
0.395144	7.24813e-08	\\
0.396611	7.20617e-08	\\
0.398773	7.14438e-08	\\
0.400663	7.09042e-08	\\
0.403071	7.02176e-08	\\
0.404874	6.9704e-08	\\
0.406304	6.92976e-08	\\
0.407502	6.89572e-08	\\
0.408969	6.85409e-08	\\
0.410986	6.79698e-08	\\
0.413545	6.72466e-08	\\
0.415279	6.67577e-08	\\
0.418128	6.59567e-08	\\
0.420064	6.54139e-08	\\
0.421823	6.49218e-08	\\
0.424593	6.41498e-08	\\
0.426398	6.36482e-08	\\
0.428145	6.3164e-08	\\
0.430117	6.26193e-08	\\
0.431719	6.21778e-08	\\
0.434004	6.15503e-08	\\
0.435639	6.11027e-08	\\
0.438002	6.04584e-08	\\
0.44003	5.99077e-08	\\
0.441192	5.9593e-08	\\
0.444417	5.87234e-08	\\
0.446955	5.80428e-08	\\
0.44909	5.74731e-08	\\
0.451835	5.67445e-08	\\
0.453309	5.63551e-08	\\
0.454813	5.59588e-08	\\
0.457958	5.51349e-08	\\
0.459862	5.46389e-08	\\
0.462247	5.40207e-08	\\
0.464619	5.34095e-08	\\
0.465952	5.30676e-08	\\
0.467532	5.26637e-08	\\
0.468681	5.2371e-08	\\
0.471989	5.15331e-08	\\
0.47471	5.08492e-08	\\
0.476291	5.0454e-08	\\
0.477826	5.00721e-08	\\
0.479499	4.96574e-08	\\
0.482866	4.88287e-08	\\
0.485005	4.83062e-08	\\
0.486881	4.78506e-08	\\
0.489435	4.72343e-08	\\
0.491926	4.66376e-08	\\
0.495021	4.5902e-08	\\
0.498525	4.50776e-08	\\
0.50063	4.45865e-08	\\
0.504166	4.37687e-08	\\
0.505665	4.34248e-08	\\
0.507841	4.29282e-08	\\
0.510741	4.22719e-08	\\
0.512662	4.18405e-08	\\
0.514519	4.1426e-08	\\
0.517737	4.07137e-08	\\
0.520304	4.01508e-08	\\
0.522653	3.96401e-08	\\
0.525222	3.9086e-08	\\
0.527856	3.85231e-08	\\
0.529053	3.8269e-08	\\
0.530345	3.79959e-08	\\
0.532671	3.75071e-08	\\
0.537014	3.66053e-08	\\
0.539905	3.60127e-08	\\
0.542539	3.5478e-08	\\
0.545665	3.48503e-08	\\
0.548282	3.43302e-08	\\
0.55136	3.37248e-08	\\
0.553736	3.32624e-08	\\
0.557423	3.25527e-08	\\
0.559872	3.2087e-08	\\
0.563057	3.14876e-08	\\
0.566709	3.08093e-08	\\
0.568692	3.04451e-08	\\
0.570944	3.00349e-08	\\
0.572341	2.97822e-08	\\
0.576713	2.90007e-08	\\
0.578029	2.8768e-08	\\
0.58042	2.83486e-08	\\
0.583278	2.78527e-08	\\
0.58787	2.70676e-08	\\
0.591899	2.63912e-08	\\
0.596812	2.55814e-08	\\
0.598832	2.52534e-08	\\
0.602077	2.47323e-08	\\
0.603788	2.44605e-08	\\
0.605351	2.42138e-08	\\
0.609567	2.35568e-08	\\
0.612397	2.31226e-08	\\
0.613894	2.28952e-08	\\
0.617595	2.2339e-08	\\
0.618871	2.21495e-08	\\
0.621349	2.17843e-08	\\
0.623221	2.15113e-08	\\
0.625753	2.11454e-08	\\
0.630458	2.04767e-08	\\
0.631941	2.0269e-08	\\
0.636069	1.9698e-08	\\
0.638582	1.93558e-08	\\
0.642931	1.87729e-08	\\
0.646184	1.83446e-08	\\
0.649208	1.79525e-08	\\
0.651934	1.76038e-08	\\
0.655829	1.71134e-08	\\
0.659188	1.6698e-08	\\
0.661905	1.63669e-08	\\
0.664688	1.60324e-08	\\
0.668699	1.55583e-08	\\
0.673064	1.50531e-08	\\
0.675007	1.48317e-08	\\
0.677923	1.45036e-08	\\
0.680482	1.42197e-08	\\
0.683106	1.39324e-08	\\
0.686506	1.3566e-08	\\
0.68944	1.32549e-08	\\
0.695155	1.26626e-08	\\
0.699714	1.2203e-08	\\
0.701674	1.20087e-08	\\
0.705204	1.16641e-08	\\
0.708283	1.1369e-08	\\
0.710861	1.11257e-08	\\
0.713441	1.08856e-08	\\
0.716795	1.05785e-08	\\
0.718886	1.03901e-08	\\
0.722192	1.00966e-08	\\
0.725998	9.76552e-09	\\
0.730187	9.40941e-09	\\
0.73199	9.25873e-09	\\
0.736732	8.87007e-09	\\
0.741189	8.51455e-09	\\
0.743401	8.34153e-09	\\
0.74612	8.13207e-09	\\
0.750475	7.80369e-09	\\
0.753352	7.59157e-09	\\
0.755676	7.42289e-09	\\
0.757967	7.25906e-09	\\
0.762502	6.94152e-09	\\
0.766831	6.64685e-09	\\
0.771437	6.3422e-09	\\
0.77395	6.17978e-09	\\
0.776972	5.98802e-09	\\
0.781398	5.71404e-09	\\
0.784609	5.52036e-09	\\
0.787748	5.335e-09	\\
0.791038	5.14506e-09	\\
0.794143	4.96969e-09	\\
0.79783	4.76637e-09	\\
0.802238	4.53021e-09	\\
0.806834	4.29191e-09	\\
0.809551	4.15479e-09	\\
0.812533	4.00737e-09	\\
0.81617	3.8321e-09	\\
0.820285	3.6395e-09	\\
0.824309	3.45706e-09	\\
0.826707	3.35106e-09	\\
0.833319	3.06908e-09	\\
0.835586	2.97584e-09	\\
0.838401	2.86251e-09	\\
0.842981	2.68374e-09	\\
0.847488	2.51452e-09	\\
0.851709	2.36201e-09	\\
0.855601	2.22638e-09	\\
0.860051	2.07715e-09	\\
0.864038	1.9486e-09	\\
0.867779	1.83236e-09	\\
0.870037	1.76424e-09	\\
0.87366	1.65806e-09	\\
0.880088	1.47907e-09	\\
0.884857	1.35391e-09	\\
0.889037	1.24942e-09	\\
0.893547	1.14206e-09	\\
0.896011	1.08574e-09	\\
0.900526	9.86701e-10	\\
0.903659	9.21133e-10	\\
0.907433	8.45513e-10	\\
0.912219	7.54852e-10	\\
0.913714	7.27681e-10	\\
0.915829	6.90245e-10	\\
0.918211	6.49374e-10	\\
0.922562	5.78284e-10	\\
0.926574	5.16764e-10	\\
0.93001	4.67097e-10	\\
0.939074	3.49136e-10	\\
0.94474	2.84764e-10	\\
0.947393	2.57043e-10	\\
0.951158	2.20308e-10	\\
0.957049	1.68862e-10	\\
0.963964	1.17631e-10	\\
0.967383	9.58724e-11	\\
0.969213	8.51808e-11	\\
0.974442	5.82461e-11	\\
0.979112	3.86298e-11	\\
0.985362	1.87901e-11	\\
0.987692	1.32427e-11	\\
0.991722	5.95191e-12	\\
0.996869	8.39551e-13	\\
1.00069	3.85247e-14	\\
1.00686	4.00202e-12	\\
1.0114	1.09629e-11	\\
1.01564	2.05156e-11	\\
1.01883	2.95781e-11	\\
1.02234	4.14047e-11	\\
1.02492	5.13255e-11	\\
1.02894	6.88249e-11	\\
1.03158	8.16304e-11	\\
1.0349	9.92029e-11	\\
1.04362	1.5294e-10	\\
1.04669	1.74459e-10	\\
1.05053	2.0319e-10	\\
1.05595	2.471e-10	\\
1.06253	3.05673e-10	\\
1.06742	3.52785e-10	\\
1.07338	4.14252e-10	\\
1.07781	4.62662e-10	\\
1.08253	5.16932e-10	\\
1.08601	5.58574e-10	\\
1.09	6.08066e-10	\\
1.09294	6.45515e-10	\\
1.09738	7.04059e-10	\\
1.102	7.67284e-10	\\
1.10895	8.66479e-10	\\
1.11633	9.7726e-10	\\
1.12186	1.06365e-09	\\
1.12619	1.13348e-09	\\
1.1329	1.24483e-09	\\
1.13615	1.30032e-09	\\
1.14014	1.36972e-09	\\
1.14299	1.42004e-09	\\
1.14711	1.49402e-09	\\
1.15176	1.57939e-09	\\
1.15845	1.70489e-09	\\
1.16166	1.76647e-09	\\
1.16882	1.90646e-09	\\
1.1752	2.03456e-09	\\
1.17957	2.12374e-09	\\
1.18216	2.17739e-09	\\
1.19114	2.36655e-09	\\
1.19497	2.44878e-09	\\
1.20001	2.55829e-09	\\
1.20526	2.67412e-09	\\
1.21056	2.79267e-09	\\
1.21672	2.93248e-09	\\
1.22073	3.02447e-09	\\
1.23028	3.24726e-09	\\
1.23784	3.42673e-09	\\
1.24459	3.5894e-09	\\
1.24788	3.66928e-09	\\
1.25128	3.75239e-09	\\
1.25545	3.85525e-09	\\
1.25793	3.91662e-09	\\
1.26242	4.02829e-09	\\
1.26925	4.19984e-09	\\
1.27407	4.32213e-09	\\
1.27764	4.41315e-09	\\
1.28445	4.58789e-09	\\
1.28897	4.70487e-09	\\
1.29525	4.86831e-09	\\
1.3049	5.12186e-09	\\
1.30871	5.22275e-09	\\
1.31864	5.48715e-09	\\
1.32815	5.74291e-09	\\
1.34101	6.09153e-09	\\
1.3439	6.17036e-09	\\
1.34954	6.32474e-09	\\
1.35348	6.43269e-09	\\
1.35862	6.57422e-09	\\
1.36647	6.79106e-09	\\
1.37219	6.94958e-09	\\
1.3777	7.10264e-09	\\
1.38335	7.25981e-09	\\
1.38883	7.41275e-09	\\
1.39808	7.67147e-09	\\
1.40462	7.85493e-09	\\
1.40889	7.97486e-09	\\
1.41678	8.19655e-09	\\
1.42489	8.42491e-09	\\
1.4286	8.52949e-09	\\
1.43541	8.72134e-09	\\
1.44356	8.95119e-09	\\
1.447	9.04833e-09	\\
1.45273	9.20995e-09	\\
1.45783	9.35374e-09	\\
1.46261	9.48858e-09	\\
1.46644	9.59647e-09	\\
1.47306	9.78305e-09	\\
1.48112	1.00101e-08	\\
1.48631	1.0156e-08	\\
1.49152	1.03026e-08	\\
1.498	1.04844e-08	\\
1.50889	1.07895e-08	\\
1.51635	1.09983e-08	\\
1.52411	1.12147e-08	\\
1.53298	1.14615e-08	\\
1.53867	1.16192e-08	\\
1.54339	1.175e-08	\\
1.55362	1.20326e-08	\\
1.56628	1.23803e-08	\\
1.57436	1.26014e-08	\\
1.58248	1.28226e-08	\\
1.58694	1.29438e-08	\\
1.59426	1.31419e-08	\\
1.6016	1.33397e-08	\\
1.60963	1.35554e-08	\\
1.61845	1.37909e-08	\\
1.62336	1.39215e-08	\\
1.63086	1.41202e-08	\\
1.63692	1.42801e-08	\\
1.64619	1.45233e-08	\\
1.65239	1.4685e-08	\\
1.65937	1.48664e-08	\\
1.665	1.5012e-08	\\
1.67189	1.51894e-08	\\
1.67556	1.52834e-08	\\
1.68271	1.54661e-08	\\
1.68987	1.56478e-08	\\
1.69877	1.58725e-08	\\
1.70866	1.612e-08	\\
1.71442	1.62633e-08	\\
1.72011	1.64042e-08	\\
1.72704	1.65747e-08	\\
1.73769	1.68348e-08	\\
1.74769	1.7077e-08	\\
1.75454	1.72414e-08	\\
1.76234	1.74275e-08	\\
1.77118	1.76365e-08	\\
1.78279	1.79087e-08	\\
1.79008	1.80779e-08	\\
1.80116	1.83325e-08	\\
1.80927	1.85173e-08	\\
1.81677	1.86867e-08	\\
1.8269	1.89134e-08	\\
1.83609	1.91168e-08	\\
1.84246	1.92567e-08	\\
1.84909	1.94011e-08	\\
1.86067	1.9651e-08	\\
1.86797	1.9807e-08	\\
1.87509	1.99577e-08	\\
1.88462	2.01574e-08	\\
1.89612	2.03957e-08	\\
1.90382	2.05531e-08	\\
1.91835	2.08466e-08	\\
1.92937	2.10657e-08	\\
1.94517	2.13745e-08	\\
1.95358	2.15365e-08	\\
1.96322	2.17198e-08	\\
1.97067	2.18599e-08	\\
1.97847	2.20051e-08	\\
1.99118	2.22385e-08	\\
1.99933	2.23861e-08	\\
2.00577	2.25015e-08	\\
2.01732	2.27059e-08	\\
2.02421	2.28263e-08	\\
2.03162	2.29544e-08	\\
2.0416	2.3125e-08	\\
2.05159	2.32931e-08	\\
2.06066	2.34437e-08	\\
2.07203	2.36296e-08	\\
2.07753	2.37184e-08	\\
2.08707	2.38707e-08	\\
2.09452	2.39882e-08	\\
2.10783	2.41944e-08	\\
2.1198	2.43765e-08	\\
2.12924	2.45175e-08	\\
2.14501	2.47485e-08	\\
2.15928	2.49523e-08	\\
2.17265	2.5139e-08	\\
2.19081	2.5386e-08	\\
2.2014	2.55264e-08	\\
2.21006	2.56393e-08	\\
2.21576	2.57127e-08	\\
2.22677	2.58524e-08	\\
2.23651	2.59737e-08	\\
2.25202	2.61623e-08	\\
2.26277	2.629e-08	\\
2.27758	2.64616e-08	\\
2.2885	2.65851e-08	\\
2.2988	2.66992e-08	\\
2.30861	2.68056e-08	\\
2.31522	2.68763e-08	\\
2.32688	2.69986e-08	\\
2.34351	2.71679e-08	\\
2.35324	2.72643e-08	\\
2.36576	2.73855e-08	\\
2.36978	2.74238e-08	\\
2.3798	2.75176e-08	\\
2.39165	2.76259e-08	\\
2.40418	2.77373e-08	\\
2.41219	2.78069e-08	\\
2.42786	2.79394e-08	\\
2.43743	2.80179e-08	\\
2.44472	2.80765e-08	\\
2.45551	2.81614e-08	\\
2.46508	2.82348e-08	\\
2.47591	2.83158e-08	\\
2.49047	2.84212e-08	\\
2.50494	2.8522e-08	\\
2.51629	2.85985e-08	\\
2.53182	2.86991e-08	\\
2.54063	2.87544e-08	\\
2.55441	2.8838e-08	\\
2.56632	2.89075e-08	\\
2.57583	2.89613e-08	\\
2.59765	2.90788e-08	\\
2.61636	2.91732e-08	\\
2.62939	2.92354e-08	\\
2.64229	2.92944e-08	\\
2.65063	2.9331e-08	\\
2.66488	2.9391e-08	\\
2.67204	2.942e-08	\\
2.68853	2.94836e-08	\\
2.70159	2.95311e-08	\\
2.71485	2.95765e-08	\\
2.72522	2.96102e-08	\\
2.73163	2.96302e-08	\\
2.74114	2.96589e-08	\\
2.75287	2.96923e-08	\\
2.75762	2.97053e-08	\\
2.76946	2.97362e-08	\\
2.79324	2.97924e-08	\\
2.80245	2.9812e-08	\\
2.82032	2.98468e-08	\\
2.83198	2.98671e-08	\\
2.84039	2.98807e-08	\\
2.85568	2.99029e-08	\\
2.87357	2.99251e-08	\\
2.88426	2.99364e-08	\\
2.90311	2.99528e-08	\\
2.91433	2.99605e-08	\\
2.91981	2.99637e-08	\\
2.93285	2.99699e-08	\\
2.94845	2.99746e-08	\\
2.95951	2.99762e-08	\\
2.97174	2.99763e-08	\\
2.98385	2.99747e-08	\\
3.00019	2.997e-08	\\
3.00996	2.99658e-08	\\
3.02438	2.99576e-08	\\
3.03676	2.99488e-08	\\
3.05065	2.9937e-08	\\
3.06015	2.99278e-08	\\
3.07312	2.99137e-08	\\
3.09063	2.9892e-08	\\
3.10426	2.98729e-08	\\
3.11856	2.9851e-08	\\
3.13097	2.98304e-08	\\
3.14182	2.98112e-08	\\
3.15581	2.97848e-08	\\
3.17855	2.9738e-08	\\
3.1944	2.97027e-08	\\
3.20509	2.96776e-08	\\
3.21843	2.96449e-08	\\
3.23677	2.95975e-08	\\
3.25536	2.95467e-08	\\
3.27116	2.95012e-08	\\
3.28145	2.94705e-08	\\
3.28776	2.94513e-08	\\
3.29755	2.94209e-08	\\
3.31668	2.93593e-08	\\
3.33641	2.92929e-08	\\
3.35121	2.92412e-08	\\
3.36002	2.92097e-08	\\
3.37035	2.9172e-08	\\
3.38202	2.91286e-08	\\
3.40333	2.90469e-08	\\
3.41024	2.90197e-08	\\
3.42871	2.89456e-08	\\
3.4392	2.89025e-08	\\
3.45	2.88574e-08	\\
3.47698	2.87416e-08	\\
3.48591	2.87023e-08	\\
3.49916	2.86431e-08	\\
3.51073	2.85906e-08	\\
3.53144	2.84946e-08	\\
3.54278	2.84411e-08	\\
3.55771	2.83695e-08	\\
3.56828	2.8318e-08	\\
3.59019	2.82096e-08	\\
3.61177	2.81004e-08	\\
3.62231	2.80461e-08	\\
3.63224	2.79946e-08	\\
3.64821	2.79107e-08	\\
3.67865	2.77475e-08	\\
3.6901	2.76849e-08	\\
3.70816	2.75852e-08	\\
3.72825	2.74726e-08	\\
3.74526	2.73759e-08	\\
3.75968	2.72931e-08	\\
3.784	2.71515e-08	\\
3.79747	2.70721e-08	\\
3.81243	2.69831e-08	\\
3.82783	2.68907e-08	\\
3.84763	2.67706e-08	\\
3.85952	2.66979e-08	\\
3.87565	2.65984e-08	\\
3.90046	2.64438e-08	\\
3.91193	2.63717e-08	\\
3.93007	2.62568e-08	\\
3.9543	2.61019e-08	\\
3.97897	2.59425e-08	\\
3.99471	2.58399e-08	\\
4.01354	2.57163e-08	\\
4.02971	2.56094e-08	\\
4.0409	2.55351e-08	\\
4.05891	2.54149e-08	\\
4.07359	2.53164e-08	\\
4.08099	2.52665e-08	\\
4.10528	2.5102e-08	\\
4.1273	2.49518e-08	\\
4.14039	2.48621e-08	\\
4.1655	2.46891e-08	\\
4.18578	2.45485e-08	\\
4.21003	2.43794e-08	\\
4.23799	2.41833e-08	\\
4.25882	2.40365e-08	\\
4.27918	2.38923e-08	\\
4.2957	2.3775e-08	\\
4.31762	2.36186e-08	\\
4.33062	2.35257e-08	\\
4.35793	2.33298e-08	\\
4.37829	2.31831e-08	\\
4.4087	2.29634e-08	\\
4.42985	2.28101e-08	\\
4.44854	2.26743e-08	\\
4.46135	2.25811e-08	\\
4.48603	2.24011e-08	\\
4.51	2.2226e-08	\\
4.53141	2.20692e-08	\\
4.55237	2.19156e-08	\\
4.5749	2.17502e-08	\\
4.59724	2.1586e-08	\\
4.60872	2.15015e-08	\\
4.62886	2.13533e-08	\\
4.64609	2.12263e-08	\\
4.70087	2.08225e-08	\\
4.72547	2.0641e-08	\\
4.74041	2.05308e-08	\\
4.75	2.046e-08	\\
4.7787	2.02484e-08	\\
4.78755	2.01831e-08	\\
4.80747	2.00362e-08	\\
4.83792	1.98119e-08	\\
4.85137	1.97128e-08	\\
4.87628	1.95295e-08	\\
4.89058	1.94244e-08	\\
4.901	1.93478e-08	\\
4.91963	1.92111e-08	\\
4.94708	1.90097e-08	\\
4.96503	1.88783e-08	\\
4.98023	1.87671e-08	\\
5.00062	1.8618e-08	\\
5.03209	1.83885e-08	\\
5.04733	1.82776e-08	\\
5.05957	1.81886e-08	\\
5.089	1.79749e-08	\\
5.11273	1.78031e-08	\\
5.139	1.76134e-08	\\
5.17036	1.73876e-08	\\
5.20238	1.71578e-08	\\
5.22068	1.70268e-08	\\
5.24171	1.68768e-08	\\
5.27082	1.66696e-08	\\
5.29698	1.64842e-08	\\
5.3122	1.63766e-08	\\
5.33363	1.62255e-08	\\
5.35787	1.6055e-08	\\
5.38709	1.58505e-08	\\
5.41346	1.56666e-08	\\
5.4475	1.54304e-08	\\
5.48522	1.51702e-08	\\
5.50637	1.5025e-08	\\
5.5248	1.48989e-08	\\
5.55313	1.47059e-08	\\
5.57634	1.45485e-08	\\
5.60366	1.43641e-08	\\
5.62191	1.42414e-08	\\
5.65958	1.39896e-08	\\
5.69471	1.37563e-08	\\
5.71707	1.36088e-08	\\
5.73437	1.34951e-08	\\
5.75999	1.33274e-08	\\
5.78241	1.31815e-08	\\
5.80886	1.30101e-08	\\
5.82328	1.29172e-08	\\
5.85852	1.26912e-08	\\
5.88786	1.25044e-08	\\
5.917	1.23201e-08	\\
5.93246	1.22229e-08	\\
5.96588	1.20138e-08	\\
5.99341	1.18429e-08	\\
6.011	1.17342e-08	\\
6.03999	1.15562e-08	\\
6.06491	1.14043e-08	\\
6.09937	1.11957e-08	\\
6.12948	1.10149e-08	\\
6.16636	1.07955e-08	\\
6.19125	1.06485e-08	\\
6.21746	1.04949e-08	\\
6.24477	1.03361e-08	\\
6.26611	1.02127e-08	\\
6.28719	1.00916e-08	\\
6.3129	9.94485e-09	\\
6.34627	9.75594e-09	\\
6.37139	9.61498e-09	\\
6.39363	9.49103e-09	\\
6.41025	9.39889e-09	\\
6.43865	9.24257e-09	\\
6.47916	9.02191e-09	\\
6.5102	8.85465e-09	\\
6.52663	8.76676e-09	\\
6.57432	8.51421e-09	\\
6.61072	8.324e-09	\\
6.63516	8.19753e-09	\\
6.68209	7.95755e-09	\\
6.70497	7.84186e-09	\\
6.7332	7.70038e-09	\\
6.76038	7.56541e-09	\\
6.79327	7.40376e-09	\\
6.82495	7.24979e-09	\\
6.86837	7.04157e-09	\\
6.88989	6.93952e-09	\\
6.91276	6.83196e-09	\\
6.93727	6.71768e-09	\\
6.99222	6.46517e-09	\\
7.0143	6.36515e-09	\\
7.03642	6.26578e-09	\\
7.07477	6.09545e-09	\\
7.09945	5.9872e-09	\\
7.12214	5.88854e-09	\\
7.15202	5.76e-09	\\
7.18136	5.63526e-09	\\
7.22839	5.43832e-09	\\
7.29291	5.17424e-09	\\
7.32937	5.02816e-09	\\
7.34602	4.9622e-09	\\
7.39887	4.75586e-09	\\
7.42534	4.65431e-09	\\
7.45516	4.5413e-09	\\
7.48223	4.44001e-09	\\
7.51391	4.32303e-09	\\
7.56059	4.15373e-09	\\
7.6048	3.99676e-09	\\
7.62181	3.93721e-09	\\
7.64773	3.84743e-09	\\
7.69234	3.69547e-09	\\
7.73359	3.55791e-09	\\
7.77596	3.41954e-09	\\
7.82277	3.2701e-09	\\
7.86155	3.14903e-09	\\
7.89303	3.05254e-09	\\
7.93264	2.93342e-09	\\
7.97912	2.7969e-09	\\
8.00993	2.7083e-09	\\
8.05018	2.59491e-09	\\
8.09962	2.45914e-09	\\
8.13586	2.36214e-09	\\
8.17892	2.24956e-09	\\
8.2152	2.157e-09	\\
8.24147	2.09129e-09	\\
8.27699	2.00415e-09	\\
8.31588	1.91102e-09	\\
8.36978	1.78584e-09	\\
8.39427	1.73045e-09	\\
8.43448	1.64151e-09	\\
8.47906	1.54584e-09	\\
8.50451	1.49259e-09	\\
8.52412	1.45224e-09	\\
8.57725	1.34584e-09	\\
8.60789	1.28644e-09	\\
8.6428	1.22047e-09	\\
8.70304	1.11097e-09	\\
8.76292	1.00749e-09	\\
8.79102	9.60752e-10	\\
8.8193	9.14907e-10	\\
8.86518	8.43028e-10	\\
8.91599	7.67004e-10	\\
8.96533	6.96788e-10	\\
8.99896	6.50947e-10	\\
9.03091	6.08897e-10	\\
9.06498	5.65678e-10	\\
9.11212	5.08601e-10	\\
9.15748	4.56643e-10	\\
9.20151	4.08995e-10	\\
9.2463	3.63304e-10	\\
9.30386	3.08691e-10	\\
9.35036	2.67908e-10	\\
9.41429	2.16682e-10	\\
9.45702	1.85548e-10	\\
9.49771	1.58197e-10	\\
9.55248	1.2489e-10	\\
9.61597	9.12931e-11	\\
9.68052	6.25977e-11	\\
9.70768	5.21621e-11	\\
9.7569	3.56947e-11	\\
9.79248	2.57522e-11	\\
9.83031	1.69684e-11	\\
9.87647	8.73762e-12	\\
9.90852	4.62202e-12	\\
9.95018	1.21797e-12	\\
};
\end{axis}
\end{tikzpicture}%

        \end{subfigure}
        \hfill
        \begin{subfigure}[b]{0.45\textwidth}
            % % This file was created by matlab2tikz v0.4.6 running on MATLAB 8.1.
% Copyright (c) 2008--2014, Nico Schlömer <nico.schloemer@gmail.com>
% All rights reserved.
% Minimal pgfplots version: 1.3
%
% The latest updates can be retrieved from
%   http://www.mathworks.com/matlabcentral/fileexchange/22022-matlab2tikz
% where you can also make suggestions and rate matlab2tikz.
%
\begin{tikzpicture}

\begin{axis}[%
width=\figurewidth,
height=\figureheight,
scale only axis,
xmin=0,
xmax=10,
ymode=log,
ymin=1e-16,
ymax=1e-10,
yminorticks=false
]
\addplot [color=red,solid,forget plot]
  table[row sep=crcr]{
0.100016	4.67545e-15	\\
0.100732	5.94295e-15	\\
0.101143	1.07496e-14	\\
0.101868	1.76893e-14	\\
0.102291	2.15012e-14	\\
0.102765	2.68423e-14	\\
0.103121	3.3222e-14	\\
0.103483	4.2636e-14	\\
0.103876	5.05671e-14	\\
0.104476	5.98027e-14	\\
0.104665	6.72105e-14	\\
0.105362	8.14177e-14	\\
0.105744	9.00998e-14	\\
0.106062	1.01363e-13	\\
0.106298	1.09892e-13	\\
0.106648	1.15768e-13	\\
0.10715	1.33233e-13	\\
0.107607	1.43627e-13	\\
0.108043	1.56425e-13	\\
0.108317	1.65605e-13	\\
0.108894	1.83986e-13	\\
0.109224	1.93031e-13	\\
0.109604	2.03922e-13	\\
0.110172	2.24796e-13	\\
0.110505	2.32652e-13	\\
0.110927	2.46092e-13	\\
0.111227	2.54833e-13	\\
0.111839	2.72483e-13	\\
0.112341	2.87975e-13	\\
0.112781	3.0484e-13	\\
0.113469	3.22952e-13	\\
0.113904	3.35854e-13	\\
0.114661	3.56413e-13	\\
0.115014	3.68689e-13	\\
0.115804	3.89261e-13	\\
0.11616	3.99824e-13	\\
0.11676	4.16598e-13	\\
0.1172	4.27666e-13	\\
0.117616	4.39144e-13	\\
0.118117	4.53148e-13	\\
0.118705	4.66758e-13	\\
0.119312	4.81297e-13	\\
0.119771	4.92774e-13	\\
0.120376	5.04823e-13	\\
0.121135	5.22133e-13	\\
0.121529	5.28431e-13	\\
0.122052	5.38439e-13	\\
0.12263	5.51916e-13	\\
0.123087	5.58128e-13	\\
0.123626	5.67338e-13	\\
0.124085	5.74484e-13	\\
0.124676	5.84126e-13	\\
0.125112	5.89932e-13	\\
0.125592	5.99684e-13	\\
0.126054	6.02836e-13	\\
0.126291	6.07632e-13	\\
0.126904	6.14422e-13	\\
0.127492	6.20299e-13	\\
0.127792	6.23962e-13	\\
0.128534	6.32498e-13	\\
0.129167	6.3765e-13	\\
0.129774	6.41433e-13	\\
0.130294	6.45816e-13	\\
0.130818	6.46885e-13	\\
0.131487	6.52647e-13	\\
0.132056	6.55123e-13	\\
0.132644	6.55095e-13	\\
0.133001	6.57055e-13	\\
0.133666	6.61682e-13	\\
0.134361	6.61179e-13	\\
0.13507	6.62658e-13	\\
0.135588	6.63772e-13	\\
0.135991	6.62895e-13	\\
0.13657	6.6174e-13	\\
0.13714	6.61e-13	\\
0.13769	6.59108e-13	\\
0.138229	6.5848e-13	\\
0.138825	6.5491e-13	\\
0.139339	6.52273e-13	\\
0.140146	6.50525e-13	\\
0.140895	6.44481e-13	\\
0.141373	6.40403e-13	\\
0.141897	6.38709e-13	\\
0.142535	6.334e-13	\\
0.143589	6.25097e-13	\\
0.144197	6.18471e-13	\\
0.144743	6.14485e-13	\\
0.145248	6.10001e-13	\\
0.145703	6.05124e-13	\\
0.146385	5.98133e-13	\\
0.147035	5.91772e-13	\\
0.147801	5.82844e-13	\\
0.148257	5.77551e-13	\\
0.149005	5.69162e-13	\\
0.149751	5.6115e-13	\\
0.150789	5.47124e-13	\\
0.151369	5.39258e-13	\\
0.15236	5.28535e-13	\\
0.153206	5.14297e-13	\\
0.153923	5.04799e-13	\\
0.154616	4.95743e-13	\\
0.155702	4.81168e-13	\\
0.156276	4.72261e-13	\\
0.157109	4.60532e-13	\\
0.157913	4.49397e-13	\\
0.158792	4.34776e-13	\\
0.159332	4.284e-13	\\
0.160076	4.17049e-13	\\
0.160613	4.08386e-13	\\
0.161543	3.95282e-13	\\
0.161952	3.89517e-13	\\
0.162943	3.74285e-13	\\
0.163795	3.61564e-13	\\
0.164577	3.511e-13	\\
0.165563	3.35103e-13	\\
0.166526	3.2204e-13	\\
0.167275	3.11416e-13	\\
0.167903	3.0223e-13	\\
0.168356	2.95368e-13	\\
0.169313	2.81414e-13	\\
0.170372	2.66996e-13	\\
0.171347	2.53158e-13	\\
0.172157	2.43072e-13	\\
0.172836	2.33934e-13	\\
0.173425	2.25358e-13	\\
0.173801	2.20681e-13	\\
0.174706	2.0885e-13	\\
0.175346	2.01965e-13	\\
0.176423	1.88108e-13	\\
0.177117	1.80308e-13	\\
0.178062	1.68179e-13	\\
0.178898	1.59366e-13	\\
0.180102	1.46319e-13	\\
0.180862	1.36455e-13	\\
0.182027	1.25981e-13	\\
0.184089	1.05644e-13	\\
0.18487	9.73721e-14	\\
0.185944	8.81327e-14	\\
0.18661	8.2511e-14	\\
0.187504	7.43513e-14	\\
0.188082	6.93894e-14	\\
0.188911	6.44978e-14	\\
0.189884	5.69439e-14	\\
0.190693	5.21482e-14	\\
0.191648	4.51869e-14	\\
0.19297	3.80105e-14	\\
0.194005	3.13723e-14	\\
0.195319	2.50128e-14	\\
0.19641	1.99131e-14	\\
0.197237	1.73386e-14	\\
0.198188	1.41937e-14	\\
0.198714	1.2028e-14	\\
0.199929	8.71167e-15	\\
0.200903	6.61217e-15	\\
0.201795	5.13358e-15	\\
0.202715	3.726e-15	\\
0.203536	2.95087e-15	\\
0.204572	2.43369e-15	\\
0.205426	1.69835e-15	\\
0.207216	3.4049e-15	\\
0.207983	3.20214e-15	\\
0.208602	3.68844e-15	\\
0.209736	5.58607e-15	\\
0.210653	8.35716e-15	\\
0.211348	1.06617e-14	\\
0.212512	1.30923e-14	\\
0.213482	1.67947e-14	\\
0.214593	2.11499e-14	\\
0.215334	2.4301e-14	\\
0.216055	2.63897e-14	\\
0.217124	3.26174e-14	\\
0.21783	3.61923e-14	\\
0.219062	4.22746e-14	\\
0.219841	4.7711e-14	\\
0.220808	5.38945e-14	\\
0.222028	6.21289e-14	\\
0.222801	6.73937e-14	\\
0.223564	7.37155e-14	\\
0.224473	8.04133e-14	\\
0.225435	8.74724e-14	\\
0.226929	1.00306e-13	\\
0.228105	1.12159e-13	\\
0.229383	1.23438e-13	\\
0.230274	1.32614e-13	\\
0.231502	1.44357e-13	\\
0.23291	1.59562e-13	\\
0.233592	1.67019e-13	\\
0.234564	1.78057e-13	\\
0.235642	1.90115e-13	\\
0.237129	2.07466e-13	\\
0.238486	2.23816e-13	\\
0.239473	2.36173e-13	\\
0.240467	2.50084e-13	\\
0.241136	2.58049e-13	\\
0.242537	2.76509e-13	\\
0.244035	2.97247e-13	\\
0.245076	3.12124e-13	\\
0.246009	3.26084e-13	\\
0.247	3.40119e-13	\\
0.247885	3.53669e-13	\\
0.249225	3.73318e-13	\\
0.250428	3.91741e-13	\\
0.251459	4.07496e-13	\\
0.252329	4.2123e-13	\\
0.253517	4.40255e-13	\\
0.254847	4.61354e-13	\\
0.255904	4.7913e-13	\\
0.257098	4.99421e-13	\\
0.258282	5.18344e-13	\\
0.259441	5.38565e-13	\\
0.260833	5.62656e-13	\\
0.262131	5.849e-13	\\
0.262987	6.0038e-13	\\
0.263852	6.15511e-13	\\
0.264951	6.35016e-13	\\
0.266916	6.70772e-13	\\
0.268033	6.90785e-13	\\
0.269649	7.21075e-13	\\
0.271025	7.45947e-13	\\
0.272281	7.70129e-13	\\
0.274193	8.05828e-13	\\
0.275028	8.21415e-13	\\
0.276161	8.43775e-13	\\
0.27739	8.66973e-13	\\
0.278326	8.84494e-13	\\
0.279306	9.03617e-13	\\
0.280599	9.284e-13	\\
0.281796	9.51427e-13	\\
0.282404	9.63054e-13	\\
0.282916	9.72942e-13	\\
0.284345	1.0008e-12	\\
0.286539	1.04418e-12	\\
0.288465	1.08195e-12	\\
0.289577	1.10312e-12	\\
0.290789	1.12713e-12	\\
0.291891	1.14902e-12	\\
0.292729	1.1655e-12	\\
0.293549	1.18199e-12	\\
0.294692	1.20354e-12	\\
0.29527	1.21517e-12	\\
0.296798	1.24543e-12	\\
0.298342	1.27543e-12	\\
0.299756	1.30286e-12	\\
0.301579	1.33904e-12	\\
0.302548	1.35762e-12	\\
0.304394	1.39399e-12	\\
0.306557	1.43614e-12	\\
0.308453	1.47265e-12	\\
0.310287	1.508e-12	\\
0.311845	1.53757e-12	\\
0.313014	1.55958e-12	\\
0.314818	1.59356e-12	\\
0.316266	1.62158e-12	\\
0.317626	1.64687e-12	\\
0.319309	1.67874e-12	\\
0.320928	1.70854e-12	\\
0.322521	1.73771e-12	\\
0.323438	1.75441e-12	\\
0.324975	1.78255e-12	\\
0.32696	1.8186e-12	\\
0.328393	1.84411e-12	\\
0.329228	1.85956e-12	\\
0.330537	1.88254e-12	\\
0.331616	1.90146e-12	\\
0.332773	1.9216e-12	\\
0.334735	1.956e-12	\\
0.335838	1.97434e-12	\\
0.336532	1.98633e-12	\\
0.338708	2.02273e-12	\\
0.339764	2.04089e-12	\\
0.340906	2.05983e-12	\\
0.342263	2.08218e-12	\\
0.344496	2.11795e-12	\\
0.346563	2.15112e-12	\\
0.348333	2.17901e-12	\\
0.349845	2.20316e-12	\\
0.352231	2.23954e-12	\\
0.353715	2.26181e-12	\\
0.355412	2.28704e-12	\\
0.35655	2.30347e-12	\\
0.358077	2.32607e-12	\\
0.360099	2.35455e-12	\\
0.361818	2.37879e-12	\\
0.364139	2.41089e-12	\\
0.36546	2.42857e-12	\\
0.367669	2.45787e-12	\\
0.369975	2.48816e-12	\\
0.371557	2.50785e-12	\\
0.372694	2.52207e-12	\\
0.373774	2.53543e-12	\\
0.375262	2.5535e-12	\\
0.377306	2.57764e-12	\\
0.380459	2.61423e-12	\\
0.382933	2.64187e-12	\\
0.384678	2.66082e-12	\\
0.385902	2.67412e-12	\\
0.388727	2.70316e-12	\\
0.391005	2.72615e-12	\\
0.393414	2.74907e-12	\\
0.395144	2.76543e-12	\\
0.396611	2.77896e-12	\\
0.398773	2.79795e-12	\\
0.400663	2.81439e-12	\\
0.403071	2.83414e-12	\\
0.404874	2.84864e-12	\\
0.406304	2.8597e-12	\\
0.407502	2.86863e-12	\\
0.408969	2.87963e-12	\\
0.410986	2.89409e-12	\\
0.413545	2.91174e-12	\\
0.415279	2.92322e-12	\\
0.418128	2.94084e-12	\\
0.420064	2.95228e-12	\\
0.421823	2.96208e-12	\\
0.424593	2.97683e-12	\\
0.426398	2.98581e-12	\\
0.428145	2.99432e-12	\\
0.430117	3.00332e-12	\\
0.431719	3.00976e-12	\\
0.434004	3.01933e-12	\\
0.435639	3.02554e-12	\\
0.438002	3.03378e-12	\\
0.44003	3.04072e-12	\\
0.441192	3.04421e-12	\\
0.444417	3.05298e-12	\\
0.446955	3.05939e-12	\\
0.44909	3.06343e-12	\\
0.451835	3.06901e-12	\\
0.453309	3.07125e-12	\\
0.454813	3.07344e-12	\\
0.457958	3.07695e-12	\\
0.459862	3.07859e-12	\\
0.462247	3.08006e-12	\\
0.464619	3.08098e-12	\\
0.465952	3.08105e-12	\\
0.467532	3.0813e-12	\\
0.468681	3.08083e-12	\\
0.471989	3.07943e-12	\\
0.47471	3.07741e-12	\\
0.476291	3.07577e-12	\\
0.477826	3.07421e-12	\\
0.479499	3.07205e-12	\\
0.482866	3.06688e-12	\\
0.485005	3.06308e-12	\\
0.486881	3.05946e-12	\\
0.489435	3.05349e-12	\\
0.491926	3.0474e-12	\\
0.495021	3.03907e-12	\\
0.498525	3.02852e-12	\\
0.50063	3.02175e-12	\\
0.504166	3.00908e-12	\\
0.505665	3.00398e-12	\\
0.507841	2.99537e-12	\\
0.510741	2.98377e-12	\\
0.512662	2.97555e-12	\\
0.514519	2.9676e-12	\\
0.517737	2.95259e-12	\\
0.520304	2.94029e-12	\\
0.522653	2.92882e-12	\\
0.525222	2.91526e-12	\\
0.527856	2.90138e-12	\\
0.529053	2.89484e-12	\\
0.530345	2.88769e-12	\\
0.532671	2.87474e-12	\\
0.537014	2.8491e-12	\\
0.539905	2.83154e-12	\\
0.542539	2.8151e-12	\\
0.545665	2.79468e-12	\\
0.548282	2.77768e-12	\\
0.55136	2.75678e-12	\\
0.553736	2.74036e-12	\\
0.557423	2.71453e-12	\\
0.559872	2.69681e-12	\\
0.563057	2.67348e-12	\\
0.566709	2.64633e-12	\\
0.568692	2.6312e-12	\\
0.570944	2.61385e-12	\\
0.572341	2.60333e-12	\\
0.576713	2.56869e-12	\\
0.578029	2.55828e-12	\\
0.58042	2.53911e-12	\\
0.583278	2.51574e-12	\\
0.58787	2.4781e-12	\\
0.591899	2.44446e-12	\\
0.596812	2.40247e-12	\\
0.598832	2.38511e-12	\\
0.602077	2.357e-12	\\
0.603788	2.34218e-12	\\
0.605351	2.3284e-12	\\
0.609567	2.29115e-12	\\
0.612397	2.26597e-12	\\
0.613894	2.25276e-12	\\
0.617595	2.21946e-12	\\
0.618871	2.20804e-12	\\
0.621349	2.18567e-12	\\
0.623221	2.16854e-12	\\
0.625753	2.14568e-12	\\
0.630458	2.10262e-12	\\
0.631941	2.08908e-12	\\
0.636069	2.05099e-12	\\
0.638582	2.02782e-12	\\
0.642931	1.98762e-12	\\
0.646184	1.95743e-12	\\
0.649208	1.92946e-12	\\
0.651934	1.90398e-12	\\
0.655829	1.86784e-12	\\
0.659188	1.83657e-12	\\
0.661905	1.81135e-12	\\
0.664688	1.78558e-12	\\
0.668699	1.74866e-12	\\
0.673064	1.7081e-12	\\
0.675007	1.69037e-12	\\
0.677923	1.66328e-12	\\
0.680482	1.63976e-12	\\
0.683106	1.61582e-12	\\
0.686506	1.58468e-12	\\
0.68944	1.5578e-12	\\
0.695155	1.50592e-12	\\
0.699714	1.46491e-12	\\
0.701674	1.44743e-12	\\
0.705204	1.41574e-12	\\
0.708283	1.38845e-12	\\
0.710861	1.36554e-12	\\
0.713441	1.34301e-12	\\
0.716795	1.3136e-12	\\
0.718886	1.29539e-12	\\
0.722192	1.26681e-12	\\
0.725998	1.23409e-12	\\
0.730187	1.19848e-12	\\
0.73199	1.1833e-12	\\
0.736732	1.14375e-12	\\
0.741189	1.10668e-12	\\
0.743401	1.08867e-12	\\
0.74612	1.06635e-12	\\
0.750475	1.03137e-12	\\
0.753352	1.00848e-12	\\
0.755676	9.89931e-13	\\
0.757967	9.71945e-13	\\
0.762502	9.36736e-13	\\
0.766831	9.03628e-13	\\
0.771437	8.6896e-13	\\
0.77395	8.50347e-13	\\
0.776972	8.27997e-13	\\
0.781398	7.95886e-13	\\
0.784609	7.73047e-13	\\
0.787748	7.50967e-13	\\
0.791038	7.27887e-13	\\
0.794143	7.06531e-13	\\
0.79783	6.81572e-13	\\
0.802238	6.5233e-13	\\
0.806834	6.22457e-13	\\
0.809551	6.04989e-13	\\
0.812533	5.86315e-13	\\
0.81617	5.63667e-13	\\
0.820285	5.38612e-13	\\
0.824309	5.14777e-13	\\
0.826707	5.00663e-13	\\
0.833319	4.62875e-13	\\
0.835586	4.50316e-13	\\
0.838401	4.34886e-13	\\
0.842981	4.1049e-13	\\
0.847488	3.86959e-13	\\
0.851709	3.65699e-13	\\
0.855601	3.46633e-13	\\
0.860051	3.25252e-13	\\
0.864038	3.06714e-13	\\
0.867779	2.9e-13	\\
0.870037	2.80162e-13	\\
0.87366	2.64616e-13	\\
0.880088	2.38113e-13	\\
0.884857	2.19335e-13	\\
0.889037	2.03494e-13	\\
0.893547	1.87009e-13	\\
0.896011	1.78444e-13	\\
0.900526	1.6302e-13	\\
0.903659	1.52793e-13	\\
0.907433	1.41049e-13	\\
0.912219	1.26672e-13	\\
0.913714	1.22282e-13	\\
0.915829	1.16394e-13	\\
0.918211	1.09809e-13	\\
0.922562	9.82727e-14	\\
0.926574	8.8116e-14	\\
0.93001	8.01662e-14	\\
0.939074	6.0514e-14	\\
0.94474	4.98352e-14	\\
0.947393	4.51237e-14	\\
0.951158	3.89369e-14	\\
0.957049	3.00378e-14	\\
0.963964	2.10808e-14	\\
0.967383	1.73381e-14	\\
0.969213	1.54061e-14	\\
0.974442	1.06536e-14	\\
0.979112	7.16658e-15	\\
0.985362	3.66493e-15	\\
0.987692	2.60528e-15	\\
0.991722	1.3391e-15	\\
0.996869	2.66139e-16	\\
1.00069	2.85312e-16	\\
1.00686	9.71098e-16	\\
1.0114	2.34828e-15	\\
1.01564	4.21185e-15	\\
1.01883	5.91706e-15	\\
1.02234	8.23732e-15	\\
1.02492	1.02278e-14	\\
1.02894	1.36408e-14	\\
1.03158	1.61952e-14	\\
1.0349	1.98264e-14	\\
1.04362	3.08964e-14	\\
1.04669	3.54754e-14	\\
1.05053	4.15258e-14	\\
1.05595	5.09472e-14	\\
1.06253	6.37083e-14	\\
1.06742	7.41782e-14	\\
1.07338	8.80661e-14	\\
1.07781	9.9194e-14	\\
1.08253	1.11837e-13	\\
1.08601	1.21502e-13	\\
1.09	1.33306e-13	\\
1.09294	1.42208e-13	\\
1.09738	1.56306e-13	\\
1.102	1.71657e-13	\\
1.10895	1.96158e-13	\\
1.11633	2.24112e-13	\\
1.12186	2.46203e-13	\\
1.12619	2.6432e-13	\\
1.1329	2.93526e-13	\\
1.13615	3.08231e-13	\\
1.14014	3.26887e-13	\\
1.14299	3.40529e-13	\\
1.14711	3.60698e-13	\\
1.15176	3.84244e-13	\\
1.15845	4.19255e-13	\\
1.16166	4.36762e-13	\\
1.16882	4.76871e-13	\\
1.1752	5.14241e-13	\\
1.17957	5.4047e-13	\\
1.18216	5.56415e-13	\\
1.19114	6.13326e-13	\\
1.19497	6.38541e-13	\\
1.20001	6.72299e-13	\\
1.20526	7.08453e-13	\\
1.21056	7.45952e-13	\\
1.21672	7.90698e-13	\\
1.22073	8.2048e-13	\\
1.23028	8.93698e-13	\\
1.23784	9.53753e-13	\\
1.24459	1.00916e-12	\\
1.24788	1.03662e-12	\\
1.25128	1.06539e-12	\\
1.25545	1.10139e-12	\\
1.25793	1.12311e-12	\\
1.26242	1.1626e-12	\\
1.26925	1.22413e-12	\\
1.27407	1.26849e-12	\\
1.27764	1.30184e-12	\\
1.28445	1.36652e-12	\\
1.28897	1.41031e-12	\\
1.29525	1.47223e-12	\\
1.3049	1.5698e-12	\\
1.30871	1.60916e-12	\\
1.31864	1.71381e-12	\\
1.32815	1.81705e-12	\\
1.34101	1.96089e-12	\\
1.3439	1.99395e-12	\\
1.34954	2.05918e-12	\\
1.35348	2.10524e-12	\\
1.35862	2.1662e-12	\\
1.36647	2.26068e-12	\\
1.37219	2.33062e-12	\\
1.3777	2.39903e-12	\\
1.38335	2.46993e-12	\\
1.38883	2.53961e-12	\\
1.39808	2.65911e-12	\\
1.40462	2.74515e-12	\\
1.40889	2.80204e-12	\\
1.41678	2.90822e-12	\\
1.42489	3.01925e-12	\\
1.4286	3.07065e-12	\\
1.43541	3.16598e-12	\\
1.44356	3.28165e-12	\\
1.447	3.33096e-12	\\
1.45273	3.41388e-12	\\
1.45783	3.48843e-12	\\
1.46261	3.55894e-12	\\
1.46644	3.61568e-12	\\
1.47306	3.71473e-12	\\
1.48112	3.837e-12	\\
1.48631	3.91643e-12	\\
1.49152	3.99705e-12	\\
1.498	4.09802e-12	\\
1.50889	4.27002e-12	\\
1.51635	4.38953e-12	\\
1.52411	4.51512e-12	\\
1.53298	4.66037e-12	\\
1.53867	4.7544e-12	\\
1.54339	4.83301e-12	\\
1.55362	5.00518e-12	\\
1.56628	5.22118e-12	\\
1.57436	5.36081e-12	\\
1.58248	5.50246e-12	\\
1.58694	5.58086e-12	\\
1.59426	5.71031e-12	\\
1.6016	5.84118e-12	\\
1.60963	5.98567e-12	\\
1.61845	6.14558e-12	\\
1.62336	6.23524e-12	\\
1.63086	6.37314e-12	\\
1.63692	6.48524e-12	\\
1.64619	6.65804e-12	\\
1.65239	6.77438e-12	\\
1.65937	6.90613e-12	\\
1.665	7.01303e-12	\\
1.67189	7.14461e-12	\\
1.67556	7.21505e-12	\\
1.68271	7.35292e-12	\\
1.68987	7.49157e-12	\\
1.69877	7.66513e-12	\\
1.70866	7.85938e-12	\\
1.71442	7.97314e-12	\\
1.72011	8.08613e-12	\\
1.72704	8.22419e-12	\\
1.73769	8.43756e-12	\\
1.74769	8.63953e-12	\\
1.75454	8.77849e-12	\\
1.76234	8.93744e-12	\\
1.77118	9.11848e-12	\\
1.78279	9.35796e-12	\\
1.79008	9.50892e-12	\\
1.80116	9.73961e-12	\\
1.80927	9.90939e-12	\\
1.81677	1.00669e-11	\\
1.8269	1.02806e-11	\\
1.83609	1.04751e-11	\\
1.84246	1.06106e-11	\\
1.84909	1.07519e-11	\\
1.86067	1.09996e-11	\\
1.86797	1.11565e-11	\\
1.87509	1.13099e-11	\\
1.88462	1.15157e-11	\\
1.89612	1.1765e-11	\\
1.90382	1.19323e-11	\\
1.91835	1.22495e-11	\\
1.92937	1.24909e-11	\\
1.94517	1.28382e-11	\\
1.95358	1.30237e-11	\\
1.96322	1.32367e-11	\\
1.97067	1.34015e-11	\\
1.97847	1.35744e-11	\\
1.99118	1.38566e-11	\\
1.99933	1.4038e-11	\\
2.00577	1.41813e-11	\\
2.01732	1.44389e-11	\\
2.02421	1.45928e-11	\\
2.03162	1.47584e-11	\\
2.0416	1.49817e-11	\\
2.05159	1.52054e-11	\\
2.06066	1.54088e-11	\\
2.07203	1.56638e-11	\\
2.07753	1.57872e-11	\\
2.08707	1.60012e-11	\\
2.09452	1.61687e-11	\\
2.10783	1.64674e-11	\\
2.1198	1.67364e-11	\\
2.12924	1.69483e-11	\\
2.14501	1.73025e-11	\\
2.15928	1.76227e-11	\\
2.17265	1.79225e-11	\\
2.19081	1.83295e-11	\\
2.2014	1.85666e-11	\\
2.21006	1.87601e-11	\\
2.21576	1.88874e-11	\\
2.22677	1.91332e-11	\\
2.23651	1.93503e-11	\\
2.25202	1.96952e-11	\\
2.26277	1.99338e-11	\\
2.27758	2.0262e-11	\\
2.2885	2.05034e-11	\\
2.2988	2.07307e-11	\\
2.30861	2.09465e-11	\\
2.31522	2.1092e-11	\\
2.32688	2.13476e-11	\\
2.34351	2.17113e-11	\\
2.35324	2.19232e-11	\\
2.36576	2.21954e-11	\\
2.36978	2.22826e-11	\\
2.3798	2.24995e-11	\\
2.39165	2.27552e-11	\\
2.40418	2.30245e-11	\\
2.41219	2.31962e-11	\\
2.42786	2.35309e-11	\\
2.43743	2.37343e-11	\\
2.44472	2.3889e-11	\\
2.45551	2.41172e-11	\\
2.46508	2.43188e-11	\\
2.47591	2.45463e-11	\\
2.49047	2.48507e-11	\\
2.50494	2.51515e-11	\\
2.51629	2.53864e-11	\\
2.53182	2.57057e-11	\\
2.54063	2.58861e-11	\\
2.55441	2.61669e-11	\\
2.56632	2.64082e-11	\\
2.57583	2.66001e-11	\\
2.59765	2.70369e-11	\\
2.61636	2.74083e-11	\\
2.62939	2.76647e-11	\\
2.64229	2.79173e-11	\\
2.65063	2.80797e-11	\\
2.66488	2.83552e-11	\\
2.67204	2.84932e-11	\\
2.68853	2.88084e-11	\\
2.70159	2.90565e-11	\\
2.71485	2.9306e-11	\\
2.72522	2.95003e-11	\\
2.73163	2.96196e-11	\\
2.74114	2.97959e-11	\\
2.75287	3.00121e-11	\\
2.75762	3.00992e-11	\\
2.76946	3.03152e-11	\\
2.79324	3.07446e-11	\\
2.80245	3.09093e-11	\\
2.82032	3.1226e-11	\\
2.83198	3.14308e-11	\\
2.84039	3.15775e-11	\\
2.85568	3.18421e-11	\\
2.87357	3.21481e-11	\\
2.88426	3.23292e-11	\\
2.90311	3.26455e-11	\\
2.91433	3.28318e-11	\\
2.91981	3.29221e-11	\\
2.93285	3.31359e-11	\\
2.94845	3.33889e-11	\\
2.95951	3.35665e-11	\\
2.97174	3.3761e-11	\\
2.98385	3.39523e-11	\\
3.00019	3.42071e-11	\\
3.00996	3.43579e-11	\\
3.02438	3.45784e-11	\\
3.03676	3.47656e-11	\\
3.05065	3.49733e-11	\\
3.06015	3.51143e-11	\\
3.07312	3.53047e-11	\\
3.09063	3.55584e-11	\\
3.10426	3.57532e-11	\\
3.11856	3.59551e-11	\\
3.13097	3.61281e-11	\\
3.14182	3.6278e-11	\\
3.15581	3.6469e-11	\\
3.17855	3.6774e-11	\\
3.1944	3.69827e-11	\\
3.20509	3.71219e-11	\\
3.21843	3.72932e-11	\\
3.23677	3.75251e-11	\\
3.25536	3.77557e-11	\\
3.27116	3.79485e-11	\\
3.28145	3.80724e-11	\\
3.28776	3.81476e-11	\\
3.29755	3.82633e-11	\\
3.31668	3.84858e-11	\\
3.33641	3.87105e-11	\\
3.35121	3.88757e-11	\\
3.36002	3.89728e-11	\\
3.37035	3.90852e-11	\\
3.38202	3.92107e-11	\\
3.40333	3.94354e-11	\\
3.41024	3.95071e-11	\\
3.42871	3.96954e-11	\\
3.4392	3.98006e-11	\\
3.45	3.99072e-11	\\
3.47698	4.01675e-11	\\
3.48591	4.02515e-11	\\
3.49916	4.03743e-11	\\
3.51073	4.04799e-11	\\
3.53144	4.06647e-11	\\
3.54278	4.07635e-11	\\
3.55771	4.08911e-11	\\
3.56828	4.09797e-11	\\
3.59019	4.11591e-11	\\
3.61177	4.13301e-11	\\
3.62231	4.14113e-11	\\
3.63224	4.14868e-11	\\
3.64821	4.16054e-11	\\
3.67865	4.18229e-11	\\
3.6901	4.19018e-11	\\
3.70816	4.20229e-11	\\
3.72825	4.2153e-11	\\
3.74526	4.22595e-11	\\
3.75968	4.23468e-11	\\
3.784	4.24888e-11	\\
3.79747	4.25642e-11	\\
3.81243	4.26455e-11	\\
3.82783	4.27264e-11	\\
3.84763	4.28263e-11	\\
3.85952	4.2884e-11	\\
3.87565	4.29597e-11	\\
3.90046	4.30702e-11	\\
3.91193	4.31191e-11	\\
3.93007	4.3193e-11	\\
3.9543	4.3286e-11	\\
3.97897	4.33736e-11	\\
3.99471	4.34261e-11	\\
4.01354	4.34851e-11	\\
4.02971	4.35327e-11	\\
4.0409	4.35637e-11	\\
4.05891	4.3611e-11	\\
4.07359	4.36469e-11	\\
4.08099	4.36642e-11	\\
4.10528	4.37163e-11	\\
4.1273	4.37584e-11	\\
4.14039	4.37807e-11	\\
4.1655	4.38187e-11	\\
4.18578	4.38444e-11	\\
4.21003	4.38696e-11	\\
4.23799	4.38912e-11	\\
4.25882	4.39019e-11	\\
4.27918	4.39081e-11	\\
4.2957	4.39102e-11	\\
4.31762	4.39087e-11	\\
4.33062	4.39054e-11	\\
4.35793	4.38935e-11	\\
4.37829	4.38797e-11	\\
4.4087	4.38518e-11	\\
4.42985	4.38272e-11	\\
4.44854	4.38018e-11	\\
4.46135	4.37824e-11	\\
4.48603	4.37412e-11	\\
4.51	4.36957e-11	\\
4.53141	4.36506e-11	\\
4.55237	4.36026e-11	\\
4.5749	4.35467e-11	\\
4.59724	4.34868e-11	\\
4.60872	4.34543e-11	\\
4.62886	4.33944e-11	\\
4.64609	4.33407e-11	\\
4.70087	4.31529e-11	\\
4.72547	4.30604e-11	\\
4.74041	4.30019e-11	\\
4.75	4.29634e-11	\\
4.7787	4.28439e-11	\\
4.78755	4.28057e-11	\\
4.80747	4.27175e-11	\\
4.83792	4.25769e-11	\\
4.85137	4.25124e-11	\\
4.87628	4.23896e-11	\\
4.89058	4.23173e-11	\\
4.901	4.22633e-11	\\
4.91963	4.21649e-11	\\
4.94708	4.20157e-11	\\
4.96503	4.1915e-11	\\
4.98023	4.18283e-11	\\
5.00062	4.1709e-11	\\
5.03209	4.15194e-11	\\
5.04733	4.14255e-11	\\
5.05957	4.13488e-11	\\
5.089	4.11601e-11	\\
5.11273	4.10042e-11	\\
5.139	4.08273e-11	\\
5.17036	4.06103e-11	\\
5.20238	4.03828e-11	\\
5.22068	4.02498e-11	\\
5.24171	4.00948e-11	\\
5.27082	3.98759e-11	\\
5.29698	3.96751e-11	\\
5.3122	3.95565e-11	\\
5.33363	3.93876e-11	\\
5.35787	3.91934e-11	\\
5.38709	3.89549e-11	\\
5.41346	3.87361e-11	\\
5.4475	3.84485e-11	\\
5.48522	3.8123e-11	\\
5.50637	3.79375e-11	\\
5.5248	3.77743e-11	\\
5.55313	3.75202e-11	\\
5.57634	3.73093e-11	\\
5.60366	3.70583e-11	\\
5.62191	3.68885e-11	\\
5.65958	3.65342e-11	\\
5.69471	3.61986e-11	\\
5.71707	3.59824e-11	\\
5.73437	3.58138e-11	\\
5.75999	3.55622e-11	\\
5.78241	3.53402e-11	\\
5.80886	3.50757e-11	\\
5.82328	3.49309e-11	\\
5.85852	3.45729e-11	\\
5.88786	3.42719e-11	\\
5.917	3.39703e-11	\\
5.93246	3.38094e-11	\\
5.96588	3.34588e-11	\\
5.99341	3.31678e-11	\\
6.011	3.29805e-11	\\
6.03999	3.26703e-11	\\
6.06491	3.24019e-11	\\
6.09937	3.20285e-11	\\
6.12948	3.16998e-11	\\
6.16636	3.12943e-11	\\
6.19125	3.1019e-11	\\
6.21746	3.07278e-11	\\
6.24477	3.0423e-11	\\
6.26611	3.01839e-11	\\
6.28719	2.99471e-11	\\
6.3129	2.9657e-11	\\
6.34627	2.92788e-11	\\
6.37139	2.89932e-11	\\
6.39363	2.87397e-11	\\
6.41025	2.85497e-11	\\
6.43865	2.82241e-11	\\
6.47916	2.77584e-11	\\
6.5102	2.74002e-11	\\
6.52663	2.72103e-11	\\
6.57432	2.66579e-11	\\
6.61072	2.62352e-11	\\
6.63516	2.59504e-11	\\
6.68209	2.54037e-11	\\
6.70497	2.51367e-11	\\
6.7332	2.48074e-11	\\
6.76038	2.44899e-11	\\
6.79327	2.41055e-11	\\
6.82495	2.37356e-11	\\
6.86837	2.3228e-11	\\
6.88989	2.29768e-11	\\
6.91276	2.27098e-11	\\
6.93727	2.24242e-11	\\
6.99222	2.17842e-11	\\
7.0143	2.15275e-11	\\
7.03642	2.12705e-11	\\
7.07477	2.08259e-11	\\
7.09945	2.05404e-11	\\
7.12214	2.02784e-11	\\
7.15202	1.99341e-11	\\
7.18136	1.95969e-11	\\
7.22839	1.90586e-11	\\
7.29291	1.83242e-11	\\
7.32937	1.79115e-11	\\
7.34602	1.7724e-11	\\
7.39887	1.71307e-11	\\
7.42534	1.68359e-11	\\
7.45516	1.65046e-11	\\
7.48223	1.62053e-11	\\
7.51391	1.58566e-11	\\
7.56059	1.53467e-11	\\
7.6048	1.4868e-11	\\
7.62181	1.46849e-11	\\
7.64773	1.44071e-11	\\
7.69234	1.39325e-11	\\
7.73359	1.34981e-11	\\
7.77596	1.30563e-11	\\
7.82277	1.25736e-11	\\
7.86155	1.21781e-11	\\
7.89303	1.186e-11	\\
7.93264	1.14639e-11	\\
7.97912	1.10049e-11	\\
8.00993	1.07044e-11	\\
8.05018	1.03161e-11	\\
8.09962	9.846e-12	\\
8.13586	9.50665e-12	\\
8.17892	9.10944e-12	\\
8.2152	8.77953e-12	\\
8.24147	8.54343e-12	\\
8.27699	8.22844e-12	\\
8.31588	7.88882e-12	\\
8.36978	7.42763e-12	\\
8.39427	7.2214e-12	\\
8.43448	6.88823e-12	\\
8.47906	6.52623e-12	\\
8.50451	6.32335e-12	\\
8.52412	6.16899e-12	\\
8.57725	5.75785e-12	\\
8.60789	5.52642e-12	\\
8.6428	5.26758e-12	\\
8.70304	4.83338e-12	\\
8.76292	4.41768e-12	\\
8.79102	4.22852e-12	\\
8.8193	4.04149e-12	\\
8.86518	3.74612e-12	\\
8.91599	3.43093e-12	\\
8.96533	3.13638e-12	\\
8.99896	2.94268e-12	\\
9.03091	2.76377e-12	\\
9.06498	2.57851e-12	\\
9.11212	2.33211e-12	\\
9.15748	2.10548e-12	\\
9.20151	1.89641e-12	\\
9.2463	1.69383e-12	\\
9.30386	1.44889e-12	\\
9.35036	1.26486e-12	\\
9.41429	1.03122e-12	\\
9.45702	8.8736e-13	\\
9.49771	7.599e-13	\\
9.55248	6.03866e-13	\\
9.61597	4.44516e-13	\\
9.68052	3.0675e-13	\\
9.70768	2.5644e-13	\\
9.7569	1.75894e-13	\\
9.79248	1.27377e-13	\\
9.83031	8.38534e-14	\\
9.87647	4.3093e-14	\\
9.90852	2.26895e-14	\\
9.95018	5.28578e-15	\\
};
\addplot [color=blue,dashed,forget plot]
  table[row sep=crcr]{
0.100016	0	\\
0.100732	6.66134e-15	\\
0.101143	6.21725e-15	\\
0.101868	4.88498e-15	\\
0.102291	6.66134e-15	\\
0.102765	5.77316e-15	\\
0.103121	1.33227e-15	\\
0.103483	8.43769e-15	\\
0.103876	1.33227e-14	\\
0.104476	8.43769e-15	\\
0.104665	1.33227e-14	\\
0.105362	1.33227e-14	\\
0.105744	9.76996e-15	\\
0.106062	2.08722e-14	\\
0.106298	1.59872e-14	\\
0.106648	2.84217e-14	\\
0.10715	2.35367e-14	\\
0.107607	1.73195e-14	\\
0.108043	2.66454e-14	\\
0.108317	2.22045e-14	\\
0.108894	3.37508e-14	\\
0.109224	2.93099e-14	\\
0.109604	3.59712e-14	\\
0.110172	3.90799e-14	\\
0.110505	3.90799e-14	\\
0.110927	3.37508e-14	\\
0.111227	5.55112e-14	\\
0.111839	5.63993e-14	\\
0.112341	5.32907e-14	\\
0.112781	4.08562e-14	\\
0.113469	7.72715e-14	\\
0.113904	7.06102e-14	\\
0.114661	7.68274e-14	\\
0.115014	6.79456e-14	\\
0.115804	8.43769e-14	\\
0.11616	7.81597e-14	\\
0.11676	8.52651e-14	\\
0.1172	8.4821e-14	\\
0.117616	8.52651e-14	\\
0.118117	8.52651e-14	\\
0.118705	9.28146e-14	\\
0.119312	9.99201e-14	\\
0.119771	1.01252e-13	\\
0.120376	1.04361e-13	\\
0.121135	1.15463e-13	\\
0.121529	1.00364e-13	\\
0.122052	1.13687e-13	\\
0.12263	1.0747e-13	\\
0.123087	1.22125e-13	\\
0.123626	1.17684e-13	\\
0.124085	1.17684e-13	\\
0.124676	1.35003e-13	\\
0.125112	1.20792e-13	\\
0.125592	1.35003e-13	\\
0.126054	1.27898e-13	\\
0.126291	1.35891e-13	\\
0.126904	1.28342e-13	\\
0.127492	1.42997e-13	\\
0.127792	1.33671e-13	\\
0.128534	1.42109e-13	\\
0.129167	1.43441e-13	\\
0.129774	1.54987e-13	\\
0.130294	1.44773e-13	\\
0.130818	1.43885e-13	\\
0.131487	1.33227e-13	\\
0.132056	1.44773e-13	\\
0.132644	1.56763e-13	\\
0.133001	1.48326e-13	\\
0.133666	1.57652e-13	\\
0.134361	1.57652e-13	\\
0.13507	1.64757e-13	\\
0.135588	1.50546e-13	\\
0.135991	1.49658e-13	\\
0.13657	1.58984e-13	\\
0.13714	1.53211e-13	\\
0.13769	1.54543e-13	\\
0.138229	1.66089e-13	\\
0.138825	1.58984e-13	\\
0.139339	1.62093e-13	\\
0.140146	1.55431e-13	\\
0.140895	1.58096e-13	\\
0.141373	1.54099e-13	\\
0.141897	1.51434e-13	\\
0.142535	1.54987e-13	\\
0.143589	1.51879e-13	\\
0.144197	1.5099e-13	\\
0.144743	1.49214e-13	\\
0.145248	1.57208e-13	\\
0.145703	1.5099e-13	\\
0.146385	1.34559e-13	\\
0.147035	1.38556e-13	\\
0.147801	1.36779e-13	\\
0.148257	1.40776e-13	\\
0.149005	1.42553e-13	\\
0.149751	1.36779e-13	\\
0.150789	1.3145e-13	\\
0.151369	1.38556e-13	\\
0.15236	1.35891e-13	\\
0.153206	1.43441e-13	\\
0.153923	1.3145e-13	\\
0.154616	1.33671e-13	\\
0.155702	1.33671e-13	\\
0.156276	1.23013e-13	\\
0.157109	1.29674e-13	\\
0.157913	1.12799e-13	\\
0.158792	1.0969e-13	\\
0.159332	1.1191e-13	\\
0.160076	1.18128e-13	\\
0.160613	1.10134e-13	\\
0.161543	1.06581e-13	\\
0.161952	1.08358e-13	\\
0.162943	9.54792e-14	\\
0.163795	1.03029e-13	\\
0.164577	9.90319e-14	\\
0.165563	9.81437e-14	\\
0.166526	9.50351e-14	\\
0.167275	8.61533e-14	\\
0.167903	7.9492e-14	\\
0.168356	8.74856e-14	\\
0.169313	8.08242e-14	\\
0.170372	7.28306e-14	\\
0.171347	6.88338e-14	\\
0.172157	6.92779e-14	\\
0.172836	5.90639e-14	\\
0.173425	5.68434e-14	\\
0.173801	5.86198e-14	\\
0.174706	5.06262e-14	\\
0.175346	6.26166e-14	\\
0.176423	6.30607e-14	\\
0.177117	5.72875e-14	\\
0.178062	4.92939e-14	\\
0.178898	3.41949e-14	\\
0.180102	3.86358e-14	\\
0.180862	3.73035e-14	\\
0.182027	3.37508e-14	\\
0.184089	3.59712e-14	\\
0.18487	3.24185e-14	\\
0.185944	2.79776e-14	\\
0.18661	2.62013e-14	\\
0.187504	1.73195e-14	\\
0.188082	1.06581e-14	\\
0.188911	1.42109e-14	\\
0.189884	1.37668e-14	\\
0.190693	1.24345e-14	\\
0.191648	9.76996e-15	\\
0.19297	7.10543e-15	\\
0.194005	1.64313e-14	\\
0.195319	8.88178e-16	\\
0.19641	6.66134e-15	\\
0.197237	4.88498e-15	\\
0.198188	7.10543e-15	\\
0.198714	0	\\
0.199929	2.66454e-15	\\
0.200903	4.44089e-15	\\
0.201795	7.99361e-15	\\
0.202715	5.77316e-15	\\
0.203536	3.9968e-15	\\
0.204572	9.76996e-15	\\
0.205426	9.76996e-15	\\
0.207216	3.10862e-15	\\
0.207983	3.10862e-15	\\
0.208602	1.9984e-15	\\
0.209736	1.11022e-15	\\
0.210653	6.66134e-15	\\
0.211348	1.9984e-15	\\
0.212512	1.55431e-15	\\
0.213482	5.9952e-15	\\
0.214593	6.66134e-16	\\
0.215334	8.21565e-15	\\
0.216055	1.02141e-14	\\
0.217124	7.99361e-15	\\
0.21783	1.15463e-14	\\
0.219062	6.21725e-15	\\
0.219841	8.65974e-15	\\
0.220808	1.77636e-14	\\
0.222028	1.84297e-14	\\
0.222801	2.93099e-14	\\
0.223564	1.77636e-14	\\
0.224473	2.77556e-14	\\
0.225435	2.9754e-14	\\
0.226929	3.24185e-14	\\
0.228105	3.84137e-14	\\
0.229383	3.79696e-14	\\
0.230274	4.77396e-14	\\
0.231502	4.79616e-14	\\
0.23291	6.57252e-14	\\
0.233592	6.28386e-14	\\
0.234564	6.30607e-14	\\
0.235642	6.55032e-14	\\
0.237129	7.21645e-14	\\
0.238486	7.86038e-14	\\
0.239473	8.43769e-14	\\
0.240467	8.9706e-14	\\
0.241136	1.02363e-13	\\
0.242537	1.01252e-13	\\
0.244035	1.10134e-13	\\
0.245076	1.16573e-13	\\
0.246009	1.26565e-13	\\
0.247	1.22347e-13	\\
0.247885	1.35225e-13	\\
0.249225	1.4877e-13	\\
0.250428	1.59428e-13	\\
0.251459	1.64535e-13	\\
0.252329	1.66978e-13	\\
0.253517	1.72085e-13	\\
0.254847	1.78302e-13	\\
0.255904	1.95843e-13	\\
0.257098	2.01617e-13	\\
0.258282	2.07168e-13	\\
0.259441	2.18936e-13	\\
0.260833	2.38254e-13	\\
0.262131	2.42029e-13	\\
0.262987	2.52021e-13	\\
0.263852	2.50022e-13	\\
0.264951	2.62901e-13	\\
0.266916	2.74003e-13	\\
0.268033	2.85327e-13	\\
0.269649	2.97096e-13	\\
0.271025	3.14415e-13	\\
0.272281	3.25073e-13	\\
0.274193	3.43281e-13	\\
0.275028	3.51497e-13	\\
0.276161	3.58602e-13	\\
0.27739	3.75477e-13	\\
0.278326	3.80362e-13	\\
0.279306	3.93685e-13	\\
0.280599	3.97682e-13	\\
0.281796	4.09006e-13	\\
0.282404	4.1811e-13	\\
0.282916	4.24771e-13	\\
0.284345	4.43423e-13	\\
0.286539	4.61409e-13	\\
0.288465	4.865e-13	\\
0.289577	5.01155e-13	\\
0.290789	5.04929e-13	\\
0.291891	5.13367e-13	\\
0.292729	5.34905e-13	\\
0.293549	5.278e-13	\\
0.294692	5.4623e-13	\\
0.29527	5.52669e-13	\\
0.296798	5.69989e-13	\\
0.298342	5.74873e-13	\\
0.299756	5.92415e-13	\\
0.301579	6.17506e-13	\\
0.302548	6.22835e-13	\\
0.304394	6.50591e-13	\\
0.306557	6.6458e-13	\\
0.308453	6.90337e-13	\\
0.310287	7.17648e-13	\\
0.311845	7.2542e-13	\\
0.313014	7.2764e-13	\\
0.314818	7.66054e-13	\\
0.316266	7.78266e-13	\\
0.317626	7.97362e-13	\\
0.319309	7.9714e-13	\\
0.320928	8.31113e-13	\\
0.322521	8.47544e-13	\\
0.323438	8.58869e-13	\\
0.324975	8.67084e-13	\\
0.32696	8.92619e-13	\\
0.328393	9.07052e-13	\\
0.329228	9.15712e-13	\\
0.330537	9.24372e-13	\\
0.331616	9.50351e-13	\\
0.332773	9.53015e-13	\\
0.334735	9.78995e-13	\\
0.335838	9.92539e-13	\\
0.336532	9.98313e-13	\\
0.338708	1.02385e-12	\\
0.339764	1.02696e-12	\\
0.340906	1.03739e-12	\\
0.342263	1.06359e-12	\\
0.344496	1.07891e-12	\\
0.346563	1.10134e-12	\\
0.348333	1.12266e-12	\\
0.349845	1.13087e-12	\\
0.352231	1.15707e-12	\\
0.353715	1.16951e-12	\\
0.355412	1.18305e-12	\\
0.35655	1.20548e-12	\\
0.358077	1.2228e-12	\\
0.360099	1.2379e-12	\\
0.361818	1.25411e-12	\\
0.364139	1.28741e-12	\\
0.36546	1.28786e-12	\\
0.367669	1.3205e-12	\\
0.369975	1.33937e-12	\\
0.371557	1.35825e-12	\\
0.372694	1.37024e-12	\\
0.373774	1.37979e-12	\\
0.375262	1.39533e-12	\\
0.377306	1.41287e-12	\\
0.380459	1.43685e-12	\\
0.382933	1.45306e-12	\\
0.384678	1.47438e-12	\\
0.385902	1.48148e-12	\\
0.388727	1.51146e-12	\\
0.391005	1.53433e-12	\\
0.393414	1.54587e-12	\\
0.395144	1.56453e-12	\\
0.396611	1.57274e-12	\\
0.398773	1.58984e-12	\\
0.400663	1.60294e-12	\\
0.403071	1.6287e-12	\\
0.404874	1.63292e-12	\\
0.406304	1.64624e-12	\\
0.407502	1.65179e-12	\\
0.408969	1.66911e-12	\\
0.410986	1.68265e-12	\\
0.413545	1.70441e-12	\\
0.415279	1.70464e-12	\\
0.418128	1.72928e-12	\\
0.420064	1.74594e-12	\\
0.421823	1.75526e-12	\\
0.424593	1.77169e-12	\\
0.426398	1.77658e-12	\\
0.428145	1.809e-12	\\
0.430117	1.81277e-12	\\
0.431719	1.81366e-12	\\
0.434004	1.82454e-12	\\
0.435639	1.83098e-12	\\
0.438002	1.84674e-12	\\
0.44003	1.85896e-12	\\
0.441192	1.8654e-12	\\
0.444417	1.87694e-12	\\
0.446955	1.87761e-12	\\
0.44909	1.88538e-12	\\
0.451835	1.90559e-12	\\
0.453309	1.90581e-12	\\
0.454813	1.92135e-12	\\
0.457958	1.9329e-12	\\
0.459862	1.94178e-12	\\
0.462247	1.94578e-12	\\
0.464619	1.95022e-12	\\
0.465952	1.94533e-12	\\
0.467532	1.96154e-12	\\
0.468681	1.96598e-12	\\
0.471989	1.97287e-12	\\
0.47471	1.98508e-12	\\
0.476291	1.9762e-12	\\
0.477826	1.98486e-12	\\
0.479499	1.98996e-12	\\
0.482866	1.99374e-12	\\
0.485005	1.99973e-12	\\
0.486881	1.99596e-12	\\
0.489435	2.00484e-12	\\
0.491926	1.99862e-12	\\
0.495021	2.00506e-12	\\
0.498525	2.00973e-12	\\
0.50063	2.01328e-12	\\
0.504166	2.01994e-12	\\
0.505665	2.01417e-12	\\
0.507841	2.00817e-12	\\
0.510741	2.00773e-12	\\
0.512662	2.01017e-12	\\
0.514519	2.01883e-12	\\
0.517737	2.01128e-12	\\
0.520304	2.00862e-12	\\
0.522653	2.0064e-12	\\
0.525222	2.00395e-12	\\
0.527856	1.99774e-12	\\
0.529053	1.99818e-12	\\
0.530345	1.9984e-12	\\
0.532671	1.99152e-12	\\
0.537014	1.97886e-12	\\
0.539905	1.97442e-12	\\
0.542539	1.97686e-12	\\
0.545665	1.97309e-12	\\
0.548282	1.96265e-12	\\
0.55136	1.95954e-12	\\
0.553736	1.95199e-12	\\
0.557423	1.93978e-12	\\
0.559872	1.93556e-12	\\
0.563057	1.91713e-12	\\
0.566709	1.91047e-12	\\
0.568692	1.90625e-12	\\
0.570944	1.90759e-12	\\
0.572341	1.89027e-12	\\
0.576713	1.87095e-12	\\
0.578029	1.87339e-12	\\
0.58042	1.86162e-12	\\
0.583278	1.8483e-12	\\
0.58787	1.82721e-12	\\
0.591899	1.81566e-12	\\
0.596812	1.79123e-12	\\
0.598832	1.78368e-12	\\
0.602077	1.77325e-12	\\
0.603788	1.76237e-12	\\
0.605351	1.75615e-12	\\
0.609567	1.7244e-12	\\
0.612397	1.70619e-12	\\
0.613894	1.71596e-12	\\
0.617595	1.68576e-12	\\
0.618871	1.68177e-12	\\
0.621349	1.67799e-12	\\
0.623221	1.66245e-12	\\
0.625753	1.65556e-12	\\
0.630458	1.62381e-12	\\
0.631941	1.61604e-12	\\
0.636069	1.59561e-12	\\
0.638582	1.57696e-12	\\
0.642931	1.55476e-12	\\
0.646184	1.53255e-12	\\
0.649208	1.52012e-12	\\
0.651934	1.49192e-12	\\
0.655829	1.47482e-12	\\
0.659188	1.45617e-12	\\
0.661905	1.44018e-12	\\
0.664688	1.42752e-12	\\
0.668699	1.39644e-12	\\
0.673064	1.36824e-12	\\
0.675007	1.36025e-12	\\
0.677923	1.33427e-12	\\
0.680482	1.32983e-12	\\
0.683106	1.31584e-12	\\
0.686506	1.2883e-12	\\
0.68944	1.26565e-12	\\
0.695155	1.23412e-12	\\
0.699714	1.20415e-12	\\
0.701674	1.18394e-12	\\
0.705204	1.16906e-12	\\
0.708283	1.15041e-12	\\
0.710861	1.1362e-12	\\
0.713441	1.10778e-12	\\
0.716795	1.09623e-12	\\
0.718886	1.07514e-12	\\
0.722192	1.05382e-12	\\
0.725998	1.04183e-12	\\
0.730187	1.00897e-12	\\
0.73199	9.96758e-13	\\
0.736732	9.70557e-13	\\
0.741189	9.42801e-13	\\
0.743401	9.22595e-13	\\
0.74612	9.07496e-13	\\
0.750475	8.75966e-13	\\
0.753352	8.57314e-13	\\
0.755676	8.51541e-13	\\
0.757967	8.37552e-13	\\
0.762502	8.08464e-13	\\
0.766831	7.82929e-13	\\
0.771437	7.52287e-13	\\
0.77395	7.32747e-13	\\
0.776972	7.14095e-13	\\
0.781398	6.94111e-13	\\
0.784609	6.72795e-13	\\
0.787748	6.62581e-13	\\
0.791038	6.45484e-13	\\
0.794143	6.21725e-13	\\
0.79783	5.99298e-13	\\
0.802238	5.71321e-13	\\
0.806834	5.52669e-13	\\
0.809551	5.41345e-13	\\
0.812533	5.19806e-13	\\
0.81617	5.07372e-13	\\
0.820285	4.82947e-13	\\
0.824309	4.63185e-13	\\
0.826707	4.48974e-13	\\
0.833319	4.18332e-13	\\
0.835586	4.02123e-13	\\
0.838401	3.97238e-13	\\
0.842981	3.76588e-13	\\
0.847488	3.50386e-13	\\
0.851709	3.34843e-13	\\
0.855601	3.1819e-13	\\
0.860051	2.96874e-13	\\
0.864038	2.77334e-13	\\
0.867779	2.67786e-13	\\
0.870037	2.6823e-13	\\
0.87366	2.43805e-13	\\
0.880088	2.18936e-13	\\
0.884857	2.02505e-13	\\
0.889037	1.86517e-13	\\
0.893547	1.65201e-13	\\
0.896011	1.67866e-13	\\
0.900526	1.4766e-13	\\
0.903659	1.42775e-13	\\
0.907433	1.35669e-13	\\
0.912219	1.23457e-13	\\
0.913714	1.07914e-13	\\
0.915829	1.09912e-13	\\
0.918211	1.06581e-13	\\
0.922562	8.9484e-14	\\
0.926574	7.92699e-14	\\
0.93001	7.23865e-14	\\
0.939074	5.9952e-14	\\
0.94474	4.64073e-14	\\
0.947393	3.88578e-14	\\
0.951158	3.73035e-14	\\
0.957049	2.53131e-14	\\
0.963964	1.53211e-14	\\
0.967383	1.46549e-14	\\
0.969213	1.24345e-14	\\
0.974442	9.76996e-15	\\
0.979112	3.9968e-15	\\
0.985362	1.77636e-15	\\
0.987692	5.10703e-15	\\
0.991722	1.11022e-15	\\
0.996869	5.32907e-15	\\
1.00069	2.9976e-15	\\
1.00686	4.21885e-15	\\
1.0114	4.44089e-15	\\
1.01564	3.55271e-15	\\
1.01883	7.77156e-15	\\
1.02234	4.66294e-15	\\
1.02492	7.88258e-15	\\
1.02894	1.34337e-14	\\
1.03158	1.44329e-14	\\
1.0349	1.72085e-14	\\
1.04362	2.44249e-14	\\
1.04669	3.94129e-14	\\
1.05053	4.07452e-14	\\
1.05595	4.76286e-14	\\
1.06253	5.66214e-14	\\
1.06742	6.75016e-14	\\
1.07338	8.08242e-14	\\
1.07781	9.23706e-14	\\
1.08253	1.05804e-13	\\
1.08601	1.13132e-13	\\
1.09	1.26454e-13	\\
1.09294	1.31006e-13	\\
1.09738	1.45772e-13	\\
1.102	1.60538e-13	\\
1.10895	1.86628e-13	\\
1.11633	2.07279e-13	\\
1.12186	2.31148e-13	\\
1.12619	2.5091e-13	\\
1.1329	2.7045e-13	\\
1.13615	2.92988e-13	\\
1.14014	3.03424e-13	\\
1.14299	3.16525e-13	\\
1.14711	3.3118e-13	\\
1.15176	3.51608e-13	\\
1.15845	3.85691e-13	\\
1.16166	3.9535e-13	\\
1.16882	4.36651e-13	\\
1.1752	4.72067e-13	\\
1.17957	4.94826e-13	\\
1.18216	5.10147e-13	\\
1.19114	5.56e-13	\\
1.19497	5.77871e-13	\\
1.20001	6.09957e-13	\\
1.20526	6.37046e-13	\\
1.21056	6.6791e-13	\\
1.21672	7.08766e-13	\\
1.22073	7.43405e-13	\\
1.23028	7.97473e-13	\\
1.23784	8.54095e-13	\\
1.24459	8.9595e-13	\\
1.24788	9.16822e-13	\\
1.25128	9.43134e-13	\\
1.25545	9.77884e-13	\\
1.25793	9.94649e-13	\\
1.26242	1.02296e-12	\\
1.26925	1.07414e-12	\\
1.27407	1.11333e-12	\\
1.27764	1.14297e-12	\\
1.28445	1.19904e-12	\\
1.28897	1.23146e-12	\\
1.29525	1.28286e-12	\\
1.3049	1.35969e-12	\\
1.30871	1.39033e-12	\\
1.31864	1.47737e-12	\\
1.32815	1.56175e-12	\\
1.34101	1.67544e-12	\\
1.3439	1.70297e-12	\\
1.34954	1.75215e-12	\\
1.35348	1.79001e-12	\\
1.35862	1.83853e-12	\\
1.36647	1.90969e-12	\\
1.37219	1.96809e-12	\\
1.3777	2.02005e-12	\\
1.38335	2.06846e-12	\\
1.38883	2.12508e-12	\\
1.39808	2.21334e-12	\\
1.40462	2.28129e-12	\\
1.40889	2.32259e-12	\\
1.41678	2.4033e-12	\\
1.42489	2.48512e-12	\\
1.4286	2.52576e-12	\\
1.43541	2.59204e-12	\\
1.44356	2.67986e-12	\\
1.447	2.72182e-12	\\
1.45273	2.78078e-12	\\
1.45783	2.8324e-12	\\
1.46261	2.89524e-12	\\
1.46644	2.92921e-12	\\
1.47306	3.00093e-12	\\
1.48112	3.09386e-12	\\
1.48631	3.14704e-12	\\
1.49152	3.20344e-12	\\
1.498	3.27627e-12	\\
1.50889	3.40517e-12	\\
1.51635	3.49287e-12	\\
1.52411	3.57148e-12	\\
1.53298	3.68228e-12	\\
1.53867	3.74345e-12	\\
1.54339	3.80262e-12	\\
1.55362	3.92253e-12	\\
1.56628	4.06786e-12	\\
1.57436	4.16789e-12	\\
1.58248	4.25993e-12	\\
1.58694	4.31133e-12	\\
1.59426	4.40314e-12	\\
1.6016	4.48841e-12	\\
1.60963	4.58922e-12	\\
1.61845	4.69391e-12	\\
1.62336	4.75486e-12	\\
1.63086	4.84757e-12	\\
1.63692	4.91773e-12	\\
1.64619	5.03309e-12	\\
1.65239	5.10958e-12	\\
1.65937	5.19373e-12	\\
1.665	5.26468e-12	\\
1.67189	5.34794e-12	\\
1.67556	5.39113e-12	\\
1.68271	5.48339e-12	\\
1.68987	5.56633e-12	\\
1.69877	5.67901e-12	\\
1.70866	5.79969e-12	\\
1.71442	5.87497e-12	\\
1.72011	5.94469e-12	\\
1.72704	6.03562e-12	\\
1.73769	6.1634e-12	\\
1.74769	6.29174e-12	\\
1.75454	6.37357e-12	\\
1.76234	6.47304e-12	\\
1.77118	6.58362e-12	\\
1.78279	6.72884e-12	\\
1.79008	6.82121e-12	\\
1.80116	6.95677e-12	\\
1.80927	7.05502e-12	\\
1.81677	7.14717e-12	\\
1.8269	7.2734e-12	\\
1.83609	7.38765e-12	\\
1.84246	7.46836e-12	\\
1.84909	7.55074e-12	\\
1.86067	7.69307e-12	\\
1.86797	7.78633e-12	\\
1.87509	7.87126e-12	\\
1.88462	7.9845e-12	\\
1.89612	8.12617e-12	\\
1.90382	8.22153e-12	\\
1.91835	8.39417e-12	\\
1.92937	8.53306e-12	\\
1.94517	8.73168e-12	\\
1.95358	8.82505e-12	\\
1.96322	8.94418e-12	\\
1.97067	9.027e-12	\\
1.97847	9.11604e-12	\\
1.99118	9.26592e-12	\\
1.99933	9.3644e-12	\\
2.00577	9.44267e-12	\\
2.01732	9.5729e-12	\\
2.02421	9.6525e-12	\\
2.03162	9.74198e-12	\\
2.0416	9.85489e-12	\\
2.05159	9.97047e-12	\\
2.06066	1.00712e-11	\\
2.07203	1.01995e-11	\\
2.07753	1.02629e-11	\\
2.08707	1.0372e-11	\\
2.09452	1.04557e-11	\\
2.10783	1.0601e-11	\\
2.1198	1.07339e-11	\\
2.12924	1.084e-11	\\
2.14501	1.1011e-11	\\
2.15928	1.11615e-11	\\
2.17265	1.13018e-11	\\
2.19081	1.14989e-11	\\
2.2014	1.16035e-11	\\
2.21006	1.1696e-11	\\
2.21576	1.1753e-11	\\
2.22677	1.18657e-11	\\
2.23651	1.19607e-11	\\
2.25202	1.21199e-11	\\
2.26277	1.2227e-11	\\
2.27758	1.23712e-11	\\
2.2885	1.24784e-11	\\
2.2988	1.25746e-11	\\
2.30861	1.26662e-11	\\
2.31522	1.273e-11	\\
2.32688	1.28437e-11	\\
2.34351	1.29938e-11	\\
2.35324	1.30848e-11	\\
2.36576	1.31954e-11	\\
2.36978	1.32325e-11	\\
2.3798	1.33268e-11	\\
2.39165	1.34273e-11	\\
2.40418	1.35355e-11	\\
2.41219	1.36097e-11	\\
2.42786	1.37406e-11	\\
2.43743	1.38193e-11	\\
2.44472	1.3879e-11	\\
2.45551	1.39715e-11	\\
2.46508	1.40458e-11	\\
2.47591	1.41339e-11	\\
2.49047	1.42509e-11	\\
2.50494	1.43621e-11	\\
2.51629	1.44516e-11	\\
2.53182	1.4568e-11	\\
2.54063	1.46316e-11	\\
2.55441	1.47347e-11	\\
2.56632	1.48196e-11	\\
2.57583	1.48938e-11	\\
2.59765	1.50423e-11	\\
2.61636	1.51688e-11	\\
2.62939	1.52577e-11	\\
2.64229	1.53418e-11	\\
2.65063	1.53925e-11	\\
2.66488	1.54871e-11	\\
2.67204	1.55325e-11	\\
2.68853	1.56316e-11	\\
2.70159	1.57151e-11	\\
2.71485	1.57928e-11	\\
2.72522	1.58543e-11	\\
2.73163	1.58884e-11	\\
2.74114	1.59452e-11	\\
2.75287	1.60065e-11	\\
2.75762	1.60351e-11	\\
2.76946	1.60979e-11	\\
2.79324	1.62236e-11	\\
2.80245	1.6267e-11	\\
2.82032	1.63622e-11	\\
2.83198	1.64224e-11	\\
2.84039	1.64606e-11	\\
2.85568	1.65342e-11	\\
2.87357	1.66144e-11	\\
2.88426	1.66656e-11	\\
2.90311	1.67489e-11	\\
2.91433	1.6789e-11	\\
2.91981	1.68112e-11	\\
2.93285	1.68691e-11	\\
2.94845	1.69308e-11	\\
2.95951	1.697e-11	\\
2.97174	1.70203e-11	\\
2.98385	1.70626e-11	\\
3.00019	1.71191e-11	\\
3.00996	1.71544e-11	\\
3.02438	1.72022e-11	\\
3.03676	1.72435e-11	\\
3.05065	1.72826e-11	\\
3.06015	1.7316e-11	\\
3.07312	1.73536e-11	\\
3.09063	1.74022e-11	\\
3.10426	1.74419e-11	\\
3.11856	1.74754e-11	\\
3.13097	1.75073e-11	\\
3.14182	1.75362e-11	\\
3.15581	1.75692e-11	\\
3.17855	1.76208e-11	\\
3.1944	1.7649e-11	\\
3.20509	1.7676e-11	\\
3.21843	1.77001e-11	\\
3.23677	1.77344e-11	\\
3.25536	1.77667e-11	\\
3.27116	1.77911e-11	\\
3.28145	1.78052e-11	\\
3.28776	1.78119e-11	\\
3.29755	1.78253e-11	\\
3.31668	1.78474e-11	\\
3.33641	1.78755e-11	\\
3.35121	1.78904e-11	\\
3.36002	1.78987e-11	\\
3.37035	1.79071e-11	\\
3.38202	1.79171e-11	\\
3.40333	1.7935e-11	\\
3.41024	1.7937e-11	\\
3.42871	1.79476e-11	\\
3.4392	1.79537e-11	\\
3.45	1.79564e-11	\\
3.47698	1.79648e-11	\\
3.48591	1.79671e-11	\\
3.49916	1.797e-11	\\
3.51073	1.79702e-11	\\
3.53144	1.79682e-11	\\
3.54278	1.79672e-11	\\
3.55771	1.79678e-11	\\
3.56828	1.7963e-11	\\
3.59019	1.79598e-11	\\
3.61177	1.79456e-11	\\
3.62231	1.79416e-11	\\
3.63224	1.79353e-11	\\
3.64821	1.79211e-11	\\
3.67865	1.79043e-11	\\
3.6901	1.78855e-11	\\
3.70816	1.78699e-11	\\
3.72825	1.78529e-11	\\
3.74526	1.7833e-11	\\
3.75968	1.78146e-11	\\
3.784	1.77846e-11	\\
3.79747	1.77672e-11	\\
3.81243	1.77455e-11	\\
3.82783	1.77212e-11	\\
3.84763	1.76921e-11	\\
3.85952	1.76748e-11	\\
3.87565	1.76468e-11	\\
3.90046	1.76028e-11	\\
3.91193	1.75825e-11	\\
3.93007	1.75467e-11	\\
3.9543	1.7496e-11	\\
3.97897	1.74456e-11	\\
3.99471	1.74139e-11	\\
4.01354	1.73697e-11	\\
4.02971	1.73319e-11	\\
4.0409	1.73085e-11	\\
4.05891	1.72625e-11	\\
4.07359	1.72296e-11	\\
4.08099	1.72107e-11	\\
4.10528	1.71484e-11	\\
4.1273	1.70935e-11	\\
4.14039	1.7058e-11	\\
4.1655	1.69889e-11	\\
4.18578	1.69341e-11	\\
4.21003	1.68633e-11	\\
4.23799	1.67809e-11	\\
4.25882	1.67222e-11	\\
4.27918	1.66573e-11	\\
4.2957	1.66087e-11	\\
4.31762	1.65382e-11	\\
4.33062	1.65005e-11	\\
4.35793	1.64062e-11	\\
4.37829	1.634e-11	\\
4.4087	1.62384e-11	\\
4.42985	1.61684e-11	\\
4.44854	1.61029e-11	\\
4.46135	1.60572e-11	\\
4.48603	1.59703e-11	\\
4.51	1.58833e-11	\\
4.53141	1.58052e-11	\\
4.55237	1.57283e-11	\\
4.5749	1.5649e-11	\\
4.59724	1.5562e-11	\\
4.60872	1.55172e-11	\\
4.62886	1.54402e-11	\\
4.64609	1.53781e-11	\\
4.70087	1.51619e-11	\\
4.72547	1.50652e-11	\\
4.74041	1.50047e-11	\\
4.75	1.49663e-11	\\
4.7787	1.48516e-11	\\
4.78755	1.48146e-11	\\
4.80747	1.47329e-11	\\
4.83792	1.46094e-11	\\
4.85137	1.45515e-11	\\
4.87628	1.44517e-11	\\
4.89058	1.43924e-11	\\
4.901	1.43482e-11	\\
4.91963	1.42683e-11	\\
4.94708	1.41528e-11	\\
4.96503	1.40758e-11	\\
4.98023	1.40106e-11	\\
5.00062	1.39257e-11	\\
5.03209	1.37875e-11	\\
5.04733	1.37206e-11	\\
5.05957	1.36723e-11	\\
5.089	1.35421e-11	\\
5.11273	1.34374e-11	\\
5.139	1.33223e-11	\\
5.17036	1.31836e-11	\\
5.20238	1.30441e-11	\\
5.22068	1.29586e-11	\\
5.24171	1.28664e-11	\\
5.27082	1.27369e-11	\\
5.29698	1.26228e-11	\\
5.3122	1.2555e-11	\\
5.33363	1.24574e-11	\\
5.35787	1.23468e-11	\\
5.38709	1.2218e-11	\\
5.41346	1.20989e-11	\\
5.4475	1.19463e-11	\\
5.48522	1.17766e-11	\\
5.50637	1.16816e-11	\\
5.5248	1.15953e-11	\\
5.55313	1.14712e-11	\\
5.57634	1.13644e-11	\\
5.60366	1.12444e-11	\\
5.62191	1.11616e-11	\\
5.65958	1.09901e-11	\\
5.69471	1.08324e-11	\\
5.71707	1.07314e-11	\\
5.73437	1.06551e-11	\\
5.75999	1.05388e-11	\\
5.78241	1.04381e-11	\\
5.80886	1.03184e-11	\\
5.82328	1.02559e-11	\\
5.85852	1.00998e-11	\\
5.88786	9.9713e-12	\\
5.917	9.83963e-12	\\
5.93246	9.77268e-12	\\
5.96588	9.62608e-12	\\
5.99341	9.50201e-12	\\
6.011	9.42607e-12	\\
6.03999	9.29562e-12	\\
6.06491	9.18737e-12	\\
6.09937	9.03744e-12	\\
6.12948	8.9091e-12	\\
6.16636	8.74889e-12	\\
6.19125	8.64103e-12	\\
6.21746	8.53273e-12	\\
6.24477	8.41421e-12	\\
6.26611	8.32334e-12	\\
6.28719	8.23441e-12	\\
6.3129	8.12361e-12	\\
6.34627	7.985e-12	\\
6.37139	7.88053e-12	\\
6.39363	7.78777e-12	\\
6.41025	7.71794e-12	\\
6.43865	7.60325e-12	\\
6.47916	7.4345e-12	\\
6.5102	7.30943e-12	\\
6.52663	7.2452e-12	\\
6.57432	7.05053e-12	\\
6.61072	6.90858e-12	\\
6.63516	6.81e-12	\\
6.68209	6.62498e-12	\\
6.70497	6.53588e-12	\\
6.7332	6.42708e-12	\\
6.76038	6.32233e-12	\\
6.79327	6.19754e-12	\\
6.82495	6.07669e-12	\\
6.86837	5.91449e-12	\\
6.88989	5.83467e-12	\\
6.91276	5.75168e-12	\\
6.93727	5.6587e-12	\\
6.99222	5.46002e-12	\\
7.0143	5.38336e-12	\\
7.03642	5.3027e-12	\\
7.07477	5.16809e-12	\\
7.09945	5.08193e-12	\\
7.12214	5.00411e-12	\\
7.15202	4.89947e-12	\\
7.18136	4.80155e-12	\\
7.22839	4.64001e-12	\\
7.29291	4.42768e-12	\\
7.32937	4.30905e-12	\\
7.34602	4.25687e-12	\\
7.39887	4.08718e-12	\\
7.42534	4.00174e-12	\\
7.45516	3.90904e-12	\\
7.48223	3.82699e-12	\\
7.51391	3.73285e-12	\\
7.56059	3.59068e-12	\\
7.6048	3.46412e-12	\\
7.62181	3.41371e-12	\\
7.64773	3.34016e-12	\\
7.69234	3.21093e-12	\\
7.73359	3.0978e-12	\\
7.77596	2.98084e-12	\\
7.82277	2.85622e-12	\\
7.86155	2.75263e-12	\\
7.89303	2.67292e-12	\\
7.93264	2.57044e-12	\\
7.97912	2.45548e-12	\\
8.00993	2.38021e-12	\\
8.05018	2.28378e-12	\\
8.09962	2.16871e-12	\\
8.13586	2.08428e-12	\\
8.17892	1.98941e-12	\\
8.2152	1.90908e-12	\\
8.24147	1.85274e-12	\\
8.27699	1.7768e-12	\\
8.31588	1.69642e-12	\\
8.36978	1.58829e-12	\\
8.39427	1.54088e-12	\\
8.43448	1.46161e-12	\\
8.47906	1.37979e-12	\\
8.50451	1.33477e-12	\\
8.52412	1.29807e-12	\\
8.57725	1.20398e-12	\\
8.60789	1.15369e-12	\\
8.6428	1.09779e-12	\\
8.70304	9.97702e-13	\\
8.76292	9.08162e-13	\\
8.79102	8.64808e-13	\\
8.8193	8.23897e-13	\\
8.86518	7.61002e-13	\\
8.91599	6.94056e-13	\\
8.96533	6.31384e-13	\\
8.99896	5.88363e-13	\\
9.03091	5.51226e-13	\\
9.06498	5.1309e-13	\\
9.11212	4.62408e-13	\\
9.15748	4.14502e-13	\\
9.20151	3.73757e-13	\\
9.2463	3.3179e-13	\\
9.30386	2.81108e-13	\\
9.35036	2.44471e-13	\\
9.41429	1.99785e-13	\\
9.45702	1.6992e-13	\\
9.49771	1.45384e-13	\\
9.55248	1.15075e-13	\\
9.61597	8.39884e-14	\\
9.68052	5.66214e-14	\\
9.70768	4.84612e-14	\\
9.7569	3.14748e-14	\\
9.79248	2.45359e-14	\\
9.83031	1.42664e-14	\\
9.87647	7.54952e-15	\\
9.90852	4.05231e-15	\\
9.95018	5.55112e-17	\\
};
\end{axis}
\end{tikzpicture}%

        \end{subfigure}
    \end{center}
    \caption{Vergleich des tatsächlichen Fehlers $s(\mu) - s_N(\mu)$ (gestrichelt) und Fehlerschätzer $\Delta^s_N(\mu)$ über $\Xi_\text{test} \subset \mathcal D$ mit $10^4$ Elementen für $N = 1 \ldots 4$.}
    \label{fig:plot_s_fehler}
\end{figure}

Diese wird bereits für $N = 4$ erreicht, wie man an den Ergebnissen in Tabelle \ref{tab:eindim} erkennt.

\begin{table}[tb]
    \begin{center}
        \begin{tabular}{l|cccc}
        $N$ & $\Delta^s_{N,\max}$ & $\eta^s_{N,\text{ave}}$ & $\eta^s_{N,\max}$ & $\rho^s_{\text{err}, N}$ \\
        \hline
            $1$ & $1.7855 \cdot 10^{+00}$ & $1.4809$ & $1.8186$ & $1.8186$ \\
            $2$ & $9.9356 \cdot 10^{-02}$ & $2.2728$ & $8.1925$ & $1.1817$ \\
            $3$ & $2.7957 \cdot 10^{-07}$ & $2.4607$ & $5.5208$ & $2.7146$ \\
            $4$ & $4.3910 \cdot 10^{-11}$ & $2.4864$ & $-$ & $2.4431$ \\
        \end{tabular}
        \caption{Der hohe Wert von $\eta^s_{\max,\text{UB}}$ für $N = 4$ liegt sehr wahrscheinlich an Rechenungenauigkeiten, da dieses Verhalten erst bei $\Delta^s_{N,\max} < 10^{-8}$ zu beobachten ist.}
        \label{tab:eindim}
    \end{center}
\end{table}

Abbildung \ref{fig:plot_s_fehler} zeigt, welche Parameter vom Greedy-Verfahren gewählt werden, und liefert einen visuellen Vergleich zwischen tatsächlichem Fehler und dem Fehlerschätzer.

\subsection{Mehrdimensionale Parameterräume} % (fold)
\label{sub:mehrdimensionale_parameterr_ume}

Wir können das oben betrachtete Problem auch auf höherdimensionale Parameterräume verallgemeinern. Teilen wir $\Omega$ in vier gleichgroße Quadrate statt den zwei Rechtecken ein, dann erhalten wir damit einen dreidimensionalen Parameterraum, zum Beispiel $\mathcal D = [0.1, 10]^3 \subset \mathbb{R}^3$.

Wir wählen nun $\Xi_\text{train} = \Xi_\text{test} \subset \mathcal D$ mit $10^5$ Elementen, um so ein besseres Bild des Fehlers über $\mathcal D$ zu erhalten. Tabelle \ref{tab:dreidim} zeigt die Entwicklung des Fehlers für dieses Beispiel.

\begin{table}[h]
    \begin{center}
        \begin{tabular}{l|cccc}
        $N$ & $\Delta^s_{N,\max}$ & $\eta^s_{N,\text{ave}}$ & $\eta^s_{N,\max}$ & $\rho^s_{\text{err}, N}$ \\
        \hline
            $1$ & $1.7855 \cdot 10^{+00}$ & $1.4809$ & $1.8186$ & $1.8186$ \\
            $2$ & $9.9356 \cdot 10^{-02}$ & $2.2728$ & $8.1925$ & $1.1817$ \\
            $3$ & $2.7957 \cdot 10^{-07}$ & $2.4607$ & $5.5208$ & $2.7146$ \\
            $4$ & $4.3910 \cdot 10^{-11}$ & $2.4864$ & $-$ & $2.4431$ \\
        \end{tabular}
        \caption{Auch hier liegen die hohen Werte von $\eta^s_{\max,\text{UB}}$ für große $N$ sehr wahrscheinlich an Rechenungenauigkeiten.}
        \label{tab:dreidim}
    \end{center}
\end{table}

% subsection mehrdimensionale_parameterr_ume (end)

% paragraph numerische_ergebnisse (end)

% section beispiel (end)
